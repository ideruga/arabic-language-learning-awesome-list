% arabic_rap.tex — formatted XeLaTeX file for the supplied Arabic/English lyrics
% Compile with: xelatex arabic_rap.tex

\documentclass[a4paper,12pt]{article}

\usepackage[utf8]{inputenc}
\usepackage[english,arabic]{babel}
\usepackage{tikz}
\usepackage{geometry}
\usepackage{array}
\usepackage{multicol}
\usepackage{xcolor}
\usepackage[table]{xcolor}
\usepackage{polyglossia}
\setmainlanguage{english}
\setotherlanguage{arabic}
\usepackage{arabtex}
\usepackage{utf8}
\usepackage{fontspec}
\setmainfont{Charis SIL}
\newfontfamily\arabicfont[Script=Arabic,Scale=1.2]{Amiri}
\geometry{margin=1.5cm}
\usepackage[most]{tcolorbox}
\usepackage{booktabs}
\usepackage{graphicx}
\usetikzlibrary{shapes,arrows,decorations.pathmorphing}

% Custom colors
\definecolor{headercolor}{RGB}{70,130,180}
\definecolor{boxcolor}{RGB}{240,248,255}
\definecolor{accentcolor}{RGB}{220,20,60}
\definecolor{tableheader}{RGB}{220,220,220}
\definecolor{dialectcolor}{RGB}{34,139,34}

\begin{document}

\title{\textbf{\Large Oh My God}\\
\large Arabic / English Rap Lyrics}
\author{Formatted for Presentation}
\date{}
\maketitle

% ======================== Lyrics ========================

\begin{tcolorbox}[colback=boxcolor,colframe=headercolor,title=\textbf{Stanza 1},breakable]
Oh My God \\
\textarabic{
لمَ يجب عليّ تعلّم ذي اللغة العربية؟ \\
لمَ أُقحم في مخي لغة لا طائلة منها في الدنيا؟ \\
ما جدوى النطق بكلماتٍ من زمن قريش محكيّة؟ \\
هل نحن بعصر عباسي أم عصر الدولة الأموية؟ \\
في هذي الحقبة الزمنية قل لّي أين هي العربيّة \\
بصروح العلم المبنية لن تجد كتاباً عربيا \\
في عصر نهوض التقنية ما زالت تحبو كصبية \\
حتى إن أَحسَنت النية \\
سمعتها تبدو ضبابية \\
قد برعت في وصف السيف \\
أهدته أغاني طربية \\
ما هذا العنف المتخفي برداء وصوف أدبية؟ \\
قلة أشعارها في الحب والباقي قصائد حربية \\
وكأن الشاعر موعود في الحرب بعبدٍ وسبية \\
ثاني أصعب لغة تنطق من بعد اللغة الصينية \\
أتقارن نفسها بحضارة أبناء الصين الشعبية؟ \\
لا أعلم ما سر صعوبة لغة ليست بأساسية \\
في عالم أرقام الديجتال والمصفوفات الرقمية \\
حدثني عن لغة النصبِ لن أثق بلغة نصابة \\
وحروف جرت بالغصبِ كلمات كان لها مهابة \\
مجهول يبني أفعالها بضمير مستتر غابا \\
فاعلها مرفوع دوما حتى لو يفعل بإنابة \\
}
Oh My God! \\
\textarabic{
للوصف الواحد كلمات أكثر من ألفٍ \\
في نفس الوقت ترى كلمة قد يوصف بها مئة وصف \\
لا أفهم منطق ذي اللغة سأهجرها بلا ذرة أسف \\
وقتي ثمين لن أصرفه عبثاً في نحو أو صرف \\
أتحداك بأن تذكر لي إسهاماتها في الهندسة \\
أو بعلوم فضاء الكون بتلسكوب أو بعدسة \\
كم مخترعا يتحدثها؟ ليس لها أي منافسة \\
لو كانت تسهم في العلم لم نهملها في المدرسة؟ \\
بئس حروفها إذ تتغير حسب مواقعها في الكلمة \\
تنفصل ببعض الكلمات وفي بعضها تبدو ملتئمة \\
لابد لقارئها غصبا أن يجتهد وأن يلتزم \\
بالإعراب فإن لم يفعل لن نسمعه ولن نحترمه \\
}
\end{tcolorbox}

\begin{tcolorbox}[colback=boxcolor,colframe=headercolor,title=\textbf{Stanza 2} ,breakable]
\textarabic{
لا تغتر بلغتك فلغتك اشتقت من بضع لغات \\
من فارس اقتبست أيضاً، من بابل وصل لها زاد \\
حتى الهندية تغذتها أعطتها بعض الكلمات \\
فأرحني من لغة النسخ وسلملي على لغة الضاد \\
}
\end{tcolorbox}

\begin{tcolorbox}[colback=boxcolor,colframe=headercolor,title=\textbf{Chorus (English)},breakable]
Face it, man! Your language's dying \\
Why don't you just get rid of it? \\
Stop whining. No point in crying \\
It's history you need to get ahead of it \\
Oh My God! I'm not even lying \\
I'mma teach my kids how to benefit \\
From a language that keeps'em flying \\
For a better future, for a better fit \\
\end{tcolorbox}

\begin{tcolorbox}[colback=boxcolor,colframe=headercolor,title=\textbf{Stanza 3},breakable]
\textarabic{
لسنا في العصر العباسي ولسنا في الدولة الأموية \\
لكن في عصر شبابٍ يتنصل من أقوى هوية \\
في زمن شباب مهزومٍ منبهرٍ بالإنجليزية \\
وكأن عصورنا في الماضي ما كانت أبداً ذهبية \\
لا عجب بأن تسأل عن سر غياب اللغة العربية \\
أن تتسال عن دورها بدهاليز البحث العلمية \\
وإجابة سؤلك في أنها قد غابت ظلماً وبفرية \\
من أبنائها هم ظلموها واتهموها بالرجعية \\
اللغة تعبر عن قوة أهلها من بين الأمم \\
ألسنه شعوب العالم يفتخر بها منذ القدم \\
راجع تاريخك كي تعرف من هم أسياد الكلم \\
الأدب العربي تربع زمناً في أعلى القمم \\
تتحسس من لغه تتغزل بالسيف وشدة ضربه \\
ما بالك لم تذكر من يتغنى ببطولات الحرب \\
من أممٍ اخرى ولغاتٍ وأولئك من غير العرب؟ \\
الكل يمجد أسلحته من أقصى الشرق إلى الغرب \\
ثاني أصعب لغة تنطق مع ذلك يدرسها الصبية \\
في كل الأقطار تردد بالقرآن لغتنا ستحيا \\
كيف كيف تغيب دور العربي في المصفوفات الرقمية \\
والعالم يكتب أرقاماً خرجت من رحم العربية؟ \\
ما بالك تستهزئ بقواعد لغتك وتشكك فيها؟ \\
تسخر من جرها من نصبها وتركت جمال قوافيها \\
روعة لغتك ذي بينة قدحك هذا لن يخفيها \\
شبهاتك في النقد ركيكة لو أنفخ فيها سأطفيها \\
لم تتشكى من كثرة وتعدد كلمات الوصف؟ \\
الأمر هنا كرداءٍ إذ نلبسه حسب الظرف \\
لا تتوعد برحيلك عن لغتي وبغض الطرف \\
هجرك للغة صديقي لن يفقدها ذرة شرف \\
لو تبحث عن إسهاماتها في العلم ستجد كثيرا \\
لكن للأسف هجرنا العلم وبات العلم أسيرا \\
ما بين فساد في الدول به أضحى الأمر عسيرا \\
وهروب عقول عربية فالفرص هناك وفيرة \\
تتذمّر من شكل حروفها وكأنك طفل في روضة \\
تبحث عن عذر لكسلك لكي تبقى في هذه الفوضى \\
أنا متبرع دعني أدربك بقلمي فضلاً لو ترضى \\
سجل اسمك في مدرستي وبسرعة لا تضع العرض \\
كلمات لغات بني البشر ما كانت حكراً أو حصرا \\
لا توجد لغة في العالم لم تأخذ شيئا من أخرى \\
بدأت تتضح لي الرؤية أظن بأني عرفت الأمر \\
يبدو أن العلة فيك ويبدو أن العلة كبرى \\
}
\end{tcolorbox}

\vspace{12pt}
{\small\textit{Formatted with XeLaTeX. Change the Arabic font if Amiri is not installed on your system.}}

\end{document}
