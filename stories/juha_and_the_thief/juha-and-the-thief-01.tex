\documentclass[a4paper,12pt]{article}

\usepackage[utf8]{inputenc}
\usepackage[english,arabic]{babel}
\usepackage{tikz}
\usepackage{geometry}
\usepackage{array}
\usepackage{multicol}
\usepackage{polyglossia}
\setmainlanguage{english}
\setotherlanguage{arabic}
\usepackage{arabtex}
\usepackage{utf8}
\usepackage{fontspec}
\setmainfont{Charis SIL}
\newfontfamily\arabicfont[Script=Arabic,Scale=1.2]{Amiri}
\geometry{margin=1.5cm}
\usepackage{xcolor}
\usepackage{tcolorbox}
\usepackage{booktabs}
\usepackage{graphicx}
\usetikzlibrary{shapes,arrows,decorations.pathmorphing}

% Custom colors
\definecolor{headercolor}{RGB}{70,130,180}
\definecolor{boxcolor}{RGB}{240,248,255}
\definecolor{accentcolor}{RGB}{220,20,60}

\begin{document}

\title{\textbf{\Large Arabic Learning Poster}\\
\large Juha and the Thief – Full Phrase Analysis\\
\normalsize \textarabic{شَعَرَ جُحا بِلِصٍّ في دَارِهِ لَيْلًا، فَقَامَ إِلَى خِزَانَةِ غُرْفَةِ البُومِ وَاخْتَبَأَ فِيهَا}}
\author{Teaching Material for Arabic Students}
\date{}
\maketitle

% ======================== Phrase Display ========================
\begin{tcolorbox}[colback=boxcolor,colframe=headercolor,title=\textbf{Arabic Text Analysis}]
\centering
\textarabic{شعر جحا بلص في داره ليلا، فقام إلى خزانة غرفة البوم واختبا فيها}
\\[0.5em]
\textbf{Without Diacritics}
\\[1em]
\textarabic{شَعَرَ جُحا بِلِصٍّ في دَارِهِ لَيْلًا، فَقَامَ إِلَى خِزَانَةِ غُرْفَةِ البُومِ وَاخْتَبَأَ فِيهَا}
\\[0.5em]
\textbf{With Full Diacritics}
\end{tcolorbox}

% ======================== Translations ========================
\section{Word-by-Word English Translation}
\begin{tcolorbox}[colback=white,colframe=accentcolor]
\textbf{Literal:} Felt • Juha • with-thief • in • house-his • at-night, • so-rose • to • wardrobe • room-of-owls • and-hid • in-it
\\[0.5em]
\textbf{Adapted:} Juha sensed a thief in his house at night, so he went to the wardrobe of the “owl room” and hid inside it.
\end{tcolorbox}

% ======================== Detailed Word Analysis ========================
\section{Detailed Word Analysis}

% ------- Existing words (kept from first poster) -------
\subsection{\textarabic{شَعَرَ} – \textbf{shaʿara} [ʃaʕara]}
\begin{tabular}{p{3cm}p{10cm}}
\toprule
\textbf{Translation} & felt, sensed, perceived\\
\textbf{Root} & \textarabic{ش-ع-ر} (sh-ʕ-r)\\
\textbf{Pattern} & \textarabic{فَعَلَ} (faʕala) – basic triliteral\\
\textbf{Grammar} & Past-tense verb, 3rd person masc. sing., active\\
\textbf{Construction} & \textarabic{شَعَرَ بِـ} “to feel/sense something”\\
\textbf{Examples} & \textarabic{شَعَرَ بِالْخَوْفِ} (felt fear) [ʃaʕara bil-xawfi], \textarabic{شَعَرَ بِالْفَرَحِ} (felt joy) [ʃaʕara bil-faraḥi]\\
\bottomrule
\end{tabular}

\textbf{Conjugation Examples:}
\begin{multicols}{2}
\small
\textarabic{شَعَرْتُ} – I felt [ʃaʕartu]\\
\textarabic{شَعَرْتَ} – you felt (m.) [ʃaʕarta]\\
\textarabic{شَعَرَتْ} – she felt [ʃaʕarat]\\
\textarabic{أَشْعُرُ} – I feel (pres.) [aʃʕuru]\\
\textarabic{يَشْعُرُ} – he feels (pres.) [jaʃʕuru]\\
\textarabic{تَشْعُرُ} – she feels (pres.) [taʃʕuru]
\end{multicols}

\subsection{\textarabic{جُحا} - \textbf{Juḥā} [d͡ʒuħaː]}

\begin{tabular}{p{3cm}p{10cm}}
\toprule
\textbf{Translation} & Juha (proper name) \\
\textbf{Root} & \textarabic{ج-ح-ا} (j-ḥ-ā) \\
\textbf{Pattern} & \textarabic{فُعَل} (fuʕal) \\
\textbf{Grammar} & Masculine proper noun, subject (\textarabic{مُبْتَدَأ}) \\
\textbf{Cultural Note} & Legendary wise fool character in Middle Eastern folklore \\
\textbf{Examples} & \textarabic{جُحا رَجُلٌ ذَكِيٌّ} (Juha is a smart man) [juḥā rajulun ðakiyyun], \textarabic{قِصَصُ جُحا} (stories of Juha) [qiṣaṣu juḥā] \\
\bottomrule
\end{tabular}

\textbf{Related Names and Variants:}
\begin{multicols}{2}
\small
\textarabic{جُحا} - Juha [d͡ʒuħaː]\\
\textarabic{نَصْرُ الدِّينِ} - Nasruddin [naṣru d-dīni]\\
\textarabic{مُلَّا نَصْرُ الدِّينِ} - Mulla Nasruddin [mullā naṣru d-dīni]\\
\textarabic{خُوجا} - Khoja [xuːd͡ʒa]\\
\textarabic{أَفَنْدِي} - Effendi [afandi]\\
\textarabic{الحَكِيمُ المَجْنُونُ} - the wise madman [al-ḥakīmu l-mad͡ʒnūnu]
\end{multicols}

\subsection{\textarabic{بِلِصٍّ} - \textbf{bi-liṣṣin} [bilissin]}

\begin{tabular}{p{3cm}p{10cm}}
\toprule
\textbf{Translation} & with/by a thief \\
\textbf{Root} & \textarabic{ل-ص-ص} (l-ṣ-ṣ) \\
\textbf{Pattern} & \textarabic{بِ} + \textarabic{فِعْل} (preposition + noun) \\
\textbf{Grammar} & Preposition \textarabic{بِ} + indefinite noun in genitive case \\
\textbf{Simple Form} & \textarabic{لِصٌّ} (a thief) [liṣṣun] \\
\textbf{Examples} & \textarabic{لِصٌّ دَخَلَ الْبَيْتَ} (a thief entered the house) [liṣṣun daxala l-bayta], \textarabic{اللُّصُوصُ} (the thieves) [al-luṣūṣu] \\
\bottomrule
\end{tabular}

\textbf{Declension Examples:}
\begin{multicols}{2}
\small
\textarabic{لِصٌّ} - a thief (nom.) [liṣṣun]\\
\textarabic{لِصًّا} - a thief (acc.) [liṣṣan]\\
\textarabic{لِصٍّ} - a thief (gen.) [liṣṣin]\\
\textarabic{اللِّصُّ} - the thief (nom.) [al-liṣṣu]\\
\textarabic{اللِّصَّ} - the thief (acc.) [al-liṣṣa]\\
\textarabic{اللِّصِّ} - the thief (gen.) [al-liṣṣi]
\end{multicols}

\subsection{\textarabic{في} - \textbf{fī} [fiː]}

\begin{tabular}{p{3cm}p{10cm}}
\toprule
\textbf{Translation} & in, at, within \\
\textbf{Type} & Preposition (\textarabic{حَرْف جَرّ}) \\
\textbf{Function} & Indicates location or position \\
\textbf{Grammar} & Always governs genitive case \\
\textbf{Examples} & \textarabic{في الْبَيْتِ} (in the house) [fī l-bayti], \textarabic{في الْمَدْرَسَةِ} (in the school) [fī l-madrasati] \\
\bottomrule
\end{tabular}

\textbf{Common Expressions with \textarabic{في}:}
\begin{multicols}{2}
\small
\textarabic{في الصَّبَاحِ} - in the morning [fī ṣ-ṣabāḥi]\\
\textarabic{في الظُّهْرِ} - at noon [fī ḏ-ḏuhri]\\
\textarabic{في اللَّيْلِ} - at night [fī l-layli]\\
\textarabic{في السُّوقِ} - in the market [fī s-sūqi]\\
\textarabic{في المَكْتَبِ} - in the office [fī l-maktabi]\\
\textarabic{في الحَدِيقَةِ} - in the garden [fī l-ḥadīqati]
\end{multicols}

\subsection{\textarabic{دَارِهِ} - \textbf{dārihi} [daːrihi]}

\begin{tabular}{p{3cm}p{10cm}}
\toprule
\textbf{Translation} & his house \\
\textbf{Root} & \textarabic{د-و-ر} (d-w-r) \\
\textbf{Pattern} & \textarabic{فَاعِل} + possessive pronoun \textarabic{ـه} \\
\textbf{Grammar} & Feminine noun in genitive case + 3rd person masculine possessive \\
\textbf{Simple Form} & \textarabic{دَارٌ} (a house) [dārun] \\
\textbf{Examples} & \textarabic{دَارُ الرَّجُلِ} (the man's house) [dāru r-rajuli], \textarabic{الدَّارُ كَبِيرَةٌ} (the house is big) [ad-dāru kabīratun] \\
\bottomrule
\end{tabular}

\textbf{Possessive Pronoun Examples:}
\begin{multicols}{2}
\small
\textarabic{دَارِي} - my house [dārī]\\
\textarabic{دَارُكَ} - your house (m.) [dāruka]\\
\textarabic{دَارُكِ} - your house (f.) [dāruki]\\
\textarabic{دَارُهُ} - his house [dāruhu]\\
\textarabic{دَارُهَا} - her house [dāruhā]\\
\textarabic{دَارُنَا} - our house [dārunā]
\end{multicols}

\subsection{\textarabic{لَيْلًا} - \textbf{laylan} [lajlan]}

\begin{tabular}{p{3cm}p{10cm}}
\toprule
\textbf{Translation} & at night, during the night \\
\textbf{Root} & \textarabic{ل-ي-ل} (l-y-l) \\
\textbf{Pattern} & \textarabic{فَعْل} (faʕl) \\
\textbf{Grammar} & Adverb of time (\textarabic{ظَرْف زَمَان}) in accusative case \\
\textbf{Simple Form} & \textarabic{لَيْلٌ} (night) [laylun] \\
\textbf{Examples} & \textarabic{نَهَارًا وَلَيْلًا} (day and night) [nahāran wa-laylan], \textarabic{اللَّيْلُ طَوِيلٌ} (the night is long) [al-laylu ṭawīlun] \\
\bottomrule
\end{tabular}

\textbf{Time Adverbs with Similar Pattern:}
\begin{multicols}{2}
\small
\textarabic{صَبَاحًا} - in the morning [ṣabāḥan]\\
\textarabic{ظُهْرًا} - at noon [ḍuhran]\\
\textarabic{عَصْرًا} - in the afternoon [ʕaṣran]\\
\textarabic{مَسَاءً} - in the evening [masāʔan]\\
\textarabic{لَيْلًا} - at night [laylan]\\
\textarabic{فَجْرًا} - at dawn [fajran]
\end{multicols}

% ------- NEW words -------
\subsection{\textarabic{فَقَامَ} – \textbf{fa-qāma} [faqaːma]}
\begin{tabular}{p{3cm}p{10cm}}
\toprule
\textbf{Translation} & so he rose / got up\\
\textbf{Root} & \textarabic{ق-و-م} (q-w-m)\\
\textbf{Pattern} & \textarabic{فَعَلَ} with prefix \textarabic{فَـ} (fa-) “so/then”\\
\textbf{Grammar} & Past-tense verb, 3rd person masc. sing.\\
\textbf{Simple Form} & \textarabic{قَامَ} (he stood) [qaːma]\\
\textbf{Examples} & \textarabic{قَامَ مِنَ النَّوْمِ} (got out of bed) [qaːma mina n-nawmi]\\
\bottomrule
\end{tabular}

\textbf{Conjugation (simple verb \textarabic{قَامَ}):}
\begin{multicols}{2}
\small
\textarabic{قُمْتُ} – I rose [qumtu]\\
\textarabic{قُمْتَ} – you rose (m.) [qumta]\\
\textarabic{قَامُوا} – they rose [qaːmū]\\
\textarabic{يَقُومُ} – he rises (pres.) [jaquːmu]\\
\end{multicols}

\subsection{\textarabic{إِلَى} – \textbf{ilā} [ʔilaː]}
\begin{tabular}{p{3cm}p{10cm}}
\toprule
\textbf{Translation} & to, toward\\
\textbf{Type} & Preposition (\textarabic{حَرْف جَرّ})\\
\textbf{Function} & Indicates movement toward a place or goal\\
\textbf{Examples} & \textarabic{ذَهَبَ إِلَى السُّوقِ} (went to the market) [ðahaba ʔila s-sūqi]\\
\bottomrule
\end{tabular}

\subsection{\textarabic{خِزَانَةِ} – \textbf{xizānati} [xizaːnati]}
\begin{tabular}{p{3cm}p{10cm}}
\toprule
\textbf{Translation} & wardrobe, closet (genitive)\\
\textbf{Root} & \textarabic{خ-ز-ن} (x-z-n) “to store”\\
\textbf{Pattern} & \textarabic{فِعَالَة} (fiʕāla) noun pattern\\
\textbf{Grammar} & Feminine noun, genitive after \textarabic{إِلَى}\\
\textbf{Simple Form} & \textarabic{خِزَانَةٌ} (a wardrobe) [xizaːnatun]\\
\textbf{Examples} & \textarabic{خِزَانَةُ المَلَابِسِ} (clothes closet) [xizaːnatu l-malābisi]\\
\bottomrule
\end{tabular}

\subsection{\textarabic{غُرْفَةِ} – \textbf{ġurfati} [ɣurfa ti]}
\begin{tabular}{p{3cm}p{10cm}}
\toprule
\textbf{Translation} & room (genitive)\\
\textbf{Root} & \textarabic{غ-ر-ف} (ġ-r-f) “to scoop/room”\\
\textbf{Pattern} & \textarabic{فُعْلَة} (fuʕla)\\
\textbf{Grammar} & Feminine noun, genitive in construct phrase\\
\textbf{Simple Form} & \textarabic{غُرْفَةٌ} (a room) [ɣurfa tun]\\
\textbf{Examples} & \textarabic{غُرْفَةُ الجُلُوسِ} (living room) [ɣurfa tu l-julūsi]\\
\bottomrule
\end{tabular}

\subsection{\textarabic{البُومِ} – \textbf{al-būmi} [albˤuːmi]}
\begin{tabular}{p{3cm}p{10cm}}
\toprule
\textbf{Translation} & the owls (genitive plural)\\
\textbf{Root} & \textarabic{ب-و-م} (b-w-m) “owl”\\
\textbf{Pattern} & Sound plural of \textarabic{بُومَة} (owl)\\
\textbf{Grammar} & Definite noun in genitive\\
\textbf{Cultural Note} & The phrase “room of the owls” is humorous; many printed texts read \textarabic{غُرْفَةِ النَّوْمِ} (bedroom).\\
\textbf{Examples} & \textarabic{يُحِبُّ البُومَ} (he likes owls) [juḥibbu l-būma]\\
\bottomrule
\end{tabular}

\subsection{\textarabic{وَاخْتَبَأَ} – \textbf{wa-xtaʔa} [waxtabaʔa]}
\begin{tabular}{p{3cm}p{10cm}}
\toprule
\textbf{Translation} & and he hid\\
\textbf{Root} & \textarabic{خ-ب-أ} (x-b-ʔ) “to hide”\\
\textbf{Pattern} & Form VIII \textarabic{اِفْتَعَلَ} – \textarabic{اِخْتَبَأَ}\\
\textbf{Grammar} & Past-tense verb, 3rd person masc. sing.\\
\textbf{Simple Form} & \textarabic{خَبَأَ} (he hid) [xabaʔa]\\
\textbf{Examples} & \textarabic{اِخْتَبَأَ الطِّفْلُ خَلْفَ البابِ} (the child hid behind the door) [ixtaʔa ṭ-ṭiflu xalfa l-bābi]\\
\bottomrule
\end{tabular}

\subsection{\textarabic{فِيهَا} – \textbf{fī-hā} [fiːhaː]}
\begin{tabular}{p{3cm}p{10cm}}
\toprule
\textbf{Translation} & in it\\
\textbf{Components} & \textarabic{في} (in) + \textarabic{ها} (her/it, fem.)\\
\textbf{Grammar} & Prepositional phrase with attached pronoun\\
\textbf{Examples} & \textarabic{كُتِبَ فِيهَا} (written in it) [kutiba fiːhaː]\\
\bottomrule
\end{tabular}

% ======================== Phrase Analysis ========================
\section{Phrase Analysis}

\begin{tcolorbox}[colback=boxcolor,colframe=headercolor]
\textbf{Grammatical Structure:}\\
Verb + Subject + Prepositional Phrase (object of sensation) + Prepositional Phrase (location) + Adverb (time)\\
\hspace*{1em} \textit{then} Conjunction + Verb + Prepositional Phrase (direction) + Construct Phrase + Conjunction + Verb + Prepositional Pronoun\\[1em]
\textbf{Key Grammar Points:}
\begin{itemize}
\item The prefix \textarabic{فَـ} links sequential actions (“so/then”).
\item \textarabic{إِلَى} governs the genitive; nouns that follow appear in genitive case.
\item The chain \textarabic{خِزَانَةِ غُرْفَةِ البُومِ} forms an \textit{iḍāfa} (possessive chain) translating as “the wardrobe of the owl room.”
\item Form VIII verbs like \textarabic{اِخْتَبَأَ} often convey reflexive or internal actions.
\item \textarabic{فِيهَا} ends the sentence with a resumptive pronoun referring back to the wardrobe.
\end{itemize}
\end{tcolorbox}

% ======================== Additional Practice Phrases ========================
\section{Similar Phrases for Practice}

\begin{enumerate}
\item \textarabic{سَمِعَتْ فَاطِمَةُ صَوْتًا غَرِيبًا في غُرْفَتِهَا لَيْلًا، فَقَامَتْ إِلَى الشُّبَّاكِ وَأَغْلَقَتْهُ}\\
Fatima heard a strange sound in her room at night, so she went to the window and closed it. [samiʕat fāṭimatu ṣawtan ɣarīban fī ɣurfatihā laylan, faqāmat ʔila ʃ-ʃubbāki waʔaɣlaqat-hu]

\item \textarabic{شَعَرَ الرَّجُلُ بِحَرَكَةٍ في الحَدِيقَةِ، فَقَامَ إِلَى البَابِ وَاخْتَبَأَ وَرَاءَهُ}\\
The man sensed movement in the garden, so he went to the door and hid behind it. [ʃaʕara r-rajulu bi-ḥaraka-tin fī l-ḥadīqati, faqāma ʔila l-bābi waxtaʔa waraːʔa-hu]

\item \textarabic{وَجَدَ عَلِيٌّ لِصًّا في المَطْبَخِ صَبَاحًا، فَأَخَذَ هَاتِفَهُ وَاتَّصَلَ بِالشُّرْطَةِ}\\
Ali found a thief in the kitchen in the morning, so he took his phone and called the police. [wad͡ʒada ʕaliyyun liṣṣan fī l-maṭbaxi ṣabāḥan, faʔaxaða hātifahu waʔittaṣala biʃ-ʃurṭati]
\end{enumerate}

% ======================== Levantine Dialect Version ========================
\section{Levantine (Shaami) Dialect Version}

\begin{tcolorbox}[colback=white,colframe=accentcolor]
\textbf{Modern Standard Arabic:}\\
\textarabic{شَعَرَ جُحا بِلِصٍّ في دَارِهِ لَيْلًا، فَقَامَ إِلَى خِزَانَةِ غُرْفَةِ البُومِ وَاخْتَبَأَ فِيهَا}
\\[1em]
\textbf{Levantine Equivalent:}\\
\textarabic{حَسّ جُحا بِحَرَامِي بِبَيْتُو بِاللَّيْل، قام عَ خْزانِة غُرْفِة النُّوم وَخْتَبَى جُوَّاتْهَا}\\[0.5em]
\textbf{Transliteration:} ḥass juḥā bi-ḥarāmī b-bētu b-éllēl, qām ʕa xzané ġurfé n-nōm w-xtabā jwāthā
\end{tcolorbox}

\textbf{Dialectal Notes:}
\begin{itemize}
\item \textarabic{شَعَرَ} → \textarabic{حَسّ} (“to feel”) is preferred in spoken Shaami.
\item \textarabic{لِصّ} → \textarabic{حَرَامِي} (colloquial “thief”).
\item Preposition \textarabic{عَ} (ʕa “to/on”) replaces \textarabic{إِلَى}.
\item \textarabic{غُرْفَةِ النُّوم} (“bedroom”) is used instead of the playful “owl room.”
\item The verb \textarabic{خْتَبَى} is the dialectal past form of \textarabic{اِخْتَبَأَ}.
\end{itemize}

% ======================== Cultural & Learning Notes ========================
\section{Cultural and Learning Notes}

\begin{tcolorbox}[colback=boxcolor,colframe=headercolor]
\textbf{About Juha:} A legendary folk character famed for wisdom disguised as folly. His tales often end with humorous twists that reveal moral lessons.

\vspace{0.5em}
\textbf{Story Context:} In this scene Juha, certain he owns nothing worth stealing, hides in the wardrobe so the thief can “look around undisturbed.” When caught, he jokes that embarrassment drove him to hide.

\vspace{0.5em}
\textbf{Learning Tips:}
\begin{itemize}
\item Practice chaining actions with the prefix \textarabic{فَـ} to narrate sequences.
\item Build \textit{iḍāfa} chains of 2–3 nouns to master genitive agreement.
\item Compare Form I verbs to their Form VIII counterparts to notice self-directed meanings.
\item Shadow-read both MSA and dialect versions to internalize sound patterns.
\end{itemize}
\end{tcolorbox}

% ======================== Memory Aids ========================
\section{Memory Aids}

\begin{itemize}
\item Associate \textarabic{قَامَ} with “came up” to recall its meaning “got up.”
\item \textarabic{خِزَانَةٌ} starts with “kh” like “closet” starts with “c,” both places for clothes.
\item The root \textarabic{خ-ب-أ} appears in many detective stories—think “hide evidence.”
\item For time adverbs, remember the common pattern ending in \textarabic{ـًا}: \textarabic{لَيْلًا} (at night), \textarabic{صَبَاحًا} (in the morning), \textarabic{ظُهْرًا} (at noon).
\end{itemize}

\end{document}

