\documentclass[letter,12pt]{article}

\usepackage[most]{tcolorbox}
\usepackage{longtable}
\usepackage{makecell}
\usepackage[utf8]{inputenc}
\usepackage[english,arabic]{babel}
\usepackage{tikz}
\usepackage{geometry}
\usepackage{array}
\usepackage{multicol}
\usepackage[table]{xcolor}
\usepackage{polyglossia}
\setmainlanguage{english}
\setotherlanguage{arabic}
\setotherlanguage{hebrew}
\usepackage{arabtex}
\usepackage{fontspec}
\setmainfont{Charis SIL}
\newfontfamily\arabicfont[Script=Arabic,Scale=1.2]{Amiri}
\newfontfamily\hebrewfont{Ezra SIL}[Script=Hebrew]
\geometry{margin=1.5cm}
\usepackage{booktabs}
\usepackage{graphicx}
\usepackage{multirow}

\usetikzlibrary{shapes,arrows,decorations.pathmorphing}

% Custom colors
\definecolor{headercolor}{RGB}{70,130,180}
\definecolor{boxcolor}{RGB}{240,248,255}
\definecolor{accentcolor}{RGB}{220,20,60}
\definecolor{tableheader}{RGB}{220,220,220}
\definecolor{dialectcolor}{RGB}{34,139,34}

\begin{document}

\title{\textbf{\Large Arabic Phrase Analyser}\\
\large Juha Hides from a Thief in His House\\
\normalsize \textarabic{شَعَرَ جُحا بِلِصٍّ فِي دَارِهِ لَيْلًا، فَقَامَ إِلَى خِزَانَةِ غُرْفَةِ البُومِ وَاخْتَبَأَ فِيهَا}}
\author{Igor Deruga}
\date{}
\maketitle

% ======================== Phrase Display ========================
\begin{tcolorbox}[colback=boxcolor,colframe=headercolor,title=\textbf{Arabic Phrase},breakable]
\centering
\textarabic{شعر جحا بلص في داره ليلا، فقام إلى خزانة غرفة البوم واختبأ فيها}
\\[0.5em]
\textbf{Without Diacritics}
\\[1em]
\textarabic{شَعَرَ جُحا بِلِصٍّ فِي دَارِهِ لَيْلًا، فَقَامَ إِلَى خِزَانَةِ غُرْفَةِ البُومِ وَاخْتَبَأَ فِيهَا}
\\[0.5em]
\textbf{With Full Diacritics}
\end{tcolorbox}

% ======================== Translations ========================
\section{English Translation}
\begin{tcolorbox}[colback=white,colframe=accentcolor,breakable]
\textbf{Literal:} Felt Juha with-thief in house-his night, so-stood to closet room the-owl and-hid in-it \\
\textit{[Arabic order retained for direct mapping]}\\[0.5em]
\textbf{Adapted:} Juha felt a thief in his house at night, so he went to the closet of the owl room and hid in it
\end{tcolorbox}

% ======================== Detailed Word Analysis ========================
\section{Detailed Word Analysis}

\subsection{\textarabic{شَعَرَ} — [ʃaʕara]}
\begin{tabular}{p{3cm}p{10cm}}
\toprule
\textbf{Translation} & felt / sensed \\
\textbf{Root} & \textarabic{ش-ع-ر} (ʃ-ʕ-r) \\
\textbf{Pattern} & \textarabic{فَعَلَ} (faʕala) \\
\textbf{Grammar} & Past tense verb, 3rd person masculine singular, active voice \\
\midrule \\
\textbf{Examples} & \makecell[l]{\parbox{9.5cm}{
1. \textarabic{شَعَرَ الرَّجُلُ بِالأَلَمِ} - The man felt pain [ʃaʕara r-radʒulu bil-ʔalam]\\
2. \textarabic{يَشْعُرُ بِالسَّعَادَةِ} - He feels happiness [jaʃʕuru bis-saʕaːda]\\
3. \textarabic{سَيَشْعُرُ بِالنَّدَمِ} - He will feel regret [sajaʃʕuru bin-nadam]
}} \\
\midrule \\
\textbf{Synonyms} & \textarabic{أَحَسَّ} (felt), \textarabic{وَجَدَ} (found/felt), \textarabic{لَمَسَ} (touched/sensed) \\
\textbf{Etymology} & From Proto-Semitic *ʃaʕar-, related to Hebrew \texthebrew{שער} (shaʕar) "hair/sense" \\
\bottomrule
\end{tabular}

\subsection{Conjugation}
\begin{longtable}{|>{\raggedright}p{3.5cm}|p{5cm}|p{5cm}|}
\hline
\textbf{Person} & \textbf{Perfect (Past)} & \textbf{Imperfect (Present)} \\
\hline
\textbf{3rd person masculine singular} & \textarabic{شَعَرَ} [ʃaʕara] & \textarabic{يَشْعُرُ} [jaʃʕuru] \\
\hline
\textbf{3rd person feminine singular} & \textarabic{شَعَرَتْ} [ʃaʕarat] & \textarabic{تَشْعُرُ} [taʃʕuru] \\
\hline
\textbf{3rd person masculine dual} & \textarabic{شَعَرَا} [ʃaʕaraː] & \textarabic{يَشْعُرَانِ} [jaʃʕuraːni] \\
\hline
\textbf{3rd person feminine dual} & \textarabic{شَعَرَتَا} [ʃaʕarataː] & \textarabic{تَشْعُرَانِ} [taʃʕuraːni] \\
\hline
\textbf{3rd person masculine plural} & \textarabic{شَعَرُوا} [ʃaʕaruː] & \textarabic{يَشْعُرُونَ} [jaʃʕuruːna] \\
\hline
\textbf{3rd person feminine plural} & \textarabic{شَعَرْنَ} [ʃaʕarna] & \textarabic{يَشْعُرْنَ} [jaʃʕurna] \\
\hline
\textbf{2nd person masculine singular} & \textarabic{شَعَرْتَ} [ʃaʕarta] & \textarabic{تَشْعُرُ} [taʃʕuru] \\
\hline
\textbf{2nd person feminine singular} & \textarabic{شَعَرْتِ} [ʃaʕarti] & \textarabic{تَشْعُرِينَ} [taʃʕuriːna] \\
\hline
\textbf{2nd person dual (m./f.)} & \textarabic{شَعَرْتُمَا} [ʃaʕartumaː] & \textarabic{تَشْعُرَانِ} [taʃʕuraːni] \\
\hline
\textbf{2nd person masculine plural} & \textarabic{شَعَرْتُمْ} [ʃaʕartum] & \textarabic{تَشْعُرُونَ} [taʃʕuruːna] \\
\hline
\textbf{2nd person feminine plural} & \textarabic{شَعَرْتُنَّ} [ʃaʕartunna] & \textarabic{تَشْعُرْنَ} [taʃʕurna] \\
\hline
\textbf{1st person singular} & \textarabic{شَعَرْتُ} [ʃaʕartu] & \textarabic{أَشْعُرُ} [ʔaʃʕuru] \\
\hline
\textbf{1st person plural} & \textarabic{شَعَرْنَا} [ʃaʕarnaː] & \textarabic{نَشْعُرُ} [naʃʕuru] \\
\hline
\end{longtable}

\subsubsection*{Conjugation Notes}
\begin{itemize}
  \item The \textbf{future} is formed with the prefix \textarabic{سَ} [sa-] or \textarabic{سَوْفَ} [sawfa] before the imperfect (e.g., \textarabic{سَيَشْعُرُ} "he will feel").
  \item The \textbf{moods of the imperfect}:
    \begin{itemize}
      \item Indicative: \textarabic{يَشْعُرُ} [jaʃʕuru]
      \item Subjunctive: \textarabic{لَنْ يَشْعُرَ} [lan jaʃʕura]
      \item Jussive: \textarabic{لَمْ يَشْعُرْ} [lam jaʃʕur]
      \item Imperative: \textarabic{اُشْعُرْ} [uʃʕur!]
    \end{itemize}
  \item The verb \textarabic{شَعَرَ} typically requires the preposition \textarabic{بِ} to mean "felt with/sensed"
\end{itemize}

\subsection{\textarabic{جُحا} — [dʒuħaː]}
\begin{tabular}{p{3cm}p{10cm}}
\toprule
\textbf{Translation} & Juha (proper name of folkloric character) \\
\textbf{Root} & Proper noun, no trilateral root \\
\textbf{Pattern} & \textarabic{فُعَلَ} pattern for names \\
\textbf{Grammar} & Proper noun, masculine, subject of the verb \\
\midrule \\
\textbf{Examples} & \makecell[l]{\parbox{9.5cm}{
1. \textarabic{قَالَ جُحا شَيْئًا مُضْحِكًا} - Juha said something funny [qaːla dʒuħaː ʃajʔan mudħikan]\\
2. \textarabic{حِكَايَاتُ جُحا مَشْهُورَةٌ} - Juha's stories are famous [ħikaːjaːtu dʒuħaː maʃhuːra]\\
3. \textarabic{يُحِبُّ الأَطْفَالُ قِصَصَ جُحا} - Children love Juha's stories [juħibbu l-ʔatfaːlu qisas dʒuħaː]
}} \\
\midrule \\
\textbf{Synonyms} & \textarabic{نَصْرُ الدِّينِ} (Nasreddin), regional variations of the same character \\
\textbf{Etymology} & Persian origin, possibly from Jahā "world" or related to Turkish Hoca "teacher" \\
\bottomrule
\end{tabular}

\subsubsection*{Full Declension Matrix}
\begin{tabular}{|c|c|c|}
\hline
\textbf{Case} & \textbf{Form} & \textbf{Usage} \\
\hline
Nominative & \textarabic{جُحا} [dʒuħaː] & Subject of sentence \\
\hline
Accusative & \textarabic{جُحا} [dʒuħaː] & Direct object \\
\hline
Genitive & \textarabic{جُحا} [dʒuħaː] & After prepositions, in construct \\
\hline
\end{tabular}

\subsection{\textarabic{بِلِصٍّ} — [bilisˤsˤin]}
\begin{tabular}{p{3cm}p{10cm}}
\toprule
\textbf{Translation} & with a thief \\
\textbf{Root} & \textarabic{ل-ص-ص} (l-sˤ-sˤ) \\
\textbf{Pattern} & \textarabic{فِعْل} (fiʕl) with preposition \textarabic{بِ} \\
\textbf{Grammar} & Preposition + indefinite noun, genitive case, masculine singular \\
\midrule \\
\textbf{Examples} & \makecell[l]{\parbox{9.5cm}{
1. \textarabic{قَبَضَ الشُّرْطِيُّ عَلَى اللِّصِّ} - The policeman caught the thief [qabaðˤa ʃ-ʃurtˤijju ʕala l-lisˤsˤi]\\
2. \textarabic{دَخَلَ لِصٌّ إِلَى البَيْتِ} - A thief entered the house [daxala lisˤsˤun ʔila l-bajti]\\
3. \textarabic{سَيُعَاقِبُ القَاضِي اللُّصُوصَ} - The judge will punish the thieves [sajuʕaːqibu l-qaːdˤi l-lusˤuːsˤa]
}} \\
\midrule \\
\textbf{Synonyms} & \textarabic{سَارِق} (thief), \textarabic{حَرَامِيّ} (burglar), \textarabic{نَشَّال} (pickpocket) \\
\textbf{Etymology} & From root meaning "to stick/adhere" - one who "sticks" to others' property \\
\bottomrule
\end{tabular}

\subsubsection*{Full Declension Matrix}
\begin{tabular}{|c|c|c|c|}
\hline
\textbf{Number} & \textbf{Case} & \textbf{Indefinite} & \textbf{Definite} \\
\hline
\multirow{3}{*}{Singular}
 & Nominative   & \textarabic{لِصٌّ} (lisˤsˤun) & \textarabic{اللِّصُّ} (al-lisˤsˤu) \\
 & Accusative   & \textarabic{لِصًّا} (lisˤsˤan) & \textarabic{اللِّصَّ} (al-lisˤsˤa) \\
 & Genitive     & \textarabic{لِصٍّ} (lisˤsˤin) & \textarabic{اللِّصِّ} (al-lisˤsˤi) \\
\hline
\multirow{3}{*}{Plural}
 & Nominative   & \textarabic{لُصُوصٌ} (lusˤuːsˤun) & \textarabic{اللُّصُوصُ} (al-lusˤuːsˤu) \\
 & Accusative   & \textarabic{لُصُوصًا} (lusˤuːsˤan) & \textarabic{اللُّصُوصَ} (al-lusˤuːsˤa) \\
 & Genitive     & \textarabic{لُصُوصٍ} (lusˤuːsˤin) & \textarabic{اللُّصُوصِ} (al-lusˤuːsˤi) \\
\hline
\end{tabular}

\subsection{\textarabic{فِي دَارِهِ} — [fiː daːrihi]}
\begin{tabular}{p{3cm}p{10cm}}
\toprule
\textbf{Translation} & in his house \\
\textbf{Root} & \textarabic{د-و-ر} (d-w-r) \\
\textbf{Pattern} & \textarabic{فَاعِل} (faːʕil) + possessive pronoun \\
\textbf{Grammar} & Preposition + definite noun in genitive + 3rd person masculine singular possessive \\
\midrule \\
\textbf{Examples} & \makecell[l]{\parbox{9.5cm}{
1. \textarabic{دَارُ الأُسْتَاذِ كَبِيرَةٌ} - The professor's house is big [daːru l-ʔustaːði kabiːra]\\
2. \textarabic{سَأَزُورُ دَارَكَ غَدًا} - I will visit your house tomorrow [saʔazuːru daːraka ɣadan]\\
3. \textarabic{بَنَى دَارًا جَدِيدَةً} - He built a new house [banaː daːran dʒadiːda]
}} \\
\midrule \\
\textbf{Synonyms} & \textarabic{بَيْت} (house), \textarabic{مَنْزِل} (residence), \textarabic{مَسْكَن} (dwelling) \\
\textbf{Etymology} & From root d-w-r meaning "to turn/circle" - the place one circles back to \\
\bottomrule
\end{tabular}

\subsubsection*{Full Declension Matrix}
\begin{tabular}{|c|c|c|c|}
\hline
\textbf{Person} & \textbf{Nominative} & \textbf{Accusative} & \textbf{Genitive} \\
\hline
My house & \textarabic{دَارِي} [daːriː] & \textarabic{دَارِيَ} [daːrijja] & \textarabic{دَارِي} [daːriː] \\
\hline
Your house (m.) & \textarabic{دَارُكَ} [daːruka] & \textarabic{دَارَكَ} [daːraka] & \textarabic{دَارِكَ} [daːrika] \\
\hline
His house & \textarabic{دَارُهُ} [daːruhu] & \textarabic{دَارَهُ} [daːrahu] & \textarabic{دَارِهِ} [daːrihi] \\
\hline
\end{tabular}

\subsection{\textarabic{لَيْلًا} — [lajlan]}
\begin{tabular}{p{3cm}p{10cm}}
\toprule
\textbf{Translation} & at night / nocturnally \\
\textbf{Root} & \textarabic{ل-ي-ل} (l-j-l) \\
\textbf{Pattern} & \textarabic{فَعْل} (fajl) in accusative (adverbial) \\
\textbf{Grammar} & Adverbial accusative of time, indefinite \\
\midrule \\
\textbf{Examples} & \makecell[l]{\parbox{9.5cm}{
1. \textarabic{نَامَ الطِّفْلُ لَيْلًا} - The child slept at night [naːma tˤ-tˤiflu lajlan]\\
2. \textarabic{اللَّيْلُ طَوِيلٌ} - The night is long [al-lajlu tˤawiːl]\\
3. \textarabic{سَيَسْهَرُ فِي اللَّيْلِ} - He will stay up at night [sajasharu fiː l-lajli]
}} \\
\midrule \\
\textbf{Synonyms} & \textarabic{عَشِيَّة} (evening), \textarabic{ظَلَام} (darkness), \textarabic{دُجَى} (dark night) \\
\textbf{Etymology} & From Proto-Semitic *lajl-, related to Hebrew \texthebrew{לילה} (lajla) "night" \\
\bottomrule
\end{tabular}

\subsubsection*{Full Declension Matrix}
\begin{tabular}{|c|c|c|c|}
\hline
\textbf{Number} & \textbf{Case} & \textbf{Indefinite} & \textbf{Definite} \\
\hline
\multirow{3}{*}{Singular}
 & Nominative   & \textarabic{لَيْلٌ} (lajlun) & \textarabic{اللَّيْلُ} (al-lajlu) \\
 & Accusative   & \textarabic{لَيْلًا} (lajlan) & \textarabic{اللَّيْلَ} (al-lajla) \\
 & Genitive     & \textarabic{لَيْلٍ} (lajlin) & \textarabic{اللَّيْلِ} (al-lajli) \\
\hline
\multirow{3}{*}{Plural}
 & Nominative   & \textarabic{لَيَالٍ} (lajaːlin) & \textarabic{اللَّيَالِي} (al-lajaːliː) \\
 & Accusative   & \textarabic{لَيَالِيَ} (lajaːlija) & \textarabic{اللَّيَالِيَ} (al-lajaːlija) \\
 & Genitive     & \textarabic{لَيَالٍ} (lajaːlin) & \textarabic{اللَّيَالِي} (al-lajaːliː) \\
\hline
\end{tabular}

\subsection{\textarabic{فَقَامَ} — [faqaːma]}
\begin{tabular}{p{3cm}p{10cm}}
\toprule
\textbf{Translation} & so he stood up / then he got up \\
\textbf{Root} & \textarabic{ق-و-م} (q-w-m) \\
\textbf{Pattern} & \textarabic{فَعَلَ} (faʕala) with conjunction prefix \\
\textbf{Grammar} & Past tense verb, 3rd person masculine singular, with \textarabic{فَ} conjunction \\
\midrule \\
\textbf{Examples} & \makecell[l]{\parbox{9.5cm}{
1. \textarabic{قَامَ الرَّجُلُ مِنْ كُرْسِيِّهِ} - The man got up from his chair [qaːma r-radʒulu min kursijjihi]\\
2. \textarabic{يَقُومُ كُلَّ صَبَاحٍ بَاكِرًا} - He gets up early every morning [jaquːmu kulla sˤabaːħin baːkiran]\\
3. \textarabic{قَامُوا جَمِيعًا} - They all stood up [qaːmuː dʒamiːʕan]
}} \\
\midrule \\
\textbf{Synonyms} & \textarabic{وَقَفَ} (stood), \textarabic{نَهَضَ} (rose up), \textarabic{اسْتَقَامَ} (straightened up) \\
\textbf{Etymology} & From Proto-Semitic *qwm, related to Hebrew \texthebrew{קום} (qum) "rise" \\
\bottomrule
\end{tabular}

\subsection{Conjugation}
\begin{longtable}{|>{\raggedright}p{3.5cm}|p{5cm}|p{5cm}|}
\hline
\textbf{Person} & \textbf{Perfect (Past)} & \textbf{Imperfect (Present)} \\
\hline
\textbf{3rd person masculine singular} & \textarabic{قَامَ} [qaːma] & \textarabic{يَقُومُ} [jaquːmu] \\
\hline
\textbf{3rd person feminine singular} & \textarabic{قَامَتْ} [qaːmat] & \textarabic{تَقُومُ} [taquːmu] \\
\hline
\textbf{3rd person masculine dual} & \textarabic{قَامَا} [qaːmaː] & \textarabic{يَقُومَانِ} [jaquːmaːni] \\
\hline
\textbf{3rd person feminine dual} & \textarabic{قَامَتَا} [qaːmataː] & \textarabic{تَقُومَانِ} [taquːmaːni] \\
\hline
\textbf{3rd person masculine plural} & \textarabic{قَامُوا} [qaːmuː] & \textarabic{يَقُومُونَ} [jaquːmuːna] \\
\hline
\textbf{3rd person feminine plural} & \textarabic{قُمْنَ} [qumna] & \textarabic{يَقُمْنَ} [jaqumna] \\
\hline
\textbf{2nd person masculine singular} & \textarabic{قُمْتَ} [qumta] & \textarabic{تَقُومُ} [taquːmu] \\
\hline
\textbf{2nd person feminine singular} & \textarabic{قُمْتِ} [qumti] & \textarabic{تَقُومِينَ} [taquːmiːna] \\
\hline
\textbf{2nd person dual (m./f.)} & \textarabic{قُمْتُمَا} [qumtumaː] & \textarabic{تَقُومَانِ} [taquːmaːni] \\
\hline
\textbf{2nd person masculine plural} & \textarabic{قُمْتُمْ} [qumtum] & \textarabic{تَقُومُونَ} [taquːmuːna] \\
\hline
\textbf{2nd person feminine plural} & \textarabic{قُمْتُنَّ} [qumtunna] & \textarabic{تَقُمْنَ} [taqumna] \\
\hline
\textbf{1st person singular} & \textarabic{قُمْتُ} [qumtu] & \textarabic{أَقُومُ} [ʔaquːmu] \\
\hline
\textbf{1st person plural} & \textarabic{قُمْنَا} [qumnaː] & \textarabic{نَقُومُ} [naquːmu] \\
\hline
\end{longtable}

\subsection{\textarabic{إِلَى خِزَانَةِ} — [ʔilaː xizaːnati]}
\begin{tabular}{p{3cm}p{10cm}}
\toprule
\textbf{Translation} & to the closet / toward the cabinet \\
\textbf{Root} & \textarabic{خ-ز-ن} (x-z-n) \\
\textbf{Pattern} & \textarabic{فِعَالَة} (fiʕaːla) \\
\textbf{Grammar} & Preposition + definite noun in genitive case, feminine singular \\
\midrule \\
\textbf{Examples} & \makecell[l]{\parbox{9.5cm}{
1. \textarabic{فَتَحَ خِزَانَةَ المَلَابِسِ} - He opened the clothes closet [fataħa xizaːnata l-malaːbisi]\\
2. \textarabic{خِزَانَةُ الكُتُبِ مُمْتَلِئَةٌ} - The book cabinet is full [xizaːnatu l-kutubi mumtaliʔa]\\
3. \textarabic{اشْتَرَى خِزَانَةً جَدِيدَةً} - He bought a new closet [ʔiʃtaraː xizaːnatan dʒadiːda]
}} \\
\midrule \\
\textbf{Synonyms} & \textarabic{دُولَاب} (wardrobe), \textarabic{صُنْدُوق} (chest), \textarabic{خَزِينَة} (treasury/storage) \\
\textbf{Etymology} & From root x-z-n meaning "to store/treasure" \\
\bottomrule
\end{tabular}

\subsubsection*{Full Declension Matrix}
\begin{tabular}{|c|c|c|c|}
\hline
\textbf{Number} & \textbf{Case} & \textbf{Indefinite} & \textbf{Definite} \\
\hline
\multirow{3}{*}{Singular}
 & Nominative   & \textarabic{خِزَانَةٌ} (xizaːnatun) & \textarabic{الخِزَانَةُ} (al-xizaːnatu) \\
 & Accusative   & \textarabic{خِزَانَةً} (xizaːnatan) & \textarabic{الخِزَانَةَ} (al-xizaːnata) \\
 & Genitive     & \textarabic{خِزَانَةٍ} (xizaːnatin) & \textarabic{الخِزَانَةِ} (al-xizaːnati) \\
\hline
\multirow{3}{*}{Plural}
 & Nominative   & \textarabic{خِزَانَاتٌ} (xizaːnaːtun) & \textarabic{الخِزَانَاتُ} (al-xizaːnaːtu) \\
 & Accusative   & \textarabic{خِزَانَاتٍ} (xizaːnaːtin) & \textarabic{الخِزَانَاتِ} (al-xizaːnaːti) \\
 & Genitive     & \textarabic{خِزَانَاتٍ} (xizaːnaːtin) & \textarabic{الخِزَانَاتِ} (al-xizaːnaːti) \\
\hline
\end{tabular}

\subsection{\textarabic{غُرْفَةِ البُومِ} — [ɣurfati l-buːmi]}
\begin{tabular}{p{3cm}p{10cm}}
\toprule
\textbf{Translation} & the owl's room / room of the owl \\
\textbf{Root} & \textarabic{غ-ر-ف} (ɣ-r-f) and \textarabic{ب-و-م} (b-w-m) \\
\textbf{Pattern} & \textarabic{فُعْلَة} (fuʕla) + \textarabic{فُعْل} (fuʕl) \\
\textbf{Grammar} & Definite noun in genitive + definite noun in genitive (possessive construct) \\
\midrule \\
\textbf{Examples} & \makecell[l]{\parbox{9.5cm}{
1. \textarabic{غُرْفَةُ النَّوْمِ وَاسِعَةٌ} - The bedroom is spacious [ɣurfatu n-nawmi waːsiʕa]\\
2. \textarabic{البُومُ يَطِيرُ لَيْلًا} - The owl flies at night [al-buːmu jatˤiːru lajlan]\\
3. \textarabic{سَمِعَ صَوْتَ البُومِ} - He heard the owl's sound [samiʕa sˤawta l-buːmi]
}} \\
\midrule \\
\textbf{Synonyms} & \textarabic{حُجْرَة} (room), \textarabic{قَاعَة} (hall); \textarabic{بُومَة} (female owl) \\
\textbf{Etymology} & ɣ-r-f from "scooping/taking"; b-w-m onomatopoetic for owl's hoot \\
\bottomrule
\end{tabular}

\subsubsection*{Full Declension Matrix}
\begin{tabular}{|c|c|c|c|}
\hline
\textbf{Word} & \textbf{Nominative} & \textbf{Accusative} & \textbf{Genitive} \\
\hline
Room (indef.) & \textarabic{غُرْفَةٌ} [ɣurfatun] & \textarabic{غُرْفَةً} [ɣurfatan] & \textarabic{غُرْفَةٍ} [ɣurfatin] \\
\hline
Room (def.) & \textarabic{الغُرْفَةُ} [al-ɣurfatu] & \textarabic{الغُرْفَةَ} [al-ɣurfata] & \textarabic{الغُرْفَةِ} [al-ɣurfati] \\
\hline
Owl & \textarabic{البُومُ} [al-buːmu] & \textarabic{البُومَ} [al-buːma] & \textarabic{البُومِ} [al-buːmi] \\
\hline
\end{tabular}

\subsection{\textarabic{وَاخْتَبَأَ فِيهَا} — [waxtabaʔa fiːhaː]}
\begin{tabular}{p{3cm}p{10cm}}
\toprule
\textbf{Translation} & and hid in it \\
\textbf{Root} & \textarabic{خ-ب-أ} (x-b-ʔ) \\
\textbf{Pattern} & \textarabic{افْتَعَلَ} (ʔiftaʕala - Form VIII) \\
\textbf{Grammar} & Past tense verb Form VIII, 3rd person masculine singular + preposition + pronoun \\
\midrule \\
\textbf{Examples} & \makecell[l]{\parbox{9.5cm}{
1. \textarabic{اخْتَبَأَ الطِّفْلُ تَحْتَ السَّرِيرِ} - The child hid under the bed [ʔixtabaʔa tˤ-tˤiflu taħta s-sariːri]\\
2. \textarabic{يَخْتَبِئُ اللُّصُوصُ فِي الظَّلَامِ} - Thieves hide in darkness [jaxtabiʔu l-lusˤuːsˤu fiː ðˤ-ðˤalaːmi]\\
3. \textarabic{سَيَخْتَبِئُ هُنَاكَ} - He will hide there [sajaxtabiʔu hunaːka]
}} \\
\midrule \\
\textbf{Synonyms} & \textarabic{اسْتَتَرَ} (concealed himself), \textarabic{تَوَارَى} (disappeared), \textarabic{كَمَنَ} (lay in ambush) \\
\textbf{Etymology} & From root x-b-ʔ meaning "to be hidden/concealed" \\
\bottomrule
\end{tabular}

\subsection{Conjugation}
\begin{longtable}{|>{\raggedright}p{3.5cm}|p{5cm}|p{5cm}|}
\hline
\textbf{Person} & \textbf{Perfect (Past)} & \textbf{Imperfect (Present)} \\
\hline
\textbf{3rd person masculine singular} & \textarabic{اخْتَبَأَ} [ʔixtabaʔa] & \textarabic{يَخْتَبِئُ} [jaxtabiʔu] \\
\hline
\textbf{3rd person feminine singular} & \textarabic{اخْتَبَأَتْ} [ʔixtabaʔat] & \textarabic{تَخْتَبِئُ} [taxtabiʔu] \\
\hline
\textbf{3rd person masculine dual} & \textarabic{اخْتَبَآ} [ʔixtabaʔaː] & \textarabic{يَخْتَبِئَانِ} [jaxtabiʔaːni] \\
\hline
\textbf{3rd person feminine dual} & \textarabic{اخْتَبَأَتَا} [ʔixtabaʔataː] & \textarabic{تَخْتَبِئَانِ} [taxtabiʔaːni] \\
\hline
\textbf{3rd person masculine plural} & \textarabic{اخْتَبَؤُوا} [ʔixtabaʔuː] & \textarabic{يَخْتَبِئُونَ} [jaxtabiʔuːna] \\
\hline
\textbf{3rd person feminine plural} & \textarabic{اخْتَبَأْنَ} [ʔixtabaʔna] & \textarabic{يَخْتَبِئْنَ} [jaxtabiʔna] \\
\hline
\textbf{2nd person masculine singular} & \textarabic{اخْتَبَأْتَ} [ʔixtabaʔta] & \textarabic{تَخْتَبِئُ} [taxtabiʔu] \\
\hline
\textbf{2nd person feminine singular} & \textarabic{اخْتَبَأْتِ} [ʔixtabaʔti] & \textarabic{تَخْتَبِئِينَ} [taxtabiʔiːna] \\
\hline
\textbf{2nd person dual (m./f.)} & \textarabic{اخْتَبَأْتُمَا} [ʔixtabaʔtumaː] & \textarabic{تَخْتَبِئَانِ} [taxtabiʔaːni] \\
\hline
\textbf{2nd person masculine plural} & \textarabic{اخْتَبَأْتُمْ} [ʔixtabaʔtum] & \textarabic{تَخْتَبِئُونَ} [taxtabiʔuːna] \\
\hline
\textbf{2nd person feminine plural} & \textarabic{اخْتَبَأْتُنَّ} [ʔixtabaʔtunna] & \textarabic{تَخْتَبِئْنَ} [taxtabiʔna] \\
\hline
\textbf{1st person singular} & \textarabic{اخْتَبَأْتُ} [ʔixtabaʔtu] & \textarabic{أَخْتَبِئُ} [ʔaxtabiʔu] \\
\hline
\textbf{1st person plural} & \textarabic{اخْتَبَأْنَا} [ʔixtabaʔnaː] & \textarabic{نَخْتَبِئُ} [naxtabiʔu] \\
\hline
\end{longtable}

\subsubsection*{Conjugation Notes}
\begin{itemize}
  \item Form VIII verbs like \textarabic{اخْتَبَأَ} often indicate reflexive or intensive action
  \item The pattern \textarabic{افْتَعَلَ} (ʔiftaʕala) shows the characteristic \textarabic{ت} infixed after the first radical
  \item \textarabic{فِيهَا} [fiːhaː] is the feminine pronoun referring back to \textarabic{خِزَانَة} (closet)
\end{itemize}

% ======================== Phrase Analysis ========================
\section{Phrase Analysis}
\begin{tcolorbox}[colback=boxcolor,colframe=headercolor,title=\textbf{Grammatical Structure},breakable]
\textbf{Grammatical Structure:}\\
Past verb + proper noun subject + preposition + indefinite noun (genitive) + prepositional phrase + adverbial accusative + conjunction + past verb + preposition + definite noun construct (genitive) + conjunction + Form VIII past verb + prepositional phrase with pronoun \\

\textbf{Key Grammar Points:}
\begin{itemize}
\item \textarabic{شَعَرَ بِـ} requires the preposition \textarabic{بِ} to mean "felt/sensed"
\item \textarabic{لَيْلًا} is an adverbial accusative indicating time
\item \textarabic{فَقَامَ} uses the conjunction \textarabic{فَ} to show consequence/sequence
\item \textarabic{غُرْفَةِ البُومِ} is an \textarabic{إضافة} (possessive construct) - "room of the owl"
\item \textarabic{اخْتَبَأَ} is Form VIII indicating reflexive action (hid himself)
\item \textarabic{فِيهَا} refers back to the closet (feminine pronoun agreement)
\item The narrative follows chronological sequence: sensation → reaction → movement → hiding
\item This sets up the classic Juha scenario where his unconventional response will follow
\end{itemize}
\end{tcolorbox}

% ======================== Similar Phrases for Practice ========================
\section{Similar Phrases for Practice}

\begin{enumerate}
\item \textarabic{سَمِعَ عَلِيٌّ صَوْتًا فِي بَيْتِهِ لَيْلًا، فَذَهَبَ إِلَى المَطْبَخِ وَاخْتَبَأَ خَلْفَ البَابِ}\\
Ali heard a sound in his house at night, so he went to the kitchen and hid behind the door [samiʕa ʕalijjun sˤawtan fiː bajtihi lajlan, faðahaba ʔila l-matˤbaxi waxtabaʔa xalfa l-baːbi]

\item \textarabic{أَحَسَّتْ فَاطِمَةُ بِحَرَكَةٍ فِي الحَدِيقَةِ صَبَاحًا، فَمَشَتْ إِلَى النَّافِذَةِ وَنَظَرَتْ مِنْهَا}\\
Fatima sensed movement in the garden in the morning, so she walked to the window and looked through it [ʔaħassat faːtˤimatun biħarakatin fiː l-ħadiːqati sˤabaːħan, famaʃat ʔila n-naːfiðati wanaðarat minhaː]

\item \textarabic{شَعَرَ الحَارِسُ بِخَطَرٍ فِي المَتْحَفِ مَسَاءً، فَرَكَضَ إِلَى غُرْفَةِ المُرَاقَبَةِ وَاتَّصَلَ بِالشُّرْطَةِ}\\
The guard felt danger in the museum in the evening, so he ran to the surveillance room and called the police [ʃaʕara l-ħaːrisu bixatˤarin fiː l-matħafi masaːʔan, farakaðˤa ʔila ɣurfati l-muraːqabati wattasˤala biʃ-ʃurtˤati]

\item \textarabic{لَمَسَ الوَلَدُ شَيْئًا غَرِيبًا فِي الصُّنْدُوقِ ظُهْرًا، فَفَتَحَ الغِطَاءَ وَوَجَدَ كَنْزًا قَدِيمًا}\\
The boy touched something strange in the box at noon, so he opened the lid and found an old treasure [lamasa l-waladu ʃajʔan ɣariːban fiː sˤ-sˤunduːqi ðˤuhran, fafataħa l-ɣitˤaːʔa wawadʒada kanzan qadiːman]
\end{enumerate}

% ======================== Levantine Dialect Version ========================
\section{Levantine (Shaami) Arabic Dialect}

\begin{tcolorbox}[colback=white,colframe=dialectcolor,title=\textbf{Levantine Version},breakable]
\textarabic{حَسَّ جُحا بِحَرَامِيّ بِبَيْتُه بِاللَّيْل، فَقَامَ عَ خِزَانة أُوضِة البُومَة وَخَبَّى حَالُه جُوَّاتَهَا}\\
\textbf{Phonetic:} [ħass dʒuħaː biħaraːmijj bibejtu bil-leel, faqaːm ʕa xizaːnet ʔoːðˤet el-buːme wxabba ħaːlu dʒuwwaːtehaː]\\
\textbf{Translation:} Juha sensed a thief in his house at night, so he went to the closet of the owl's room and hid himself inside it.
\end{tcolorbox}

\textbf{Key Dialectal Changes:}
\begin{itemize}
\item \textarabic{شَعَرَ} → \textarabic{حَسَّ} (ħass) — dialectal verb for "sensed"
\item \textarabic{لِصّ} → \textarabic{حَرَامِيّ} (ħaraːmijj) — common dialectal word for "thief"
\item \textarabic{دَارِهِ} → \textarabic{بَيْتُه} (bejtu) — "house" in dialect form
\item \textarabic{لَيْلًا} → \textarabic{بِاللَّيْل} (bil-leel) — "at night" with definite article
\item \textarabic{إِلَى} → \textarabic{عَ} (ʕa) — shortened preposition "to"
\item \textarabic{غُرْفَة} → \textarabic{أُوضِة} (ʔoːðˤa) — dialectal word for "room"
\item \textarabic{اخْتَبَأَ} → \textarabic{خَبَّى حَالُه} (xabba ħaːlu) — "hid himself" in dialect
\item \textarabic{فِيهَا} → \textarabic{جُوَّاتَهَا} (dʒuwwaːtehaː) — "inside it" in dialect
\end{itemize}

% ======================== Additional Learning Notes ========================
\section{Additional Learning Notes}

\begin{tcolorbox}[colback=boxcolor,colframe=accentcolor,title=\textbf{Cultural and Literary Context},breakable]
\textbf{Juha's Character:} This opening sets the stage for classic Juha humor. Instead of confronting the thief or calling for help, Juha chooses to hide — an unexpected response that will lead to an ironic twist.

\textbf{The Owl Room:} The mention of \textarabic{غُرْفَة البُوم} (owl room) adds folkloric atmosphere. Owls in Middle Eastern culture often symbolize wisdom or mystery, fitting for Juha's unconventional wisdom.

\textbf{Narrative Technique:} The sequence uses \textarabic{فَ} (fa-) to create rapid succession of events: sensing → standing → moving → hiding, building suspense for the comedic resolution.

\textbf{Social Commentary:} The story subtly comments on how the poor (Juha) respond to theft — not with anger or fear, but with an awareness of their own poverty that leads to empathy.
\end{tcolorbox}

\subsection{Memory Tips}
\begin{enumerate}
\item \textbf{Sensory Progression:} Remember \textit{Feel → Stand → Move → Hide} (\textarabic{شُعُور} → \textarabic{قِيَام} → \textarabic{حَرَكَة} → \textarabic{اخْتِبَاء})
\item \textbf{Preposition Pattern:} \textarabic{بِ} (with/by) → \textarabic{فِي} (in) → \textarabic{إِلَى} (to) → \textarabic{فِي} (in)
\item \textbf{Time Marker:} \textarabic{لَيْلًا} in accusative shows "when" something happens
\item \textbf{Conjunction Chain:} No conjunction → \textarabic{فَ} (so/then) → \textarabic{وَ} (and) creates narrative flow
\item \textbf{Form VIII Pattern:} \textarabic{اخْتَبَأَ} follows \textarabic{افْتَعَلَ} pattern for reflexive actions
\end{enumerate}

\subsection{Related Vocabulary Family}
\begin{multicols}{2}

\textbf{Sensation family:}\\
\textarabic{شُعُور} — feeling [ʃuʕuːr]\\
\textarabic{إِحْسَاس} — sensation [ʔiħsaːs]\\
\textarabic{لَمْس} — touch [lams]\\
\textarabic{سَمْع} — hearing [samʕ]\\
\textarabic{رُؤْيَة} — sight [ruʔja]\\

\textbf{Movement family:}\\
\textarabic{قِيَام} — standing [qijaːm]\\
\textarabic{مَشْي} — walking [maʃj]\\
\textarabic{جَرْي} — running [dʒarj]\\
\textarabic{حَرَكَة} — movement [ħaraka]\\
\textarabic{انْتِقَال} — transition [ʔintiqaːl]\\

\textbf{House parts family:}\\
\textarabic{غُرْفَة} — room [ɣurfa]\\
\textarabic{مَطْبَخ} — kitchen [matˤbax]\\
\textarabic{حَمَّام} — bathroom [ħammaːm]\\
\textarabic{صَالَة} — living room [sˤaːla]\\
\textarabic{شُرْفَة} — balcony [ʃurfa]\\

\textbf{Hiding family:}\\
\textarabic{إِخْفَاء} — concealment [ʔixfaːʔ]\\
\textarabic{اسْتِتَار} — hiding [ʔisitaːr]\\
\textarabic{تَوَارِي} — disappearance [tawaːriː]\\
\textarabic{كُمُون} — lurking [kumuːn]\\
\textarabic{احْتِجَاب} — veiling [ʔiħtidʒaːb]\\

\textbf{Time expressions:}\\
\textarabic{لَيْلًا} — at night [lajlan]\\
\textarabic{نَهَارًا} — during the day [nahaːran]\\
\textarabic{صَبَاحًا} — in the morning [sˤabaːħan]\\
\textarabic{مَسَاءً} — in the evening [masaːʔan]\\
\textarabic{ظُهْرًا} — at noon [ðˤuhran]\\

\textbf{Animals family:}\\
\textarabic{بُوم} — owl [buːm]\\
\textarabic{عُصْفُور} — sparrow [ʕusˤfuːr]\\
\textarabic{نَسْر} — eagle [nasr]\\
\textarabic{غُرَاب} — crow [ɣuraːb]\\
\textarabic{صَقْر} — falcon [sˤaqr]
\end{multicols}

\subsection{Grammar Focus Points}

\begin{tcolorbox}[colback=white,colframe=headercolor,title=\textbf{Key Grammatical Constructions},breakable]
\textbf{1. Sensation Verbs with Prepositions:}
\begin{itemize}
\item \textarabic{شَعَرَ بِـ} (felt/sensed with)
\item \textarabic{أَحَسَّ بِـ} (felt with)
\item \textarabic{سَمِعَ ـــ} (heard - direct object)
\item \textarabic{رَأَى ـــ} (saw - direct object)
\end{itemize}

\textbf{2. Adverbial Accusative for Time:}
\begin{itemize}
\item \textarabic{لَيْلًا} (at night) - indefinite accusative
\item \textarabic{صَبَاحًا} (in the morning) - indefinite accusative
\item \textarabic{فِي اللَّيْلِ} (at night) - prepositional phrase alternative
\end{itemize}

\textbf{3. Possessive Constructs (\textarabic{إضافة}):}
\begin{itemize}
\item \textarabic{دَارِهِ} (his house) - noun + possessive pronoun
\item \textarabic{غُرْفَةِ البُومِ} (owl's room) - first noun definite by association
\item Both nouns in genitive case in construct
\end{itemize}

\textbf{4. Form VIII Verbs:}
\begin{itemize}
\item Pattern: \textarabic{افْتَعَلَ} often indicates reflexive/intensive action
\item \textarabic{اخْتَبَأَ} (he hid himself) - reflexive hiding
\item \textarabic{اجْتَمَعَ} (they gathered) - mutual action
\end{itemize}
\end{tcolorbox}

\subsection{Morphological Analysis Summary}

\begin{tcolorbox}[colback=white,colframe=accentcolor,title=\textbf{Morphological Breakdown},breakable]
\begin{tabular}{|p{3cm}|p{3cm}|p{3cm}|p{4cm}|}
\hline
\textbf{Word} & \textbf{Root} & \textbf{Pattern} & \textbf{Morphological Category} \\
\hline
\textarabic{شَعَرَ} & \textarabic{ش-ع-ر} & \textarabic{فَعَلَ} & Form I Perfect Verb \\
\hline
\textarabic{جُحا} & — & \textarabic{فُعَلَ} & Proper Noun \\
\hline
\textarabic{بِلِصٍّ} & \textarabic{ل-ص-ص} & \textarabic{بِفِعْل} & Prep. + Indef. Noun \\
\hline
\textarabic{فِي دَارِهِ} & \textarabic{د-و-ر} & \textarabic{فِي فَاعِل} & Prep. + Poss. Noun \\
\hline
\textarabic{لَيْلًا} & \textarabic{ل-ي-ل} & \textarabic{فَعْلًا} & Adverbial Accusative \\
\hline
\textarabic{فَقَامَ} & \textarabic{ق-و-م} & \textarabic{فَفَعَلَ} & Conj. + Perfect Verb \\
\hline
\textarabic{إِلَى خِزَانَةِ} & \textarabic{خ-ز-ن} & \textarabic{إِلَى فِعَالَة} & Prep. + Def. Noun \\
\hline
\textarabic{غُرْفَةِ البُومِ} & \textarabic{غ-ر-ف/ب-و-م} & \textarabic{فُعْلَة فُعْل} & Construct State \\
\hline
\textarabic{وَاخْتَبَأَ} & \textarabic{خ-ب-أ} & \textarabic{وَافْتَعَلَ} & Conj. + Form VIII \\
\hline
\textarabic{فِيهَا} & — & \textarabic{فِي + ضمير} & Prep. + Pronoun \\
\hline
\end{tabular}
\end{tcolorbox}

\subsection{Syntactic Analysis}

\begin{tcolorbox}[colback=boxcolor,colframe=headercolor,title=\textbf{Syntactic Tree Structure},breakable]
\textbf{Sentence 1:} \textarabic{شَعَرَ جُحا بِلِصٍّ فِي دَارِهِ لَيْلًا}

\begin{itemize}
\item \textbf{Main Clause:} Verbal sentence (\textarabic{جملة فعلية})
  \begin{itemize}
  \item \textbf{Predicate:} \textarabic{شَعَرَ} (past tense verb)
  \item \textbf{Subject:} \textarabic{جُحا} (proper noun, \textarabic{فاعل})
  \item \textbf{Prepositional Object:} \textarabic{بِلِصٍّ} (\textarabic{با + مجرور})
  \item \textbf{Locative Adjunct:} \textarabic{فِي دَارِهِ} (\textarabic{في + مجرور})
  \item \textbf{Temporal Adjunct:} \textarabic{لَيْلًا} (\textarabic{ظرف زمان منصوب})
  \end{itemize}
\end{itemize}

\textbf{Sentence 2:} \textarabic{فَقَامَ إِلَى خِزَانَةِ غُرْفَةِ البُومِ وَاخْتَبَأَ فِيهَا}

\begin{itemize}
\item \textbf{Coordinated Clauses:} Two verbal sentences linked by \textarabic{وَ}
  \begin{itemize}
  \item \textbf{Clause 1:}
    \begin{itemize}
    \item \textbf{Conjunction:} \textarabic{فَ} (consequential)
    \item \textbf{Predicate:} \textarabic{قَامَ} (past tense verb)
    \item \textbf{Subject:} \textit{implied pronoun} (\textarabic{ضمير مستتر})
    \item \textbf{Directional Adjunct:} \textarabic{إِلَى خِزَانَةِ غُرْفَةِ البُومِ}
    \end{itemize}
  \item \textbf{Clause 2:}
    \begin{itemize}
    \item \textbf{Conjunction:} \textarabic{وَ} (additive)
    \item \textbf{Predicate:} \textarabic{اخْتَبَأَ} (Form VIII past verb)
    \item \textbf{Subject:} \textit{implied pronoun} (\textarabic{ضمير مستتر})
    \item \textbf{Locative Adjunct:} \textarabic{فِيهَا} (\textarabic{في + ضمير})
    \end{itemize}
  \end{itemize}
\end{itemize}
\end{tcolorbox}

\subsection{Phonological Analysis}

\begin{tcolorbox}[colback=white,colframe=dialectcolor,title=\textbf{Phonological Features},breakable]
\textbf{Emphasis Spread (Tafkheem):}
\begin{itemize}
\item \textarabic{لِصّ} [lisˤsˤ] — The emphatic /sˤ/ spreads to adjacent vowels
\item No other emphatic consonants in this phrase
\end{itemize}

\textbf{Assimilation Patterns:}
\begin{itemize}
\item \textarabic{اخْتَبَأَ} — The /t/ of the Form VIII prefix assimilates with certain root initials
\item \textarabic{بِلِصّ} — The preposition vowel harmonizes with following consonant
\end{itemize}

\textbf{Syllable Structure Analysis:}
\begin{itemize}
\item \textarabic{شَعَرَ} — [ʃa.ʕa.ra] — CV.CV.CV (open syllables)
\item \textarabic{جُحا} — [dʒu.ħaː] — CV.CVː (long vowel final)
\item \textarabic{لَيْلًا} — [laj.lan] — CVC.CVC (closed syllables)
\end{itemize}

\textbf{Stress Patterns:}
\begin{itemize}
\item Penultimate stress in most words: [ʃaˈʕara], [qaːˈma]
\item Final stress on long vowels: [dʒuˈħaː]
\item Antepenultimate stress on some forms: [ʔixˈtabaʔa]
\end{itemize}
\end{tcolorbox}

\subsection{Register and Style Analysis}

\begin{tcolorbox}[colback=boxcolor,colframe=accentcolor,title=\textbf{Linguistic Register},breakable]
\textbf{Classical Arabic Features:}
\begin{itemize}
\item Full case marking (\textarabic{إعراب}): \textarabic{لِصٍّ}, \textarabic{دَارِهِ}, \textarabic{خِزَانَةِ}
\item Classical verb forms: \textarabic{شَعَرَ}, \textarabic{قَامَ}
\item Formal sentence structure with proper conjunctions
\item Literary vocabulary: \textarabic{غُرْفَة البُوم} (poetic/folkloric)
\end{itemize}

\textbf{Narrative Style Markers:}
\begin{itemize}
\item Sequential conjunctions: \textarabic{فَ} (then), \textarabic{وَ} (and)
\item Past tense throughout (narrative perfect)
\item Third person perspective (omniscient narrator)
\item Folkloric elements: character name \textarabic{جُحا}, setting details
\end{itemize}

\textbf{Semantic Field:}
\begin{itemize}
\item \textbf{Sensation:} \textarabic{شَعَرَ} (felt)
\item \textbf{Crime:} \textarabic{لِصّ} (thief)
\item \textbf{Domestic space:} \textarabic{دَار} (house), \textarabic{غُرْفَة} (room), \textarabic{خِزَانَة} (closet)
\item \textbf{Movement:} \textarabic{قَامَ} (stood/went), \textarabic{اخْتَبَأَ} (hid)
\item \textbf{Time:} \textarabic{لَيْلًا} (at night)
\end{itemize}
\end{tcolorbox}

\subsection{Pedagogical Applications}

\begin{tcolorbox}[colback=white,colframe=headercolor,title=\textbf{Teaching Applications},breakable]
\textbf{Grammar Points for Intermediate Learners:}
\begin{enumerate}
\item \textbf{Preposition Usage:} \textarabic{شَعَرَ بِ} vs. \textarabic{سَمِعَ} (direct object)
\item \textbf{Adverbial Accusative:} \textarabic{لَيْلًا}, \textarabic{صَبَاحًا}, \textarabic{مَسَاءً}
\item \textbf{Possessive Constructs:} \textarabic{دَارِهِ} vs. \textarabic{غُرْفَةِ البُومِ}
\item \textbf{Form VIII Verbs:} Pattern recognition and meaning
\item \textbf{Narrative Conjunctions:} \textarabic{فَ} for sequence, \textarabic{وَ} for addition
\end{enumerate}

\textbf{Cultural Learning Objectives:}
\begin{enumerate}
\item Understanding Juha as a folkloric character across Arab cultures
\item Recognizing elements of traditional storytelling
\item Appreciating humor through character response expectations
\item Exploring themes of poverty, wisdom, and social commentary
\end{enumerate}

\textbf{Phonetic Training Focus:}
\begin{enumerate}
\item Emphatic consonant production: /sˤ/ in \textarabic{لِصّ}
\item Pharyngeal fricative: /ħ/ in \textarabic{جُحا}
\item Glottal stop: /ʔ/ in \textarabic{اخْتَبَأَ}
\item Vowel length distinction: /aː/ vs. /a/
\end{enumerate}
\end{tcolorbox}

\subsection{Comparative Analysis with Other Semitic Languages}

\begin{tcolorbox}[colback=boxcolor,colframe=dialectcolor,title=\textbf{Semitic Language Connections},breakable]
\textbf{Hebrew Cognates:}
\begin{itemize}
\item \textarabic{شعر} (Arabic) ↔ \texthebrew{שער} (Hebrew) — both meaning "hair/sense"
\item \textarabic{ليل} (Arabic) ↔ \texthebrew{לילה} (Hebrew) — "night"
\item \textarabic{دار} (Arabic) ↔ \texthebrew{דור} (Hebrew) — "generation/circle"
\item \textarabic{قوم} (Arabic) ↔ \texthebrew{קום} (Hebrew) — "rise/stand"
\end{itemize}

\textbf{Aramaic Connections:}
\begin{itemize}
\item Root \textarabic{ق-و-م} appears in Aramaic \textit{qam} (stood)
\item The construct state pattern mirrors Aramaic \textit{emphata} forms
\item Form VIII pattern has parallels in Aramaic intensive stems
\end{itemize}

\textbf{Proto-Semitic Reconstructions:}
\begin{itemize}
\item *\textit{šaʕar-} → \textarabic{شَعَرَ} (to feel/perceive)
\item *\textit{lajl-} → \textarabic{لَيْل} (night)
\item *\textit{bayt-} → \textarabic{بَيْت} (house - cf. \textarabic{دار})
\end{itemize}
\end{tcolorbox}

\subsection{Historical Linguistic Development}

\begin{tcolorbox}[colback=white,colframe=accentcolor,title=\textbf{Diachronic Analysis},breakable]
\textbf{Classical vs. Modern Standard Arabic:}
\begin{itemize}
\item All forms in this text remain standard in Modern Standard Arabic
\item \textarabic{دار} is more formal/literary; \textarabic{بيت} is more common in MSA
\item \textarabic{لص} maintains the same form and meaning across periods
\item Form VIII verbs like \textarabic{اختبأ} are fully productive in MSA
\end{itemize}

\textbf{Dialectal Evolution Patterns:}
\begin{itemize}
\item Case endings lost: \textarabic{لصٍّ} → \textit{liṣṣ} (no tanween)
\item Verb simplification: \textarabic{اختبأ} → \textarabic{خبّى} (Levantine)
\item Preposition changes: \textarabic{إلى} → \textarabic{عَ} (Levantine)
\item Lexical substitution: \textarabic{غرفة} → \textarabic{أوضة} (Egyptian/Levantine)
\end{itemize}

\textbf{Borrowing and Innovation:}
\begin{itemize}
\item \textarabic{جحا} ultimately from Persian, showing cultural transmission
\item Core vocabulary (body parts, house, time) remains Semitic
\item Storytelling formulas preserved across Arabic literature
\end{itemize}
\end{tcolorbox}

\subsection{Advanced Exercises}

\begin{tcolorbox}[colback=boxcolor,colframe=headercolor,title=\textbf{Practice Activities},breakable]
\textbf{Morphological Exercises:}
\begin{enumerate}
\item Derive all forms of \textarabic{ش-ع-ر} in Forms I-X
\item Create the passive voice of all verbs in the text
\item Decline \textarabic{خزانة} with all possessive pronoun suffixes
\item Form the dual and plural of \textarabic{غرفة البوم}
\end{enumerate}

\textbf{Syntactic Transformation:}
\begin{enumerate}
\item Convert the text to present tense narrative
\item Change the subject to \textarabic{فاطمة} (feminine) and adjust agreement
\item Transform into conditional sentences: "If Juha had felt..."
\item Create questions for each clause using appropriate interrogatives
\end{enumerate}

\textbf{Semantic Expansion:}
\begin{enumerate}
\item Replace \textarabic{شعر} with \textarabic{سمع}, \textarabic{رأى}, \textarabic{وجد} and adjust syntax
\item Substitute different times of day and adjust atmosphere
\item Replace \textarabic{لص} with \textarabic{ضيف} (guest) and note how story changes
\item Add descriptive adjectives to each noun
\end{enumerate}

\textbf{Cultural Research Projects:}
\begin{enumerate}
\item Compare this Juha story across different Arab countries
\item Research the historical figure behind Juha/Nasreddin
\item Collect similar trickster figures from other cultures
\item Analyze the role of humor in Arab folk literature
\end{enumerate}
\end{tcolorbox}

\section{Bibliography and Further Reading}

\begin{tcolorbox}[colback=white,colframe=headercolor,title=\textbf{Recommended Sources},breakable]
\textbf{Arabic Grammar References:}
\begin{itemize}
\item Wright, William. \textit{A Grammar of the Arabic Language}. Cambridge University Press.
\item Ryding, Karin C. \textit{A Reference Grammar of Modern Standard Arabic}. Cambridge University Press.
\item Holes, Clive. \textit{Modern Arabic: Structures, Functions, and Varieties}. Georgetown University Press.
\end{itemize}

\textbf{Folklore and Cultural Studies:}
\begin{itemize}
\item Marzolph, Ulrich. \textit{The Wise Fool: Nasreddin Hodja in World Literature}. Garland Publishing.
\item El-Shamy, Hasan M. \textit{Folk Traditions of the Arab World}. Indiana University Press.
\item Bushnaq, Inea. \textit{Arab Folktales}. Pantheon Books.
\end{itemize}

\textbf{Linguistic Analysis:}
\begin{itemize}
\item Versteegh, Kees. \textit{The Arabic Language}. Edinburgh University Press.
\item Fischer, Wolfdietrich. \textit{Classical Arabic}. Yale University Press.
\item Owens, Jonathan. \textit{A Linguistic History of Arabic}. Oxford University Press.
\end{itemize}

\textbf{Dialectology:}
\begin{itemize}
\item Brustad, Kristen. \textit{The Syntax of Spoken Arabic}. Georgetown University Press.
\item Procházka, Stephan. \textit{The Arabic Dialects of the Levant}. Reichert Verlag.
\item Holes, Clive. \textit{Dialect, Culture, and Society in Eastern Arabia}. Brill.
\end{itemize}
\end{tcolorbox}

\section*{Acknowledgments}

This analysis was prepared as part of advanced Arabic linguistic studies, drawing on traditional Arabic grammatical analysis (\textarabic{النحو والصرف}) combined with modern linguistic methodologies. The cultural interpretations reflect established scholarship on Arab folklore and the Juha/Nasreddin tradition.

\vfill
\begin{center}
\textit{Compiled with XeLaTeX using Amiri Arabic font}\\
\textit{For educational and research purposes}
\end{center}

\end{document}