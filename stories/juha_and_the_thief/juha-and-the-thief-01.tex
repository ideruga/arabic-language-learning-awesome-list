\documentclass[letter,12pt]{article}

\usepackage[most]{tcolorbox}
\usepackage{makecell}
\usepackage[utf8]{inputenc}
\usepackage[english,arabic]{babel}
\usepackage{tikz}
\usepackage{geometry}
\usepackage{array}
\usepackage{multicol}
\usepackage[table]{xcolor}
\usepackage{polyglossia}
\setmainlanguage{english}
\setotherlanguage{arabic}
\setotherlanguage{hebrew}
\usepackage{arabtex}
\usepackage{fontspec}
\setmainfont{Charis SIL}
\newfontfamily\arabicfont[Script=Arabic,Scale=1.2]{Amiri}
\newfontfamily\hebrewfont{Ezra SIL}[Script=Hebrew]
\geometry{margin=1.5cm}
\usepackage{booktabs}
\usepackage{graphicx}

\usetikzlibrary{shapes,arrows,decorations.pathmorphing}

% Custom colors
\definecolor{headercolor}{RGB}{70,130,180}
\definecolor{boxcolor}{RGB}{240,248,255}
\definecolor{accentcolor}{RGB}{220,20,60}
\definecolor{tableheader}{RGB}{220,220,220}
\definecolor{dialectcolor}{RGB}{34,139,34}

\begin{document}

\title{\textbf{\Large Arabic Phrase Analyser}\\
\large Juha Hides from a Thief in His House\\
\normalsize \textarabic{شَعَرَ جُحا بِلِصٍّ في دَارِهِ لَيْلًا، فَقَامَ إِلَى خِزَانَةِ غُرْفَةِ البُومِ وَاخْتَبَأَ فِيهَا}}
\author{Igor Deruga}
\date{}
\maketitle

% ======================== Phrase Display ========================
\begin{tcolorbox}[colback=boxcolor,colframe=headercolor,title=\textbf{Arabic Phrase}]
\centering
\textarabic{شعر جحا بلص في داره ليلا، فقام إلى خزانة غرفة البوم واختبأ فيها}
\\[0.5em]
\textbf{Without Diacritics}
\\[1em]
\textarabic{شَعَرَ جُحا بِلِصٍّ في دَارِهِ لَيْلًا، فَقَامَ إِلَى خِزَانَةِ غُرْفَةِ البُومِ وَاخْتَبَأَ فِيهَا}
\\[0.5em]
\textbf{With Full Diacritics}
\end{tcolorbox}

% ======================== Translations ========================
\section{English Translation}
\begin{tcolorbox}[colback=white,colframe=accentcolor]
\textbf{Literal:} Felt Juha with-thief in house-his night, so-stood to closet room the-owl and-hid in-it \\
\textit{[Arabic order retained for direct mapping]}\\[0.5em]
\textbf{Adapted:} Juha felt a thief in his house at night, so he went to the closet of the owl room and hid in it
\end{tcolorbox}

% ======================== Detailed Word Analysis ========================
\section{Detailed Word Analysis}

\subsection{\textarabic{شَعَرَ} – [ʃaʕara]}
\begin{tabular}{p{3cm}p{10cm}}
\toprule
\textbf{Translation} & felt / sensed \\
\textbf{Root} & \textarabic{ش-ع-ر} (ʃ-ʕ-r) \\
\textbf{Pattern} & \textarabic{فَعَلَ} (faʕala) \\
\textbf{Grammar} & Past tense verb, 3rd person masculine singular, active voice \\
\midrule \\
\textbf{Table of Conjugations} & \makecell[l]{
Infinitive: \textarabic{شُعُورٌ} [ʃuʕuːr] \\
Present: \textarabic{يَشْعُرُ} [jaʃʕuru] \\
Past: \textarabic{شَعَرَ} [ʃaʕara]
} \\
\midrule
\textbf{Examples} & \makecell[l]{\parbox{9.5cm}{
1. \textarabic{شَعَرَ الرَّجُلُ بِالْأَلَمِ} - The man felt pain [ʃaʕara r-radʒulu bil-ʔalam]\\
2. \textarabic{يَشْعُرُ بِالسَّعَادَةِ} - He feels happiness [jaʃʕuru bis-saʕaːda]\\
3. \textarabic{سَيَشْعُرُ بِالنَّدَمِ} - He will feel regret [sajaʃʕuru bin-nadam]
}} \\
\midrule \\
\textbf{Synonyms} & \textarabic{أَحَسَّ} (felt), \textarabic{وَجَدَ} (found/felt), \textarabic{لَمَسَ} (touched/sensed) \\
\textbf{Etymology} & From Proto-Semitic *ʃaʕar-, related to Hebrew \texthebrew{שער} (shaʕar) "hair/sense" \\
\bottomrule
\end{tabular}

\subsection{\textarabic{جُحا} – [dʒuħaː]}
\begin{tabular}{p{3cm}p{10cm}}
\toprule
\textbf{Translation} & Juha (proper name of folkloric character) \\
\textbf{Root} & Proper noun, no trilateral root \\
\textbf{Pattern} & \textarabic{فُعَلَ} pattern for names \\
\textbf{Grammar} & Proper noun, masculine, subject of the verb \\
\midrule \\
\textbf{Table of Conjugations} & \makecell[l]{
Nominative: \textarabic{جُحا} [dʒuħaː] \\
Genitive: \textarabic{جُحا} [dʒuħaː] \\
Accusative: \textarabic{جُحا} [dʒuħaː]
} \\
\midrule
\textbf{Examples} & \makecell[l]{\parbox{9.5cm}{
1. \textarabic{قَالَ جُحا شَيْئًا مُضْحِكًا} - Juha said something funny [qaːla dʒuħaː ʃajʔan mudħikan]\\
2. \textarabic{حِكَايَاتُ جُحا مَشْهُورَةٌ} - Juha's stories are famous [ħikaːjaːtu dʒuħaː maʃhuːra]\\
3. \textarabic{يُحِبُّ الأَطْفَالُ قِصَصَ جُحا} - Children love Juha's stories [juħibbu l-ʔatfaːlu qisas dʒuħaː]
}} \\
\midrule \\
\textbf{Synonyms} & \textarabic{نَصْرُ الدِّينِ} (Nasreddin), regional variations of the same character \\
\textbf{Etymology} & Persian origin, possibly from Jahā "world" or related to Turkish Hoca "teacher" \\
\bottomrule
\end{tabular}

\subsection{\textarabic{بِلِصٍّ} – [bilis̱s̱in]}
\begin{tabular}{p{3cm}p{10cm}}
\toprule
\textbf{Translation} & with a thief \\
\textbf{Root} & \textarabic{ل-ص-ص} (l-sˤ-sˤ) \\
\textbf{Pattern} & \textarabic{فِعْل} (fiʕl) with preposition \textarabic{بِ} \\
\textbf{Grammar} & Preposition + indefinite noun, genitive case, masculine singular \\
\midrule \\
\textbf{Table of Conjugations} & \makecell[l]{
Nominative: \textarabic{لِصٌّ} [lis̱s̱un] \\
Genitive: \textarabic{لِصٍّ} [lis̱s̱in] \\
Accusative: \textarabic{لِصًّا} [lis̱s̱an]
} \\
\midrule
\textbf{Examples} & \makecell[l]{\parbox{9.5cm}{
1. \textarabic{قَبَضَ الشُّرْطِيُّ عَلَى اللِّصِّ} - The policeman caught the thief [qabaða ʃ-ʃurtˤijju ʕala l-lis̱s̱i]\\
2. \textarabic{دَخَلَ لِصٌّ إِلَى البَيْتِ} - A thief entered the house [daxala lis̱s̱un ʔila l-bajti]\\
3. \textarabic{سَيُعَاقِبُ القَاضِي اللُّصُوصَ} - The judge will punish the thieves [sajuʕaːqibu l-qaːɮˤi l-lus̱uːs̱a]
}} \\
\midrule \\
\textbf{Synonyms} & \textarabic{سَارِق} (thief), \textarabic{حَرَامِيّ} (burglar), \textarabic{نَشَّال} (pickpocket) \\
\textbf{Etymology} & From root meaning "to stick/adhere" - one who "sticks" to others' property \\
\bottomrule
\end{tabular}

\subsection{\textarabic{في دَارِهِ} – [fiː daːrihi]}
\begin{tabular}{p{3cm}p{10cm}}
\toprule
\textbf{Translation} & in his house \\
\textbf{Root} & \textarabic{د-و-ر} (d-w-r) \\
\textbf{Pattern} & \textarabic{فَاعِل} (faːʕil) + possessive pronoun \\
\textbf{Grammar} & Preposition + definite noun in genitive + 3rd person masculine singular possessive \\
\midrule \\
\textbf{Table of Conjugations} & \makecell[l]{
My house: \textarabic{دَارِي} [daːriː] \\
Your house (m.): \textarabic{دَارُكَ} [daːruka] \\
His house: \textarabic{دَارُهُ} [daːruhu]
} \\
\midrule
\textbf{Examples} & \makecell[l]{\parbox{9.5cm}{
1. \textarabic{دَارُ الأُسْتَاذِ كَبِيرَةٌ} - The professor's house is big [daːru l-ʔustaːði kabiːra]\\
2. \textarabic{سَأَزُورُ دَارَكَ غَدًا} - I will visit your house tomorrow [saʔazuːru daːraka ɣadan]\\
3. \textarabic{بَنَى دَارًا جَدِيدَةً} - He built a new house [banaː daːran dʒadiːda]
}} \\
\midrule \\
\textbf{Synonyms} & \textarabic{بَيْت} (house), \textarabic{مَنْزِل} (residence), \textarabic{مَسْكَن} (dwelling) \\
\textbf{Etymology} & From root d-w-r meaning "to turn/circle" - the place one circles back to \\
\bottomrule
\end{tabular}

\subsection{\textarabic{لَيْلًا} – [lajlan]}
\begin{tabular}{p{3cm}p{10cm}}
\toprule
\textbf{Translation} & at night / nocturnally \\
\textbf{Root} & \textarabic{ل-ي-ل} (l-j-l) \\
\textbf{Pattern} & \textarabic{فَعْل} (fajl) in accusative (adverbial) \\
\textbf{Grammar} & Adverbial accusative of time, indefinite \\
\midrule \\
\textbf{Table of Conjugations} & \makecell[l]{
Nominative: \textarabic{لَيْلٌ} [lajlun] \\
Genitive: \textarabic{لَيْلٍ} [lajlin] \\
Accusative: \textarabic{لَيْلًا} [lajlan]
} \\
\midrule
\textbf{Examples} & \makecell[l]{\parbox{9.5cm}{
1. \textarabic{نَامَ الطِّفْلُ لَيْلًا} - The child slept at night [naːma t̪-t̪iflu lajlan]\\
2. \textarabic{اللَّيْلُ طَوِيلٌ} - The night is long [al-lajlu t̪awiːl]\\
3. \textarabic{سَيَسْهَرُ في اللَّيْلِ} - He will stay up at night [sajasharu fiː l-lajli]
}} \\
\midrule \\
\textbf{Synonyms} & \textarabic{عَشِيَّة} (evening), \textarabic{ظَلَام} (darkness), \textarabic{دُجَى} (dark night) \\
\textbf{Etymology} & From Proto-Semitic *lajl-, related to Hebrew \texthebrew{לילה} (lajla) "night" \\
\bottomrule
\end{tabular}

\subsection{\textarabic{فَقَامَ} – [faqaːma]}
\begin{tabular}{p{3cm}p{10cm}}
\toprule
\textbf{Translation} & so he stood up / then he got up \\
\textbf{Root} & \textarabic{ق-و-م} (q-w-m) \\
\textbf{Pattern} & \textarabic{فَعَلَ} (faʕala) with conjunction prefix \\
\textbf{Grammar} & Past tense verb, 3rd person masculine singular, with \textarabic{فَ} conjunction \\
\midrule \\
\textbf{Table of Conjugations} & \makecell[l]{
Infinitive: \textarabic{قِيَامٌ} [qijaːm] \\
Present: \textarabic{يَقُومُ} [jaquːmu] \\
Past: \textarabic{قَامَ} [qaːma]
} \\
\midrule
\textbf{Examples} & \makecell[l]{\parbox{9.5cm}{
1. \textarabic{قَامَ الرَّجُلُ مِنْ كُرْسِيِّهِ} - The man got up from his chair [qaːma r-radʒulu min kursijjihi]\\
2. \textarabic{يَقُومُ كُلَّ صَبَاحٍ بَاكِرًا} - He gets up early every morning [jaquːmu kulla s̱abaːħin baːkiran]\\
3. \textarabic{قَامُوا جَمِيعًا} - They all stood up [qaːmuː dʒamiːʕan]
}} \\
\midrule \\
\textbf{Synonyms} & \textarabic{وَقَفَ} (stood), \textarabic{نَهَضَ} (rose up), \textarabic{اِسْتَقَامَ} (straightened up) \\
\textbf{Etymology} & From Proto-Semitic *qwm, related to Hebrew \texthebrew{קום} (qum) "rise" \\
\bottomrule
\end{tabular}

\subsection{\textarabic{إِلَى خِزَانَةِ} – [ʔilaː xizaːnati]}
\begin{tabular}{p{3cm}p{10cm}}
\toprule
\textbf{Translation} & to the closet / toward the cabinet \\
\textbf{Root} & \textarabic{خ-ز-ن} (x-z-n) \\
\textbf{Pattern} & \textarabic{فِعَالَة} (fiʕaːla) \\
\textbf{Grammar} & Preposition + definite noun in genitive case, feminine singular \\
\midrule \\
\textbf{Table of Conjugations} & \makecell[l]{
Nominative: \textarabic{خِزَانَةٌ} [xizaːnatun] \\
Genitive: \textarabic{خِزَانَةِ} [xizaːnati] \\
Accusative: \textarabic{خِزَانَةً} [xizaːnatan]
} \\
\midrule
\textbf{Examples} & \makecell[l]{\parbox{9.5cm}{
1. \textarabic{فَتَحَ خِزَانَةَ المَلَابِسِ} - He opened the clothes closet [fataħa xizaːnata l-malaːbisi]\\
2. \textarabic{خِزَانَةُ الكُتُبِ مُمْتَلِئَةٌ} - The book cabinet is full [xizaːnatu l-kutubi mumtaliʔa]\\
3. \textarabic{اِشْتَرَى خِزَانَةً جَدِيدَةً} - He bought a new closet [ʔiʃtaraː xizaːnatan dʒadiːda]
}} \\
\midrule \\
\textbf{Synonyms} & \textarabic{دُولَاب} (wardrobe), \textarabic{صُنْدُوق} (chest), \textarabic{خَزِينَة} (treasury/storage) \\
\textbf{Etymology} & From root x-z-n meaning "to store/treasure" \\
\bottomrule
\end{tabular}

\subsection{\textarabic{غُرْفَةِ البُومِ} – [ɣurfati l-buːmi]}
\begin{tabular}{p{3cm}p{10cm}}
\toprule
\textbf{Translation} & the owl's room / room of the owl \\
\textbf{Root} & \textarabic{غ-ر-ف} (ɣ-r-f) and \textarabic{ب-و-م} (b-w-m) \\
\textbf{Pattern} & \textarabic{فُعْلَة} (fuʕla) + \textarabic{فُعْل} (fuʕl) \\
\textbf{Grammar} & Definite noun in genitive + definite noun in genitive (possessive construct) \\
\midrule \\
\textbf{Table of Conjugations} & \makecell[l]{
Room (nom.): \textarabic{غُرْفَةٌ} [ɣurfatun] \\
Room (gen.): \textarabic{غُرْفَةِ} [ɣurfati] \\
Owl: \textarabic{البُومُ} [al-buːmu]
} \\
\midrule
\textbf{Examples} & \makecell[l]{\parbox{9.5cm}{
1. \textarabic{غُرْفَةُ النَّوْمِ وَاسِعَةٌ} - The bedroom is spacious [ɣurfatu n-nawmi waːsiʕa]\\
2. \textarabic{البُومُ يَطِيرُ لَيْلًا} - The owl flies at night [al-buːmu jat̪iːru lajlan]\\
3. \textarabic{سَمِعَ صَوْتَ البُومِ} - He heard the owl's sound [samiʕa s̱awta l-buːmi]
}} \\
\midrule \\
\textbf{Synonyms} & \textarabic{حُجْرَة} (room), \textarabic{قَاعَة} (hall); \textarabic{بُومَة} (female owl) \\
\textbf{Etymology} & ɣ-r-f from "scooping/taking"; b-w-m onomatopoetic for owl's hoot \\
\bottomrule
\end{tabular}

\subsection{\textarabic{وَاخْتَبَأَ فِيهَا} – [waxtabaʔa fiːhaː]}
\begin{tabular}{p{3cm}p{10cm}}
\toprule
\textbf{Translation} & and hid in it \\
\textbf{Root} & \textarabic{خ-ب-أ} (x-b-ʔ) \\
\textbf{Pattern} & \textarabic{اِفْتَعَلَ} (ʔiftaʕala - Form VIII) \\
\textbf{Grammar} & Past tense verb Form VIII, 3rd person masculine singular + preposition + pronoun \\
\midrule \\
\textbf{Table of Conjugations} & \makecell[l]{
Infinitive: \textarabic{اِخْتِبَاء} [ʔixtibaːʔ] \\
Present: \textarabic{يَخْتَبِئُ} [jaxtabiʔu] \\
Past: \textarabic{اِخْتَبَأَ} [ʔixtabaʔa]
} \\
\midrule
\textbf{Examples} & \makecell[l]{\parbox{9.5cm}{
1. \textarabic{اِخْتَبَأَ الطِّفْلُ تَحْتَ السَّرِيرِ} - The child hid under the bed [ʔixtabaʔa t̪-t̪iflu taħta s-sariːri]\\
2. \textarabic{يَخْتَبِئُ اللُّصُوصُ في الظَّلَامِ} - Thieves hide in darkness [jaxtabiʔu l-lus̱uːs̱u fiː ð-ð̩alaːmi]\\
3. \textarabic{سَيَخْتَبِئُ هُنَاكَ} - He will hide there [sajaxtabiʔu hunaːka]
}} \\
\midrule \\
\textbf{Synonyms} & \textarabic{اِسْتَتَرَ} (concealed himself), \textarabic{تَوَارَى} (disappeared), \textarabic{كَمَنَ} (lay in ambush) \\
\textbf{Etymology} & From root x-b-ʔ meaning "to be hidden/concealed" \\
\bottomrule
\end{tabular}

% ======================== Phrase Analysis ========================
\section{Phrase Analysis}
\begin{tcolorbox}[colback=boxcolor,colframe=headercolor,breakable]
\textbf{Grammatical Structure:}\\
Past verb + proper noun subject + preposition + indefinite noun (genitive) + prepositional phrase + adverbial accusative + conjunction + past verb + preposition + definite noun construct (genitive) + conjunction + Form VIII past verb + prepositional phrase with pronoun \\

\textbf{Key Grammar Points:}
\begin{itemize}
\item \textarabic{شَعَرَ بِـ} requires the preposition \textarabic{بِ} to mean "felt/sensed"
\item \textarabic{لَيْلًا} is an adverbial accusative indicating time
\item \textarabic{فَقَامَ} uses the conjunction \textarabic{فَ} to show consequence/sequence
\item \textarabic{غُرْفَةِ البُومِ} is an إضافة (possessive construct) - "room of the owl"
\item \textarabic{اخْتَبَأَ} is Form VIII indicating reflexive action (hid himself)
\item \textarabic{فِيهَا} refers back to the closet (feminine pronoun agreement)
\item The narrative follows chronological sequence: sensation → reaction → movement → hiding
\item This sets up the classic Juha scenario where his unconventional response will follow
\end{itemize}
\end{tcolorbox}

% ======================== Similar Phrases for Practice ========================
\section{Similar Phrases for Practice}

\begin{enumerate}
\item \textarabic{سَمِعَ عَلِيٌّ صَوْتًا في بَيْتِهِ لَيْلًا، فَذَهَبَ إِلَى المَطْبَخِ وَاخْتَبَأَ خَلْفَ البَابِ}\\
Ali heard a sound in his house at night, so he went to the kitchen and hid behind the door [samiʕa ʕalijjun s̱awtan fiː bajtihi lajlan, faðahaba ʔila l-mat̪baxi waxtabaʔa xalfa l-baːbi]

\item \textarabic{أَحَسَّتْ فَاطِمَةُ بِحَرَكَةٍ في الحَدِيقَةِ صَبَاحًا، فَمَشَتْ إِلَى النَّافِذَةِ وَنَظَرَتْ مِنْهَا}\\
Fatima sensed movement in the garden in the morning, so she walked to the window and looked through it [ʔaħassat faːt̪imatun biħarakatin fiː l-ħadiːqati s̱abaːħan, famaʃat ʔila n-naːfiðati wanaðarat minhaː]

\item \textarabic{شَعَرَ الحَارِسُ بِخَطَرٍ في المَتْحَفِ مَسَاءً، فَرَكَضَ إِلَى غُرْفَةِ المُرَاقَبَةِ وَاتَّصَلَ بِالشُّرْطَةِ}\\
The guard felt danger in the museum in the evening, so he ran to the surveillance room and called the police [ʃaʕara l-ħaːrisu bixat̪arin fiː l-matħafi masaːʔan, farakaða ʔila ɣurfati l-muraːqabati watsas̱ala biʃ-ʃurt̪ati]

\item \textarabic{لَمَسَ الوَلَدُ شَيْئًا غَرِيبًا في الصُّنْدُوقِ ظُهْرًا، فَفَتَحَ الغِطَاءَ وَوَجَدَ كَنْزًا قَدِيمًا}\\
The boy touched something strange in the box at noon, so he opened the lid and found an old treasure [lamasa l-waladu ʃajʔan ɣariːban fiː s̱-s̱unduːqi ð̩uhran, fafataħa l-ɣit̪aːʔa wawadʒada kanzan qadiːman]
\end{enumerate}

% ======================== Levantine Dialect Version ========================
\section{Levantine (Shaami) Arabic Dialect}

\begin{tcolorbox}[colback=white,colframe=dialectcolor,title=\textbf{Levantine Version},breakable]
\textarabic{حَسَّ جُحا بِحَرَامِيّ بِبَيْتُه بِاللَّيْل، فَقَامَ عَ خِزَانِة أُوضِة البُومِة وَخَبَّى حَالُه جُوَّاتَهَا}\\
\textbf{Phonetic:} [ħass dʒuħaː biħaraːmijj bibejtu bil-leel, faqaːm ʕa xizaːnet ʔoːɮet el-buːme wxabba ħaːlu dʒuwwaːtehaː]\\
\textbf{Translation:} Juha sensed a thief in his house at night, so he went to the closet of the owl's room and hid himself inside it.
\end{tcolorbox}
\textbf{Key Dialectal Changes:}
\begin{itemize}
\item \textarabic{شَعَرَ} → \textarabic{حَسَّ} (ħass) – dialectal verb for "sensed"
\item \textarabic{لِصّ} → \textarabic{حَرَامِيّ} (ħaraːmijj) – common dialectal word for "thief"
\item \textarabic{دَارِهِ} → \textarabic{بَيْتُه} (bejtu) – "house" in dialect form
\item \textarabic{لَيْلًا} → \textarabic{بِاللَّيْل} (bil-leel) – "at night" with definite article
\item \textarabic{إِلَى} → \textarabic{عَ} (ʕa) – shortened preposition "to"
\item \textarabic{غُرْفَة} → \textarabic{أُوضِة} (ʔoːɮa) – dialectal word for "room"
\item \textarabic{اخْتَبَأَ} → \textarabic{خَبَّى حَالُه} (xabba ħaːlu) – "hid himself" in dialect
\item \textarabic{فِيهَا} → \textarabic{جُوَّاتَهَا} (dʒuwwaːtehaː) – "inside it" in dialect
\end{itemize}

% ======================== Additional Learning Notes ========================
\section{Additional Learning Notes}

\begin{tcolorbox}[colback=boxcolor,colframe=accentcolor,title=\textbf{Cultural and Literary Context},breakable]
\textbf{Juha's Character:} This opening sets the stage for classic Juha humor. Instead of confronting the thief or calling for help, Juha chooses to hide – an unexpected response that will lead to an ironic twist.

\textbf{The Owl Room:} The mention of \textarabic{غُرْفَة البُوم} (owl room) adds folkloric atmosphere. Owls in Middle Eastern culture often symbolize wisdom or mystery, fitting for Juha's unconventional wisdom.

\textbf{Narrative Technique:} The sequence uses \textarabic{فَ} (fa-) to create rapid succession of events: sensing → standing → moving → hiding, building suspense for the comedic resolution.

\textbf{Social Commentary:} The story subtly comments on how the poor (Juha) respond to theft – not with anger or fear, but with an awareness of their own poverty that leads to empathy.
\end{tcolorbox}

\subsection{Memory Tips}
\begin{enumerate}
\item \textbf{Sensory Progression:} Remember \textit{Feel → Stand → Move → Hide} (\textarabic{شُعُور} → \textarabic{قِيَام} → \textarabic{حَرَكَة} → \textarabic{اخْتِبَاء})
\item \textbf{Preposition Pattern:} \textarabic{بِ} (with/by) → \textarabic{فِي} (in) → \textarabic{إِلَى} (to) → \textarabic{فِي} (in)
\item \textbf{Time Marker:} \textarabic{لَيْلًا} in accusative shows "when" something happens
\item \textbf{Conjunction Chain:} No conjunction → \textarabic{فَ} (so/then) → \textarabic{وَ} (and) creates narrative flow
\item \textbf{Form VIII Pattern:} \textarabic{اِخْتَبَأَ} follows \textarabic{اِفْتَعَلَ} pattern for reflexive actions
\end{enumerate}

\subsection{Related Vocabulary Family}
\begin{multicols}{2}

\textbf{Sensation family:}\\
\textarabic{شُعُور} – feeling [ʃuʕuːr]\\
\textarabic{إِحْسَاس} – sensation [ʔiħsaːs]\\
\textarabic{لَمْس} – touch [lams]\\
\textarabic{سَمْع} – hearing [samʕ]\\
\textarabic{رُؤْيَة} – sight [ruʔja]\\

\textbf{Movement family:}\\
\textarabic{قِيَام} – standing [qijaːm]\\
\textarabic{مَشْي} – walking [maʃj]\\
\textarabic{جَرْي} – running [dʒarj]\\
\textarabic{حَرَكَة} – movement [ħaraka]\\
\textarabic{انْتِقَال} – transition [ʔintiqaːl]\\

\textbf{House parts family:}\\
\textarabic{غُرْفَة} – room [ɣurfa]\\
\textarabic{مَطْبَخ} – kitchen [mat̪bax]\\
\textarabic{حَمَّام} – bathroom [ħammaːm]\\
\textarabic{صَالَة} – living room [s̱aːla]\\
\textarabic{شُرْفَة} – balcony [ʃurfa]\\

\textbf{Hiding family:}\\
\textarabic{إِخْفَاء} – concealment [ʔixfaːʔ]\\
\textarabic{اسْتِتَار} – hiding [ʔisitaːr]\\
\textarabic{تَوَارِي} – disappearance [tawaːriː]\\
\textarabic{كُمُون} – lurking [kumuːn]\\
\textarabic{احْتِجَاب} – veiling [ʔiħtidʒaːb]\\

\textbf{Time expressions:}\\
\textarabic{لَيْلًا} – at night [lajlan]\\
\textarabic{نَهَارًا} – during the day [nahaːran]\\
\textarabic{صَبَاحًا} – in the morning [s̱abaːħan]\\
\textarabic{مَسَاءً} – in the evening [masaːʔan]\\
\textarabic{ظُهْرًا} – at noon [ð̩uhran]\\

\textbf{Animals family:}\\
\textarabic{بُوم} – owl [buːm]\\
\textarabic{عُصْفُور} – sparrow [ʕus̱fuːr]\\
\textarabic{نَسْر} – eagle [nasr]\\
\textarabic{غُرَاب} – crow [ɣuraːb]\\
\textarabic{صَقْر} – falcon [s̱aqr]
\end{multicols}

\subsection{Grammar Focus Points}

\begin{tcolorbox}[colback=white,colframe=headercolor,title=\textbf{Key Grammatical Constructions}]
\textbf{1. Sensation Verbs with Prepositions:}
\begin{itemize}
\item \textarabic{شَعَرَ بِـ} (felt/sensed with)
\item \textarabic{أَحَسَّ بِـ} (felt with)
\item \textarabic{سَمِعَ ـــ} (heard - direct object)
\item \textarabic{رَأَى ـــ} (saw - direct object)
\end{itemize}

\textbf{2. Adverbial Accusative for Time:}
\begin{itemize}
\item \textarabic{لَيْلًا} (at night) - indefinite accusative
\item \textarabic{صَبَاحًا} (in the morning) - indefinite accusative
\item \textarabic{في اللَّيْلِ} (at night) - prepositional phrase alternative
\end{itemize}

\textbf{3. Possessive Constructs (إضافة):}
\begin{itemize}
\item \textarabic{دَارِهِ} (his house) - noun + possessive pronoun
\item \textarabic{غُرْفَةِ البُومِ} (owl's room) - first noun definite by association
\item Both nouns in genitive case in construct
\end{itemize}

\textbf{4. Form VIII Verbs:}
\begin{itemize}
\item Pattern: \textarabic{اِفْتَعَلَ} often indicates reflexive/intensive action
\item \textarabic{اِخْتَبَأَ} (he hid himself) - reflexive hiding
\item \textarabic{اِجْتَمَعَ} (they gathered) - mutual action
\end{itemize}
\end{tcolorbox}

\end{document}