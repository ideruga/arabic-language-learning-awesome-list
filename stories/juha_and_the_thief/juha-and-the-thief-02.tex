\documentclass[letter,12pt]{article}

\usepackage{fancyhdr}
\usepackage[most]{tcolorbox}
\usepackage{longtable}
\usepackage{makecell}
\usepackage[utf8]{inputenc}
\usepackage{tikz}
\usepackage{geometry}
\usepackage{array}
\usepackage{multicol}
\usepackage[table]{xcolor}
\usepackage{polyglossia}
\setmainlanguage{english}
\setotherlanguage{arabic}
\setotherlanguage{hebrew}
\usepackage{fontspec}
\setmainfont{Charis SIL}
\newfontfamily\arabicfont[Script=Arabic,Scale=1.2]{Amiri}
\newfontfamily\hebrewfont{Ezra SIL}[Script=Hebrew]
\geometry{margin=1.5cm}
\usepackage{booktabs}
\usepackage{graphicx}
\usepackage{multirow}

\pagestyle{fancy}
\fancyhf{}
\fancyhead[R]{\textit{Juha and the Thief: the Thief Searches but Finds Juha}} % Right header with document title

\usetikzlibrary{shapes,arrows,decorations.pathmorphing}

% Custom colors
\definecolor{headercolor}{RGB}{70,130,180}
\definecolor{boxcolor}{RGB}{240,248,255}
\definecolor{accentcolor}{RGB}{220,20,60}
\definecolor{tableheader}{RGB}{220,220,220}
\definecolor{dialectcolor}{RGB}{34,139,34}

\begin{document}

\title{\textbf{\Large Arabic Phrase Analyser}\\
\large The Thief Searches but Finds Juha\\
\normalsize \textarabic{وَبَحَثَ اللَّصُّ عَنْ شَيْءٍ يَسْرِقُهُ فَلَمْ يَجِدْ، فَرَأَى الْخِزَانَةَ فَفَتَحَهَا وَإِذَا بِجُحَا فِيهَا}}
\author{Igor Deruga}
\date{}
\maketitle

% ======================== Phrase Display ========================
\begin{tcolorbox}[colback=boxcolor,colframe=headercolor,title=\textbf{Arabic Phrase},breakable]
\centering
\textarabic{وبحث اللص عن شيء يسرقه فلم يجد، فرأى الخزانة ففتحها وإذا بجحا فيها}
\\[0.5em]
\textbf{Without Diacritics}
\\[1em]
\textarabic{وَبَحَثَ اللَّصُّ عَنْ شَيْءٍ يَسْرِقُهُ فَلَمْ يَجِدْ، فَرَأَى الْخِزَانَةَ فَفَتَحَهَا وَإِذَا بِجُحَا فِيهَا}
\\[0.5em]
\textbf{With Full Diacritics}
\end{tcolorbox}

% ======================== Translations ========================
\section{English Translation}
\begin{tcolorbox}[colback=white,colframe=accentcolor,breakable]
\textbf{Literal:} And searched the thief for something he-steals-it so-not he-found, so-he-saw the-wardrobe so-he-opened-it and behold with-Juha in-it \\
\textit{[Arabic order retained for direct mapping]}\\[0.5em]
\textbf{Adapted:} The thief searched for something to steal but found nothing, then he saw the wardrobe, opened it, and behold, Juha was inside.
\end{tcolorbox}

% ======================== Detailed Word Analysis ========================
\section{Detailed Word Analysis}

\subsection{\textarabic{وَبَحَثَ} — [wabaħaθa]}
\begin{tabular}{p{3cm}p{10cm}}
\toprule

\textbf{Translation} & and he searched \\
\textbf{Root} & \textarabic{ب-ح-ث} (b-ħ-θ) \\
\textbf{Pattern} & \textarabic{فَعَلَ} (faʕala) \\
\textbf{Grammar} & Past tense verb, 3rd person masculine singular, with conjunction \textarabic{وَ} (wa-) prefix \\
\midrule \\
\textbf{Examples} & \makecell[l]{\parbox{9.5cm}{
1. \textarabic{بَحَثَ الطَّالِبُ عَنِ الْكِتَابِ} - The student searched for the book [baħaθa ṭ-ṭālibu ʕan al-kitābi]\\
2. \textarabic{يَبْحَثُونَ عَنِ الْحَقِيقَةِ} - They search for the truth [yabħaθūna ʕan al-ħaqīqati]\\
3. \textarabic{ابْحَثْ عَنْ عَمَلٍ جَدِيدٍ} - Search for new work [ibħaθ ʕan ʕamalin jadīdin]
} } \\
\midrule \\
\textbf{Synonyms} & \textarabic{فَتَّشَ} (searched), \textarabic{نَقَّبَ} (investigated), \textarabic{تَقَصَّى} (explored) \\
\textbf{Etymology} & From Proto-Semitic *baħaθ-, related to Hebrew \texthebrew{בחש} (baħash) "to search" \\
\bottomrule
\end{tabular}

\subsection{Conjugation}
\begin{longtable}{|>{\raggedright}p{3.5cm}|p{5cm}|p{5cm}|}
\hline
\textbf{Person} & \textbf{Perfect (Past)} & \textbf{Imperfect (Present)} \\
\hline
\textbf{3rd person masculine singular} & \textarabic{بَحَثَ} [baħaθa] & \textarabic{يَبْحَثُ} [yabħaθu] \\
\hline
\textbf{3rd person feminine singular} & \textarabic{بَحَثَتْ} [baħaθat] & \textarabic{تَبْحَثُ} [tabħaθu] \\
\hline
\textbf{3rd person masculine dual} & \textarabic{بَحَثَا} [baħaθā] & \textarabic{يَبْحَثَانِ} [yabħaθāni] \\
\hline
\textbf{3rd person feminine dual} & \textarabic{بَحَثَتَا} [baħaθatā] & \textarabic{تَبْحَثَانِ} [tabħaθāni] \\
\hline
\textbf{3rd person masculine plural} & \textarabic{بَحَثُوا} [baħaθū] & \textarabic{يَبْحَثُونَ} [yabħaθūna] \\
\hline
\textbf{3rd person feminine plural} & \textarabic{بَحَثْنَ} [baħaθna] & \textarabic{يَبْحَثْنَ} [yabħaθna] \\
\hline
\textbf{2nd person masculine singular} & \textarabic{بَحَثْتَ} [baħaθta] & \textarabic{تَبْحَثُ} [tabħaθu] \\
\hline
\textbf{2nd person feminine singular} & \textarabic{بَحَثْتِ} [baħaθti] & \textarabic{تَبْحَثِينَ} [tabħaθīna] \\
\hline
\textbf{2nd person dual (m./f.)} & \textarabic{بَحَثْتُمَا} [baħaθtumā] & \textarabic{تَبْحَثَانِ} [tabħaθāni] \\
\hline
\textbf{2nd person masculine plural} & \textarabic{بَحَثْتُمْ} [baħaθtum] & \textarabic{تَبْحَثُونَ} [tabħaθūna] \\
\hline
\textbf{2nd person feminine plural} & \textarabic{بَحَثْتُنَّ} [baħaθtunna] & \textarabic{تَبْحَثْنَ} [tabħaθna] \\
\hline
\textbf{1st person singular} & \textarabic{بَحَثْتُ} [baħaθtu] & \textarabic{أَبْحَثُ} [abħaθu] \\
\hline
\textbf{1st person plural} & \textarabic{بَحَثْنَا} [baħaθnā] & \textarabic{نَبْحَثُ} [nabħaθu] \\
\hline
\end{longtable}

\subsubsection*{Conjugation Notes}
\begin{itemize}
  \item The \textbf{future} is formed with the prefix \textarabic{سَ} [sa-] or \textarabic{سَوْفَ} [sawfa] before the imperfect (e.g., \textarabic{سَيَبْحَثُ} "he will search").
  \item The \textbf{moods of the imperfect}: 
    \begin{itemize}
      \item Indicative: \textarabic{يَبْحَثُ} [yabħaθu] 
      \item Subjunctive: \textarabic{لَنْ يَبْحَثَ} [lan yabħaθa]
      \item Jussive: \textarabic{لَمْ يَبْحَثْ} [lam yabħaθ]
      \item Imperative: \textarabic{ابْحَثْ} [ibħaθ!]
    \end{itemize}
  \item \textbf{Passive voice}: 
    \begin{itemize}
      \item Perfect: \textarabic{بُحِثَ} [buħiθa] — it was searched
      \item Imperfect: \textarabic{يُبْحَثُ} [yubħaθu] — it is searched
    \end{itemize}
\end{itemize}

\subsection{\textarabic{اللَّصُّ} — [alliṣṣu]}

\begin{tabular}{p{3cm}p{10cm}}
\toprule
\textbf{Translation} & the thief \\
\textbf{Root} & \textarabic{ل-ص-ص} (l-ṣ-ṣ) \\
\textbf{Pattern} & \textarabic{فَعْل} (faʕl) with definite article \\
\textbf{Grammar} & Noun, masculine, definite, nominative case (subject of verb) \\
\textbf{Table of Conjugations} & \makecell[l]{
Singular: \textarabic{لِصّ} [liṣṣ]\\
Plural: \textarabic{لُصُوص} [luṣūṣ]
} \\
\textbf{Examples} & \makecell[l]{\parbox{9.5cm}{
1. \textarabic{اللِّصُّ سَرَقَ السَّيَّارَةَ} - The thief stole the car [alliṣṣu saraqa s-sayyārata]\\
2. \textarabic{رَأَيْتُ اللِّصَّ يَجْرِي} - I saw the thief running [raʔaytu l-liṣṣa yajrī]\\
3. \textarabic{اللُّصُوصُ خَطَرُونَ} - The thieves are dangerous [al-luṣūṣu xaṭarūna]
}} \\
\midrule \\
\textbf{Synonyms} & \textarabic{سَارِق} (thief), \textarabic{حَرَامِي} (burglar), \textarabic{خَاطِف} (snatcher) \\
\textbf{Etymology} & From root meaning "to stick, adhere" - one who "sticks" to others' property \\
\bottomrule
\end{tabular}

\subsubsection*{Full Declension Matrix}

\begin{tabular}{|c|c|c|c|c|}
\hline
\textbf{Number} & \textbf{Case} & \textbf{Indefinite} & \textbf{Definite} & \textbf{With Suffix (1st sg.)} \\
\hline
\multirow{3}{*}{Singular (\textarabic{مُفْرَد})} 
 & Nominative   & \textarabic{لِصٌّ} (liṣṣun) & \textarabic{اللِّصُّ} (alliṣṣu) & \textarabic{لِصِّي} (liṣṣī) \\
 & Accusative   & \textarabic{لِصّاً} (liṣṣan) & \textarabic{اللِّصَّ} (alliṣṣa) & \textarabic{لِصِّيَّ} (liṣṣiyya) \\
 & Genitive     & \textarabic{لِصٍّ} (liṣṣin) & \textarabic{اللِّصِّ} (alliṣṣi) & \textarabic{لِصِّي} (liṣṣī) \\
\hline
\multirow{3}{*}{Dual (\textarabic{مُثَنَّى})} 
 & Nominative   & \textarabic{لِصَّانِ} (liṣṣāni) & \textarabic{اللِّصَّانِ} (alliṣṣāni) & \textarabic{لِصَّايَ} (liṣṣāya) \\
 & Acc/Gen      & \textarabic{لِصَّيْنِ} (liṣṣayni) & \textarabic{اللِّصَّيْنِ} (alliṣṣayni) & \textarabic{لِصَّيَّ} (liṣṣayya) \\
\hline
\multirow{3}{*}{Plural (\textarabic{جَمْع})} 
  & Nominative    & \textarabic{لُصُوصٌ} (luṣūṣun) & \textarabic{اللُّصُوصُ} (alluṣūṣu) & \textarabic{لُصُوصِي} (luṣūṣī) \\
  & Accusative      & \textarabic{لُصُوصاً} (luṣūṣan) & \textarabic{اللُّصُوصَ} (alluṣūṣa) & \textarabic{لُصُوصِيَّ} (luṣūṣiyya) \\
  & Genitive          & \textarabic{لُصُوصٍ} (luṣūṣin) & \textarabic{اللُّصُوصِ} (alluṣūṣi) & \textarabic{لُصُوصِي} (luṣūṣī) \\
\hline
\end{tabular}

\subsection{\textarabic{عَنْ} — [ʕan]}
\begin{tabular}{p{3cm}p{10cm}}
\toprule

\textbf{Translation} & about, for, from \\
\textbf{Root} & Particle (not derived from root) \\
\textbf{Pattern} & Preposition \\
\textbf{Grammar} & Preposition governing genitive case \\
\midrule \\
\textbf{Examples} & \makecell[l]{\parbox{9.5cm}{
1. \textarabic{سَأَلَ عَنِ الطَّرِيقِ} - He asked about the road [saʔala ʕan iṭ-ṭarīqi]\\
2. \textarabic{ابْتَعَدَ عَنِ الْخَطَرِ} - He moved away from danger [ibtaʕada ʕan al-xaṭari]\\
3. \textarabic{بَحَثَ عَنِ الْحَلِّ} - He searched for the solution [baħaθa ʕan al-ħalli]
} } \\
\midrule \\
\textbf{Synonyms} & \textarabic{مِن} (from), \textarabic{بِشَأْنِ} (regarding), \textarabic{حَوْلَ} (around/about) \\
\textbf{Etymology} & From Proto-Semitic *ʕan-, related to Hebrew \texthebrew{על} (ʕal) "on, about" \\
\bottomrule
\end{tabular}

\subsection{\textarabic{شَيْءٍ} — [šayʔin]}
\begin{tabular}{p{3cm}p{10cm}}
\toprule

\textbf{Translation} & something, thing \\
\textbf{Root} & \textarabic{ش-ي-ء} (š-y-ʔ) \\
\textbf{Pattern} & \textarabic{فَعْل} (faʕl) \\
\textbf{Grammar} & Indefinite noun, masculine, genitive case (due to preposition \textarabic{عَنْ}) \\
\textbf{Table of Conjugations} & \makecell[l]{
Singular: \textarabic{شَيْء} [šayʔ]\\
Plural: \textarabic{أَشْيَاء} [ašyāʔ]
} \\
\textbf{Examples} & \makecell[l]{\parbox{9.5cm}{
1. \textarabic{أُرِيدُ شَيْئاً} - I want something [urīdu šayʔan]\\
2. \textarabic{لَا شَيْءَ هُنَا} - Nothing here [lā šayʔa hunā]\\
3. \textarabic{كُلُّ شَيْءٍ جَمِيلٌ} - Everything is beautiful [kullu šayʔin jamīlun]
}} \\
\midrule \\
\textbf{Synonyms} & \textarabic{أَمْر} (matter), \textarabic{كَائِن} (entity), \textarabic{مَوْضُوع} (object) \\
\textbf{Etymology} & From Proto-Semitic *šayʔ-, meaning "thing, entity" \\
\bottomrule
\end{tabular}

\subsection{\textarabic{يَسْرِقُهُ} — [yasriquhu]}
\begin{tabular}{p{3cm}p{10cm}}
\toprule

\textbf{Translation} & he steals it \\
\textbf{Root} & \textarabic{س-ر-ق} (s-r-q) \\
\textbf{Pattern} & Form I, \textarabic{يَفْعِلُ} (yafʕilu) \\
\textbf{Grammar} & Present tense verb, 3rd person masculine singular + attached object pronoun \textarabic{هُ} \\
\textbf{Table of Conjugations} & \makecell[l]{
Perfect: \textarabic{سَرَقَ} [saraqa]\\
Imperfect: \textarabic{يَسْرِقُ} [yasriqu]\\
Active Participle: \textarabic{سَارِق} [sāriq]
} \\
\textbf{Examples} & \makecell[l]{\parbox{9.5cm}{
1. \textarabic{سَرَقَ المِحْفَظَةَ} - He stole the wallet [saraqa l-miħfaẓata]\\
2. \textarabic{اللُّصُوصُ يَسْرِقُونَ} - The thieves steal [al-luṣūṣu yasriqūna]\\
3. \textarabic{لَا تَسْرِقْ مَالَ النَّاسِ} - Don't steal people's money [lā tasriq māla n-nāsi]
}} \\
\midrule \\
\textbf{Synonyms} & \textarabic{نَهَبَ} (plundered), \textarabic{اخْتَلَسَ} (embezzled), \textarabic{غَصَبَ} (usurped) \\
\textbf{Etymology} & From Proto-Semitic *saraq-, related to Hebrew \texthebrew{שדד} (šādad) "to rob" \\
\bottomrule
\end{tabular}

\subsection{Conjugation}
\begin{longtable}{|>{\raggedright}p{3.5cm}|p{5cm}|p{5cm}|}
\hline
\textbf{Person} & \textbf{Perfect (Past)} & \textbf{Imperfect (Present)} \\
\hline
\textbf{3rd person masculine singular} & \textarabic{سَرَقَ} [saraqa] & \textarabic{يَسْرِقُ} [yasriqu] \\
\hline
\textbf{3rd person feminine singular} & \textarabic{سَرَقَتْ} [saraqat] & \textarabic{تَسْرِقُ} [tasriqu] \\
\hline
\textbf{3rd person masculine dual} & \textarabic{سَرَقَا} [saraqā] & \textarabic{يَسْرِقَانِ} [yasriqāni] \\
\hline
\textbf{3rd person feminine dual} & \textarabic{سَرَقَتَا} [saraqatā] & \textarabic{تَسْرِقَانِ} [tasriqāni] \\
\hline
\textbf{3rd person masculine plural} & \textarabic{سَرَقُوا} [saraqū] & \textarabic{يَسْرِقُونَ} [yasriqūna] \\
\hline
\textbf{3rd person feminine plural} & \textarabic{سَرَقْنَ} [saraqna] & \textarabic{يَسْرِقْنَ} [yasriqna] \\
\hline
\textbf{2nd person masculine singular} & \textarabic{سَرَقْتَ} [saraqta] & \textarabic{تَسْرِقُ} [tasriqu] \\
\hline
\textbf{2nd person feminine singular} & \textarabic{سَرَقْتِ} [saraqti] & \textarabic{تَسْرِقِينَ} [tasriqīna] \\
\hline
\textbf{2nd person dual (m./f.)} & \textarabic{سَرَقْتُمَا} [saraqtumā] & \textarabic{تَسْرِقَانِ} [tasriqāni] \\
\hline
\textbf{2nd person masculine plural} & \textarabic{سَرَقْتُمْ} [saraqtum] & \textarabic{تَسْرِقُونَ} [tasriqūna] \\
\hline
\textbf{2nd person feminine plural} & \textarabic{سَرَقْتُنَّ} [saraqtunna] & \textarabic{تَسْرِقْنَ} [tasriqna] \\
\hline
\textbf{1st person singular} & \textarabic{سَرَقْتُ} [saraqtu] & \textarabic{أَسْرِقُ} [asriqu] \\
\hline
\textbf{1st person plural} & \textarabic{سَرَقْنَا} [saraqnā] & \textarabic{نَسْرِقُ} [nasriqu] \\
\hline
\end{longtable}

\subsubsection*{Conjugation Notes}
\begin{itemize}
  \item The \textbf{future} is formed with the prefix \textarabic{سَ} [sa-] or \textarabic{سَوْفَ} [sawfa] before the imperfect (e.g., \textarabic{سَيَسْرِقُ} "he will steal").
  \item The \textbf{moods of the imperfect}: 
    \begin{itemize}
      \item Indicative: \textarabic{يَسْرِقُ} [yasriqu] 
      \item Subjunctive: \textarabic{لَنْ يَسْرِقَ} [lan yasriqa]
      \item Jussive: \textarabic{لَمْ يَسْرِقْ} [lam yasriq]
      \item Imperative: \textarabic{اسْرِقْ} [isriq!] (not commonly used!)
    \end{itemize}
  \item \textbf{Passive voice}: 
    \begin{itemize}
      \item Perfect: \textarabic{سُرِقَ} [suriqa] — it was stolen
      \item Imperfect: \textarabic{يُسْرَقُ} [yusraqu] — it is being stolen
    \end{itemize}
\end{itemize}

\subsection{\textarabic{فَلَمْ يَجِدْ} — [fa-lam yajid]}
\begin{tabular}{p{3cm}p{10cm}}
\toprule

\textbf{Translation} & so he did not find \\
\textbf{Root} & \textarabic{و-ج-د} (w-j-d) \\
\textbf{Pattern} & \textarabic{فَ} + \textarabic{لَمْ} + jussive mood \\
\textbf{Grammar} & Conjunction + negation particle + past verb in jussive mood \\
\textbf{Table of Conjugations} & \makecell[l]{
Perfect: \textarabic{وَجَدَ} [wajada]\\
Imperfect: \textarabic{يَجِدُ} [yajidu]\\
Jussive: \textarabic{يَجِدْ} [yajid]
} \\
\textbf{Examples} & \makecell[l]{\parbox{9.5cm}{
1. \textarabic{لَمْ يَجِدِ الْمِفْتَاحَ} - He did not find the key [lam yajid al-miftāħa]\\
2. \textarabic{وَجَدْتُ الْكِتَابَ} - I found the book [wajadtu l-kitāba]\\
3. \textarabic{سَيَجِدُ الْحَلَّ} - He will find the solution [sayajidu l-ħalla]
}} \\
\midrule \\
\textbf{Synonyms} & \textarabic{عَثَرَ عَلَى} (came across), \textarabic{اكْتَشَفَ} (discovered), \textarabic{لَقِيَ} (met/found) \\
\textbf{Etymology} & From Proto-Semitic *wajad-, meaning "to find, encounter" \\
\bottomrule
\end{tabular}

\subsection{Conjugation}
\begin{longtable}{|>{\raggedright}p{3.5cm}|p{5cm}|p{5cm}|}
\hline
\textbf{Person} & \textbf{Perfect (Past)} & \textbf{Imperfect (Present)} \\
\hline
\textbf{3rd person masculine singular} & \textarabic{وَجَدَ} [wajada] & \textarabic{يَجِدُ} [yajidu] \\
\hline
\textbf{3rd person feminine singular} & \textarabic{وَجَدَتْ} [wajadat] & \textarabic{تَجِدُ} [tajidu] \\
\hline
\textbf{3rd person masculine dual} & \textarabic{وَجَدَا} [wajadā] & \textarabic{يَجِدَانِ} [yajidāni] \\
\hline
\textbf{3rd person feminine dual} & \textarabic{وَجَدَتَا} [wajadatā] & \textarabic{تَجِدَانِ} [tajidāni] \\
\hline
\textbf{3rd person masculine plural} & \textarabic{وَجَدُوا} [wajadū] & \textarabic{يَجِدُونَ} [yajidūna] \\
\hline
\textbf{3rd person feminine plural} & \textarabic{وَجَدْنَ} [wajadna] & \textarabic{يَجِدْنَ} [yajidna] \\
\hline
\textbf{2nd person masculine singular} & \textarabic{وَجَدْتَ} [wajadta] & \textarabic{تَجِدُ} [tajidu] \\
\hline
\textbf{2nd person feminine singular} & \textarabic{وَجَدْتِ} [wajadti] & \textarabic{تَجِدِينَ} [tajidīna] \\
\hline
\textbf{2nd person dual (m./f.)} & \textarabic{وَجَدْتُمَا} [wajadtumā] & \textarabic{تَجِدَانِ} [tajidāni] \\
\hline
\textbf{2nd person masculine plural} & \textarabic{وَجَدْتُمْ} [wajadtum] & \textarabic{تَجِدُونَ} [tajidūna] \\
\hline
\textbf{2nd person feminine plural} & \textarabic{وَجَدْتُنَّ} [wajadtunna] & \textarabic{تَجِدْنَ} [tajidna] \\
\hline
\textbf{1st person singular} & \textarabic{وَجَدْتُ} [wajadtu] & \textarabic{أَجِدُ} [ajidu] \\
\hline
\textbf{1st person plural} & \textarabic{وَجَدْنَا} [wajadnā] & \textarabic{نَجِدُ} [najidu] \\
\hline
\end{longtable}

\subsubsection*{Conjugation Notes}
\begin{itemize}
  \item The \textbf{future} is formed with the prefix \textarabic{سَ} [sa-] or \textarabic{سَوْفَ} [sawfa] before the imperfect (e.g., \textarabic{سَيَجِدُ} "he will find").
  \item The \textbf{moods of the imperfect}: 
    \begin{itemize}
      \item Indicative: \textarabic{يَجِدُ} [yajidu] 
      \item Subjunctive: \textarabic{لَنْ يَجِدَ} [lan yajida]
      \item Jussive: \textarabic{لَمْ يَجِدْ} [lam yajid]
      \item Imperative: \textarabic{جِدْ} [jid!]
    \end{itemize}
  \item \textbf{Passive voice}: 
    \begin{itemize}
      \item Perfect: \textarabic{وُجِدَ} [wujida] — it was found
      \item Imperfect: \textarabic{يُوجَدُ} [yūjadu] — it is found
    \end{itemize}
\end{itemize}

\subsection{\textarabic{فَرَأَى} — [fa-raʔā]}

\begin{tabular}{p{3cm}p{10cm}}
\toprule
\textbf{Translation} & then he saw \\
\textbf{Root} & \textarabic{ر-أ-ى} (r-ʔ-y) \\
\textbf{Pattern} & \textarabic{فَعَلَ} (faʕala) with conjunction \\
\textbf{Grammar} & Past tense verb, 3rd person masculine singular, with conjunction \textarabic{فَ} (fa-) prefix \\
\textbf{Table of Conjugations} & \makecell[l]{
Perfect: \textarabic{رَأَى} [raʔā]\\
Imperfect: \textarabic{يَرَى} [yarā]\\
Active Participle: \textarabic{رَاءٍ} [rāʔin]
} \\
\textbf{Examples} & \makecell[l]{\parbox{9.5cm}{
1. \textarabic{رَأَى الطَّائِرَ} - He saw the bird [raʔā ṭ-ṭāʔira]\\
2. \textarabic{تَرَى النُّجُومَ} - She sees the stars [tarā n-nujūma]\\
3. \textarabic{رَأَيْتُ حُلْماً جَمِيلاً} - I saw a beautiful dream [raʔaytu ħulman jamīlan]
}} \\
\midrule \\
\textbf{Synonyms} & \textarabic{شَاهَدَ} (witnessed), \textarabic{أَبْصَرَ} (perceived), \textarabic{نَظَرَ} (looked) \\
\textbf{Etymology} & From Proto-Semitic *raʔay-, related to Hebrew \texthebrew{ראה} (rāʔāh) "to see" \\
\bottomrule
\end{tabular}

\subsubsection*{Full Declension Matrix}

\begin{tabular}{|c|c|c|c|c|}
\hline
\textbf{Number} & \textbf{Case} & \textbf{Indefinite} & \textbf{Definite} & \textbf{With Suffix (1st sg.)} \\
\hline
\multirow{3}{*}{Singular (\textarabic{مُفْرَد})} 
 & Nominative   & \textarabic{رُؤْيَةٌ} (ruʔyatun) & \textarabic{الرُّؤْيَةُ} (ar-ruʔyatu) & \textarabic{رُؤْيَتِي} (ruʔyatī) \\
 & Accusative   & \textarabic{رُؤْيَةً} (ruʔyatan) & \textarabic{الرُّؤْيَةَ} (ar-ruʔyata) & \textarabic{رُؤْيَتِيَّ} (ruʔyatiyya) \\
 & Genitive     & \textarabic{رُؤْيَةٍ} (ruʔyatin) & \textarabic{الرُّؤْيَةِ} (ar-ruʔyati) & \textarabic{رُؤْيَتِي} (ruʔyatī) \\
\hline
\multirow{3}{*}{Dual (\textarabic{مُثَنَّى})} 
 & Nominative   & \textarabic{رُؤْيَتَانِ} (ruʔyatāni) & \textarabic{الرُّؤْيَتَانِ} (ar-ruʔyatāni) & \textarabic{رُؤْيَتَايَ} (ruʔyatāya) \\
 & Acc/Gen      & \textarabic{رُؤْيَتَيْنِ} (ruʔyatayni) & \textarabic{الرُّؤْيَتَيْنِ} (ar-ruʔyatayni) & \textarabic{رُؤْيَتَيَّ} (ruʔyatayya) \\
\hline
\multirow{3}{*}{Plural (\textarabic{جَمْع})} 
  & Nominative    & \textarabic{رُؤىً} (ruʔan) & \textarabic{الرُّؤَى} (ar-ruʔā) & \textarabic{رُؤَايَ} (ruʔāya) \\
  & Accusative      & \textarabic{رُؤىً} (ruʔan) & \textarabic{الرُّؤَى} (ar-ruʔā) & \textarabic{رُؤَايَ} (ruʔāya) \\
  & Genitive          & \textarabic{رُؤىً} (ruʔan) & \textarabic{الرُّؤَى} (ar-ruʔā) & \textarabic{رُؤَايَ} (ruʔāya) \\
\hline
\end{tabular}

\subsection{\textarabic{الْخِزَانَةَ} — [al-xizānata]}

\begin{tabular}{p{3cm}p{10cm}}
\toprule
\textbf{Translation} & the wardrobe \\
\textbf{Root} & \textarabic{خ-ز-ن} (x-z-n) \\
\textbf{Pattern} & \textarabic{فِعَالَة} (fiʕāla) \\
\textbf{Grammar} & Noun, feminine, definite, accusative case (direct object) \\
\textbf{Table of Conjugations} & \makecell[l]{
Singular: \textarabic{خِزَانَة} [xizāna]\\
Plural: \textarabic{خِزَانَات} [xizānāt]
} \\
\textbf{Examples} & \makecell[l]{\parbox{9.5cm}{
1. \textarabic{فَتَحَ خِزَانَةَ الْمَلَابِسِ} - He opened the clothes wardrobe [fataħa xizānata l-malābisi]\\
2. \textarabic{الْخِزَانَةُ مَلِيئَةٌ} - The wardrobe is full [al-xizānatu malīʔatun]\\
3. \textarabic{اشْتَرَى خِزَانَةً جَدِيدَةً} - He bought a new wardrobe [ištarā xizānatan jadīdatan]
}} \\
\midrule \\
\textbf{Synonyms} & \textarabic{دُولَاب} (closet), \textarabic{مِخْزَن} (storage), \textarabic{صُنْدُوق} (chest) \\
\textbf{Etymology} & From root meaning "to store, keep safe" \\
\bottomrule
\end{tabular}

\subsubsection*{Full Declension Matrix}

\begin{tabular}{|c|c|c|c|c|}
\hline
\textbf{Number} & \textbf{Case} & \textbf{Indefinite} & \textbf{Definite} & \textbf{With Suffix (1st sg.)} \\
\hline
\multirow{3}{*}{Singular (\textarabic{مُفْرَد})} 
 & Nominative   & \textarabic{خِزَانَةٌ} (xizānatun) & \textarabic{الْخِزَانَةُ} (al-xizānatu) & \textarabic{خِزَانَتِي} (xizānatī) \\
 & Accusative   & \textarabic{خِزَانَةً} (xizānatan) & \textarabic{الْخِزَانَةَ} (al-xizānata) & \textarabic{خِزَانَتِيَّ} (xizānatiyya) \\
 & Genitive     & \textarabic{خِزَانَةٍ} (xizānatin) & \textarabic{الْخِزَانَةِ} (al-xizānati) & \textarabic{خِزَانَتِي} (xizānatī) \\
\hline
\multirow{3}{*}{Dual (\textarabic{مُثَنَّى})} 
 & Nominative   & \textarabic{خِزَانَتَانِ} (xizānatāni) & \textarabic{الْخِزَانَتَانِ} (al-xizānatāni) & \textarabic{خِزَانَتَايَ} (xizānatāya) \\
 & Acc/Gen      & \textarabic{خِزَانَتَيْنِ} (xizānatayni) & \textarabic{الْخِزَانَتَيْنِ} (al-xizānatayni) & \textarabic{خِزَانَتَيَّ} (xizānatayya) \\
\hline
\multirow{3}{*}{Plural (\textarabic{جَمْع})} 
  & Nominative    & \textarabic{خِزَانَاتٌ} (xizānātun) & \textarabic{الْخِزَانَاتُ} (al-xizānātu) & \textarabic{خِزَانَاتِي} (xizānātī) \\
  & Accusative      & \textarabic{خِزَانَاتٍ} (xizānātin) & \textarabic{الْخِزَانَاتِ} (al-xizānāti) & \textarabic{خِزَانَاتِيَّ} (xizānātiyya) \\
  & Genitive          & \textarabic{خِزَانَاتٍ} (xizānātin) & \textarabic{الْخِزَانَاتِ} (al-xizānāti) & \textarabic{خِزَانَاتِي} (xizānātī) \\
\hline
\end{tabular}

\subsection{\textarabic{فَفَتَحَهَا} — [fa-fataħahā]}

\begin{tabular}{p{3cm}p{10cm}}
\toprule
\textbf{Translation} & so he opened it \\
\textbf{Root} & \textarabic{ف-ت-ح} (f-t-ħ) \\
\textbf{Pattern} & \textarabic{فَعَلَ} (faʕala) with conjunction and pronoun \\
\textbf{Grammar} & Past tense verb with conjunction \textarabic{فَ} + attached feminine pronoun \textarabic{هَا} \\
\textbf{Table of Conjugations} & \makecell[l]{
Perfect: \textarabic{فَتَحَ} [fataħa]\\
Imperfect: \textarabic{يَفْتَحُ} [yaftaħu]\\
Active Participle: \textarabic{فَاتِح} [fātiħ]
} \\
\textbf{Examples} & \makecell[l]{\parbox{9.5cm}{
1. \textarabic{فَتَحَ الْبَابَ} - He opened the door [fataħa l-bāba]\\
2. \textarabic{تَفْتَحُ النَّافِذَةَ} - She opens the window [taftaħu n-nāfiḏata]\\
3. \textarabic{افْتَحْ عَيْنَيْكَ} - Open your eyes [iftaħ ʕaynyka]
}} \\
\midrule \\
\textbf{Synonyms} & \textarabic{شَقَّ} (split open), \textarabic{كَشَفَ} (uncovered), \textarabic{حَلَّ} (unlocked) \\
\textbf{Etymology} & From Proto-Semitic *fataħ-, meaning "to open, breach" \\
\bottomrule
\end{tabular}

\subsection{Conjugation}
\begin{longtable}{|>{\raggedright}p{3.5cm}|p{5cm}|p{5cm}|}
\hline
\textbf{Person} & \textbf{Perfect (Past)} & \textbf{Imperfect (Present)} \\
\hline
\textbf{3rd person masculine singular} & \textarabic{فَتَحَ} [fataħa] & \textarabic{يَفْتَحُ} [yaftaħu] \\
\hline
\textbf{3rd person feminine singular} & \textarabic{فَتَحَتْ} [fataħat] & \textarabic{تَفْتَحُ} [taftaħu] \\
\hline
\textbf{3rd person masculine dual} & \textarabic{فَتَحَا} [fataħā] & \textarabic{يَفْتَحَانِ} [yaftaħāni] \\
\hline
\textbf{3rd person feminine dual} & \textarabic{فَتَحَتَا} [fataħatā] & \textarabic{تَفْتَحَانِ} [taftaħāni] \\
\hline
\textbf{3rd person masculine plural} & \textarabic{فَتَحُوا} [fataħū] & \textarabic{يَفْتَحُونَ} [yaftaħūna] \\
\hline
\textbf{3rd person feminine plural} & \textarabic{فَتَحْنَ} [fataħna] & \textarabic{يَفْتَحْنَ} [yaftaħna] \\
\hline
\textbf{2nd person masculine singular} & \textarabic{فَتَحْتَ} [fataħta] & \textarabic{تَفْتَحُ} [taftaħu] \\
\hline
\textbf{2nd person feminine singular} & \textarabic{فَتَحْتِ} [fataħti] & \textarabic{تَفْتَحِينَ} [taftaħīna] \\
\hline
\textbf{2nd person dual (m./f.)} & \textarabic{فَتَحْتُمَا} [fataħtumā] & \textarabic{تَفْتَحَانِ} [taftaħāni] \\
\hline
\textbf{2nd person masculine plural} & \textarabic{فَتَحْتُمْ} [fataħtum] & \textarabic{تَفْتَحُونَ} [taftaħūna] \\
\hline
\textbf{2nd person feminine plural} & \textarabic{فَتَحْتُنَّ} [fataħtunna] & \textarabic{تَفْتَحْنَ} [taftaħna] \\
\hline
\textbf{1st person singular} & \textarabic{فَتَحْتُ} [fataħtu] & \textarabic{أَفْتَحُ} [aftaħu] \\
\hline
\textbf{1st person plural} & \textarabic{فَتَحْنَا} [fataħnā] & \textarabic{نَفْتَحُ} [naftaħu] \\
\hline
\end{longtable}

\subsubsection*{Conjugation Notes}
\begin{itemize}
  \item The \textbf{future} is formed with the prefix \textarabic{سَ} [sa-] or \textarabic{سَوْفَ} [sawfa] before the imperfect (e.g., \textarabic{سَيَفْتَحُ} "he will open").
  \item The \textbf{moods of the imperfect}: 
    \begin{itemize}
      \item Indicative: \textarabic{يَفْتَحُ} [yaftaħu] 
      \item Subjunctive: \textarabic{لَنْ يَفْتَحَ} [lan yaftaħa]
      \item Jussive: \textarabic{لَمْ يَفْتَحْ} [lam yaftaħ]
      \item Imperative: \textarabic{افْتَحْ} [iftaħ!]
    \end{itemize}
  \item \textbf{Passive voice}: 
    \begin{itemize}
      \item Perfect: \textarabic{فُتِحَ} [futiħa] — it was opened
      \item Imperfect: \textarabic{يُفْتَحُ} [yuftaħu] — it is opened
    \end{itemize}
\end{itemize}

\subsection{\textarabic{وَإِذَا} — [wa-ʔiḏā]}

\begin{tabular}{p{3cm}p{10cm}}
\toprule
\textbf{Translation} & and behold, and suddenly \\
\textbf{Root} & Particle (not derived from root) \\
\textbf{Pattern} & Conjunction + surprise particle \\
\textbf{Grammar} & Conjunction \textarabic{وَ} + surprise particle \textarabic{إِذَا} \\
\textbf{Table of Conjugations} & \makecell[l]{
Fixed expression - no conjugation\\
Used to introduce surprise or sudden discovery
} \\
\textbf{Examples} & \makecell[l]{\parbox{9.5cm}{
1. \textarabic{وَإِذَا بِالْمَطَرِ يَهْطِلُ} - And behold, rain is falling [wa-ʔiḏā bil-maṭari yahṭilu]\\
2. \textarabic{فَتَحَ الصُّنْدُوقَ وَإِذَا بِكَنْزٍ} - He opened the box and behold, treasure [fataħa ṣ-ṣundūqa wa-ʔiḏā bi-kanzin]\\
3. \textarabic{وَإِذَا بِهِ قَادِمٌ} - And behold, he is coming [wa-ʔiḏā bihi qādimun]
}} \\
\midrule \\
\textbf{Synonyms} & \textarabic{فَجْأَةً} (suddenly), \textarabic{بَغْتَةً} (unexpectedly) \\
\textbf{Etymology} & \textarabic{إِذَا} from Proto-Semitic *ʔiḏ- meaning "when, if" \\
\bottomrule
\end{tabular}

\subsection{\textarabic{بِجُحَا} — [bi-juħā]}

\begin{tabular}{p{3cm}p{10cm}}
\toprule
\textbf{Translation} & with Juha, there was Juha \\
\textbf{Root} & Proper noun (not derived from root) \\
\textbf{Pattern} & Preposition + proper noun \\
\textbf{Grammar} & Preposition \textarabic{بِ} + proper noun \textarabic{جُحَا} \\
\textbf{Table of Conjugations} & \makecell[l]{
Proper noun - no conjugation\\
\textarabic{جُحَا} is indeclinable
} \\
\textbf{Examples} & \makecell[l]{\parbox{9.5cm}{
1. \textarabic{جُحَا رَجُلٌ حَكِيمٌ} - Juha is a wise man [juħā rajulun ħakīmun]\\
2. \textarabic{قِصَّةُ جُحَا مُضْحِكَةٌ} - Juha's story is funny [qiṣṣatu juħā muḍħikatun]\\
3. \textarabic{يُحِبُّ النَّاسُ جُحَا} - People love Juha [yuħibbu n-nāsu juħā]
}} \\
\midrule \\
\textbf{Synonyms} & \textarabic{نَصْرُ الدِّينِ خُوجَة} (Nasruddin Khoja) - Turkish equivalent \\
\textbf{Etymology} & Folkloric character name, possibly from Arabic \textarabic{جَحَا} meaning "having prominent eyes" \\
\bottomrule
\end{tabular}

\subsection{\textarabic{فِيهَا} — [fīhā]}

\begin{tabular}{p{3cm}p{10cm}}
\toprule
\textbf{Translation} & in it \\
\textbf{Root} & Preposition + pronoun \\
\textbf{Pattern} & \textarabic{فِي} + attached pronoun \\
\textbf{Grammar} & Preposition \textarabic{فِي} + attached feminine pronoun \textarabic{هَا} \\
\textbf{Table of Conjugations} & \makecell[l]{
\textarabic{فِيهِ} - in it (masc.)\\
\textarabic{فِيهَا} - in it (fem.)\\
\textarabic{فِيهِمَا} - in them (dual)\\
\textarabic{فِيهِمْ} - in them (masc. pl.)\\
\textarabic{فِيهِنَّ} - in them (fem. pl.)
} \\
\textbf{Examples} & \makecell[l]{\parbox{9.5cm}{
1. \textarabic{الْكِتَابُ فِي الْحَقِيبَةِ} - The book is in the bag [al-kitābu fī l-ħaqībati]\\
2. \textarabic{يَعِيشُ فِيهَا} - He lives in it [yaʕīšu fīhā]\\
3. \textarabic{مَا فِيهَا مِن شَكٍّ} - There is no doubt in it [mā fīhā min šakkin]
}} \\
\midrule \\
\textbf{Synonyms} & \textarabic{بِدَاخِلِهَا} (inside it), \textarabic{بِهَا} (in it) \\
\textbf{Etymology} & \textarabic{فِي} from Proto-Semitic *fī- meaning "in, at" \\
\bottomrule
\end{tabular}

% ======================== Phrase Analysis ========================
\section{Phrase Analysis}
\begin{tcolorbox}[colback=boxcolor,colframe=headercolor,breakable]
\textbf{Grammatical Structure:}\\
Conjunction + past verb + definite subject + prepositional phrase + relative clause + conjunction + negation + past verb + conjunction + past verb + direct object + conjunction + past verb + object pronoun + conjunction + surprise particle + prepositional phrase + prepositional phrase \\

\textbf{Key Grammar Points:}
\begin{itemize}
\item Sequential conjunctions \textarabic{وَ} and \textarabic{فَ} create narrative flow
\item \textarabic{عَنْ} governs genitive case in \textarabic{شَيْءٍ}
\item Relative clause \textarabic{يَسْرِقُهُ} modifies \textarabic{شَيْءٍ}
\item Negation \textarabic{لَمْ} requires jussive mood in \textarabic{يَجِدْ}
\item \textarabic{وَإِذَا} introduces sudden discovery
\item Attached pronouns \textarabic{هُ} and \textarabic{هَا} show gender agreement
\item \textarabic{بِ} in \textarabic{بِجُحَا} indicates existence/presence
\item This is a compound sentence with multiple clauses connected by conjunctions
\end{itemize}
\end{tcolorbox}

% ======================== Similar Phrases for Practice ========================
\section{Similar Phrases for Practice}

\begin{enumerate}
\item \textarabic{وَبَحَثَ الطِّفْلُ عَنْ لُعْبَتِهِ فَلَمْ يَجِدْهَا، فَرَأَى الصُّنْدُوقَ فَفَتَحَهُ وَإِذَا بِهَا فِيهِ}\\
And the child searched for his toy but did not find it, then he saw the box, opened it, and behold, it was inside [wa-baħaθa ṭ-ṭiflu ʕan luʕbatihi fa-lam yajidahā, fa-raʔā ṣ-ṣundūqa fa-fataħahu wa-ʔiḏā bihā fīhi]

\item \textarabic{وَنَقَّبَ الْعَامِلُ عَنْ مَاءٍ فَلَمْ يَعْثُرْ، فَشَاهَدَ الْبِئْرَ فَحَفَرَهَا وَإِذَا بِالْمَاءِ يَتَدَفَّقُ}\\
And the worker searched for water but did not find any, then he saw the well, dug it, and behold, water was gushing [wa-naqqaba l-ʕāmilu ʕan māʔin fa-lam yaʕθur, fa-šāhada l-biʔra fa-ħafarahā wa-ʔiḏā bil-māʔi yatadaffaqu]

\item \textarabic{وَفَتَّشَ الطَّبِيبُ عَنْ دَوَاءٍ لِلْمَرَضِ فَلَمْ يَكْتَشِفْ، فَلَاحَظَ النَّبَاتَ فَدَرَسَهُ وَإِذَا بِهِ الْعِلَاجُ}\\
And the doctor searched for medicine for the disease but did not discover any, then he noticed the plant, studied it, and behold, it was the cure [wa-fattaša ṭ-ṭabību ʕan dawāʔin lil-maraḍi fa-lam yaktašif, fa-lāħaẓa n-nabāta fa-darasahu wa-ʔiḏā bihi l-ʕilāju]

\item \textarabic{وَدَوَّرَ الطَّالِبُ عَنْ كِتَابِهِ فَلَمْ يَلْقَ، فَبَصُرَ الْمَكْتَبَةَ فَدَخَلَهَا وَإِذَا بِالْكِتَابِ عَلَى الطَّاوِلَةِ}\\
And the student searched around for his book but did not find it, then he saw the library, entered it, and behold, the book was on the table [wa-dawwara ṭ-ṭālibu ʕan kitābihi fa-lam yalqa, fa-baṣura l-maktabata fa-daxalahā wa-ʔiḏā bil-kitābi ʕalā ṭ-ṭāwilati]
\end{enumerate}

% ======================== Levantine Dialect Version ========================
\section{Levantine (Shaami) Arabic Dialect}

\begin{tcolorbox}[colback=white,colframe=dialectcolor,title=\textbf{Levantine Version},breakable]
\textarabic{وَدَوَّرَ الْحَرَامِي عَ شِي يِسْرُقُو بَسْ مَا لَاقِي، فَشَافَ الْخِزَانَة فَفَتَحَا وَإِذَا جُحَا جُوَّاتَا}\\
\textbf{Phonetic:} [wa-dawwar il-ħarāmi ʕa ši yisruqu bass mā lāqi, fa-šāf il-xizāne fa-fataħa wa-ʔiḏā juħā juwwāta]\\
\textbf{Translation:} And the thief searched for something to steal but didn't find anything, then he saw the wardrobe, opened it, and behold, Juha was inside it.
\end{tcolorbox}

\textbf{Key Dialectal Changes:}
\begin{itemize}
\item \textarabic{بَحَثَ} → \textarabic{دَوَّرَ} (dawwar) — dialectal "searched around"
\item \textarabic{اللَّصُّ} → \textarabic{الْحَرَامِي} (il-ħarāmi) — colloquial "thief"
\item \textarabic{عَنْ} → \textarabic{عَ} (ʕa) — shortened preposition "for"
\item \textarabic{شَيْءٍ} → \textarabic{شِي} (ši) — simplified "something"
\item \textarabic{يَسْرِقُهُ} → \textarabic{يِسْرُقُو} (yisruqu) — dialectal present tense
\item \textarabic{فَلَمْ يَجِدْ} → \textarabic{بَسْ مَا لَاقِي} (bass mā lāqi) — "but didn't find"
\item \textarabic{رَأَى} → \textarabic{شَافَ} (šāf) — dialectal past "saw"
\item \textarabic{فَفَتَحَهَا} → \textarabic{فَفَتَحَا} (fa-fataħa) — simplified pronoun
\item \textarabic{فِيهَا} → \textarabic{جُوَّاتَا} (juwwāta) — dialectal "inside it"
\end{itemize}

% ======================== Additional Learning Notes ========================
\section{Additional Learning Notes}

\begin{tcolorbox}[colback=boxcolor,colframe=accentcolor,title=\textbf{Cultural and Literary Context},breakable]
\textbf{Narrative Function:} This sentence serves as the climactic moment in a Juha story where the thief's expectations are completely subverted. Instead of finding valuables, he discovers Juha hiding in the wardrobe.

\textbf{Juha's Character:} Juha (Nasreddin Hodja in Turkish tradition) represents the wise fool archetype in Middle Eastern folklore. His presence in the wardrobe suggests either poverty (nothing worth stealing) or cleverness (hiding from the thief).

\textbf{Literary Technique:} The phrase uses dramatic irony and surprise revelation through \textarabic{وَإِذَا} to create humor and engage the reader. The sequential structure builds suspense before the punchline.

\textbf{Cultural Values:} The story reflects themes of poverty, resourcefulness, and the ability to find humor in difficult situations - common motifs in Arabic folk literature. It also demonstrates the Arabic narrative tradition of using conjunctions to create flow.

\textbf{Grammar Teaching Value:} This sentence is excellent for teaching sequential conjunctions, negation with \textarabic{لَمْ}, attached pronouns, and the dramatic \textarabic{وَإِذَا} construction.
\end{tcolorbox}

\subsection{Memory Tips}
\begin{enumerate}
\item \textbf{Conjunction Chain:} \textit{Search → Find (negated) → See → Open → Surprise} (\textarabic{وَ...فَ...فَ...وَ})
\item \textbf{Pronoun Gender:} Remember \textarabic{هُ} (masculine) refers to \textarabic{شَيْءٍ}, \textarabic{هَا} (feminine) to \textarabic{الْخِزَانَةَ}
\item \textbf{Negation Rule:} \textarabic{لَمْ} + jussive = past negative (\textarabic{لَمْ يَجِدْ} not \textarabic{لَمْ يَجِدُ})
\item \textbf{Surprise Structure:} \textarabic{وَإِذَا بِ...} is a fixed expression meaning "and behold..."
\item \textbf{Relative Clause:} \textarabic{يَسْرِقُهُ} modifies \textarabic{شَيْءٍ} — "something that he steals"
\end{enumerate}

\subsection{Related Vocabulary Family}
\begin{multicols}{2}

\textbf{Search/Investigation family:}\\
\textarabic{بَحْث} — research [baħθ]\\
\textarabic{تَفْتِيش} — searching [taftīš]\\
\textarabic{تَنْقِيب} — excavation [tanqīb]\\
\textarabic{اسْتِكْشَاف} — exploration [istikšāf]\\
\textarabic{فَحْص} — examination [faħṣ]\\

\textbf{Theft family:}\\
\textarabic{سَرِقَة} — theft [sariqa]\\
\textarabic{سَارِق} — thief [sāriq]\\
\textarabic{مَسْرُوق} — stolen [masrūq]\\
\textarabic{نَهْب} — plunder [nahb]\\
\textarabic{لُصُوصِيَّة} — thievery [luṣūṣiyya]\\

\textbf{Vision family:}\\
\textarabic{رُؤْيَة} — vision [ruʔya]\\
\textarabic{نَظَر} — looking [naẓar]\\
\textarabic{إِبْصَار} — sight [ibṣār]\\
\textarabic{مُشَاهَدَة} — watching [mušāhada]\\
\textarabic{مُلَاحَظَة} — observation [mulāħaẓa]\\

\textbf{Opening family:}\\
\textarabic{فَتْح} — opening [fatħ]\\
\textarabic{مِفْتَاح} — key [miftāħ]\\
\textarabic{انْفِتَاح} — being open [infitāħ]\\
\textarabic{فَتَّاح} — opener [fattāħ]\\
\textarabic{مَفْتُوح} — opened [maftūħ]\\

\textbf{Storage family:}\\
\textarabic{خَزْن} — storage [xazn]\\
\textarabic{مَخْزَن} — warehouse [maxzan]\\
\textarabic{خِزَانَة} — cupboard [xizāna]\\
\textarabic{خَازِن} — treasurer [xāzin]\\
\textarabic{مُدَّخَر} — stored [muddaxar]\\

\textbf{Surprise family:}\\
\textarabic{مُفَاجَأَة} — surprise [mufājaʔa]\\
\textarabic{بَغْتَة} — suddenly [baġta]\\
\textarabic{فَجْأَة} — suddenly [fajʔa]\\
\textarabic{صَدْمَة} — shock [ṣadma]\\
\textarabic{ذُهُول} — amazement [ḏuhūl]
\end{multicols}

\subsection{Grammatical Patterns to Remember}

\textbf{Sequential Narrative Pattern:}
\begin{itemize}
\item \textarabic{وَ + فعل} — "and [verb]" (connects to previous action)
\item \textarabic{فَ + فعل} — "so/then [verb]" (shows result or sequence)  
\item \textarabic{وَإِذَا بِ + اسم} — "and behold [noun]" (introduces surprise)
\end{itemize}

\textbf{Negation Patterns:}
\begin{itemize}
\item \textarabic{لَمْ + مجزوم} — past negation with jussive mood
\item \textarabic{مَا + فعل} — simple negation
\item \textarabic{لَا + مضارع} — present/future negation
\end{itemize}

\textbf{Attached Pronoun Patterns:}
\begin{itemize}
\item Verb + \textarabic{هُ/هَا} — "he/she [verbs] it"
\item Preposition + \textarabic{هُ/هَا} — "in/with/for it"
\item Noun + \textarabic{هُ/هَا} — "his/her [noun]"
\end{itemize}

\textbf{Form Patterns in the Text:}
\begin{itemize}
\item Form I verbs: \textarabic{بَحَثَ، سَرَقَ، وَجَدَ، رَأَى، فَتَحَ} — basic action verbs
\item Sound plurals: \textarabic{لُصُوص} (broken plural from \textarabic{لِصّ})
\item Feminine nouns with \textarabic{ـة}: \textarabic{خِزَانَة} pattern \textarabic{فِعَالَة}
\item Defective verbs: \textarabic{رَأَى} (hollow verb with weak middle radical)
\end{itemize}

\end{document}