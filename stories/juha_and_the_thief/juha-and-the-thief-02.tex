\documentclass[a4paper,12pt]{article}

\usepackage[utf8]{inputenc}
\usepackage[english,arabic]{babel}
\usepackage{tikz}
\usepackage{geometry}
\usepackage{array}
\usepackage{multicol}
\usepackage{xcolor}
\usepackage[table]{xcolor} % for \rowcolor in tables
\usepackage{polyglossia}
\setmainlanguage{english}
\setotherlanguage{arabic}
\usepackage{arabtex}
\usepackage{utf8}
\usepackage{fontspec}
\setmainfont{Charis SIL}
\newfontfamily\arabicfont[Script=Arabic,Scale=1.2]{Amiri}
\geometry{margin=1.5cm}
\usepackage{tcolorbox}
\usepackage{booktabs}
\usepackage{graphicx}
\usetikzlibrary{shapes,arrows,decorations.pathmorphing}

% Custom colors
\definecolor{headercolor}{RGB}{70,130,180}
\definecolor{boxcolor}{RGB}{240,248,255}
\definecolor{accentcolor}{RGB}{220,20,60}
\definecolor{tableheader}{RGB}{220,220,220}

\begin{document}

\title{\textbf{\Large Arabic Learning Poster}\\
\large Juha and the Thief – Full Phrase Analysis\\
\normalsize \textarabic{وَبَحَثَ اللَّصُّ عَنْ شَيْءٍ يَسْرِقُهُ فَلَمْ يَجِدْ، فَرَأَى الْخِزَانَةَ فَفَتَحَهَا وَإِذَا بِجُحَا فِيهَا}}
\author{Teaching Material for Arabic Students}
\date{}
\maketitle

% ======================== Phrase Display ========================
\begin{tcolorbox}[colback=boxcolor,colframe=headercolor,title=\textbf{Arabic Text Analysis}]
\centering
\textarabic{وبحث اللص عن شيء يسرقه فلم يجد، فرأى الخزانة ففتحها وإذا بجحا فيها}
\\[0.5em]
\textbf{Without Diacritics}
\\[1em]
\textarabic{وَبَحَثَ اللَّصُّ عَنْ شَيْءٍ يَسْرِقُهُ فَلَمْ يَجِدْ، فَرَأَى الْخِزَانَةَ فَفَتَحَهَا وَإِذَا بِجُحَا فِيهَا}
\\[0.5em]
\textbf{With Full Diacritics}
\end{tcolorbox}

% ======================== Translations ========================
\section{Word-by-Word English Translation}
\begin{tcolorbox}[colback=white,colframe=accentcolor]
\textbf{Literal:} And searched • the thief • for • something • he steals-it • so-did not find, • then saw • the wardrobe • so opened-it • and behold • Juha • in it \\
\textit{[Arabic order retained for direct mapping]}\\[0.5em]
\textbf{Adapted:} The thief searched for something to steal but found nothing, then he saw the wardrobe, opened it, and behold, Juha was inside.
\end{tcolorbox}

% ======================== Detailed Word Analysis ========================
\section{Detailed Word Analysis}

\subsection{\textarabic{وَبَحَثَ} – \textbf{wa-baḥatha} [wabahaθa]}
\begin{tabular}{p{3cm}p{10cm}}
\toprule
\textbf{Translation} & and searched/looked for \\
\textbf{Root} & \textarabic{ب-ح-ث} (b-ḥ-θ) \\
\textbf{Pattern} & \textarabic{فَعَلَ} (faʕala) \\
\textbf{Grammar} & Past tense verb, 3rd person masculine singular, with conjunction \textarabic{وَ} (and) prefix \\
\textbf{Examples} & \textarabic{بَحَثَ عَنْ كِتَابٍ} (searched for a book) [baḥatha ʕan kitābin], \textarabic{نَحْنُ نَبْحَثُ عَنْ الحَلّ} (we are searching for the solution) [naḥnu nabḥathu ʕan al-ḥall] \\
\bottomrule
\end{tabular}

\textbf{Conjugation Examples:}
\begin{multicols}{2}
\small
\textarabic{بَحَثْتُ} – I searched [baḥaθtu] \\
\textarabic{بَحَثْتَ} – you searched (m.) [baḥaθta] \\
\textarabic{بَحَثَتْ} – she searched [baḥaθat] \\
\textarabic{يَبْحَثُ} – he searches (pres.) [yabḥathu] \\
\textarabic{تَبْحَثُ} – she searches (pres.) [tabḥathu]
\end{multicols}

\subsection{\textarabic{اللَّصُّ} – \textbf{al-liṣṣu} [alːiṣːu]}
\begin{tabular}{p{3cm}p{10cm}}
\toprule
\textbf{Translation} & the thief \\
\textbf{Root} & \textarabic{ل-ص-ص} (l-ṣ-ṣ) \\
\textbf{Pattern} & \textarabic{ال} + noun definite article + singular noun \\
\textbf{Grammar} & Noun, definite, nominative case (subject of verb) \\
\textbf{Examples} & \textarabic{اللِّصُّ دَخَلَ الْبَيْتَ} (the thief entered the house) [al-liṣṣu daxala l-bayta], \textarabic{الْلُّصُوصُ يَسْرِقُونَ} (the thieves steal) [al-luṣūṣu yasriqūn] \\
\bottomrule
\end{tabular}

\subsection{\textarabic{عَنْ} – \textbf{ʕan} [ʕan]}
\begin{tabular}{p{3cm}p{10cm}}
\toprule
\textbf{Translation} & about, for, from (preposition) \\
\textbf{Grammar} & Preposition governing genitive case \\
\textbf{Examples} & \textarabic{سَأَلَتْ عَنْ الْوَقْتِ} (she asked about the time) [saʔalat ʕan al-waqti], \textarabic{يَبْحَثُ عَنْ الْكِتَابِ} (he searches for the book) [yabḥathu ʕan al-kitābi] \\
\bottomrule
\end{tabular}

\subsection{\textarabic{شَيْءٍ} – \textbf{shay'in} [ʃajʔin]}
\begin{tabular}{p{3cm}p{10cm}}
\toprule
\textbf{Translation} & something, thing \\
\textbf{Root} & \textarabic{ش-ي-ء} (sh-y-ʔ) \\
\textbf{Grammar} & Indefinite noun, genitive case (due to preposition \textarabic{عَنْ}) \\
\textbf{Examples} & \textarabic{أَرِيدُ شَيْئًا} (I want something) [ʔarīdu ʃajʔan], \textarabic{هَلْ عِنْدَكَ شَيْءٌ؟} (Do you have something?) [hal ʕindaka ʃajʔun?] \\
\bottomrule
\end{tabular}

\subsection{\textarabic{يَسْرِقُهُ} – \textbf{yasriquhu} [jasriquhu]}
\begin{tabular}{p{3cm}p{10cm}}
\toprule
\textbf{Translation} & he steals it \\
\textbf{Root} & \textarabic{س-ر-ق} (s-r-q) \\
\textbf{Pattern} & Form I, Present tense, 3rd person masculine singular + object pronoun \\
\textbf{Grammar} & Verb + attached object pronoun \\
\textbf{Simple Form} & \textarabic{سَرَقَ} (he stole) [saraqa] \\
\textbf{Examples} & \textarabic{يَسْرِقُ المالَ} (he steals the money) [yasriqu l-māla], \textarabic{سَرَقْتُ سَيَّارَةً} (I stole a car) [saraqtu sayyāratan] \\
\bottomrule
\end{tabular}

\subsection{\textarabic{فَلَمْ يَجِدْ} – \textbf{fa-lam yajid} [fa-lam jæjid]}
\begin{tabular}{p{3cm}p{10cm}}
\toprule
\textbf{Translation} & so did not find \\
\textbf{Grammar} & \textarabic{فَ} (so) + negation \textarabic{لَمْ} + past tense verb \\
\textbf{Verb} & \textarabic{وَجَدَ} (to find), 3rd person masc. sing., past tense \\
\textbf{Examples} & \textarabic{لَمْ يَجِدْ الْكِتَابَ} (he did not find the book) [lam yajid al-kitāba] \\
\bottomrule
\end{tabular}

\subsection{\textarabic{فَرَأَى} – \textbf{fa-ra'ā} [fa-raʔā]}
\begin{tabular}{p{3cm}p{10cm}}
\toprule
\textbf{Translation} & then he saw \\
\textbf{Root} & \textarabic{ر-أ-ى} (r-ʔ-y) \\
\textbf{Pattern} & Form I, past tense, 3rd person masc. sing. with prefix \textarabic{فَ} (then) \\
\textbf{Examples} & \textarabic{رَأَى الطِّفْلَ} (he saw the child) [raʔā ṭ-ṭifla] \\
\bottomrule
\end{tabular}

\subsection{\textarabic{الْخِزَانَةَ} – \textbf{al-xizānata} [al-xizāːnata]}
\begin{tabular}{p{3cm}p{10cm}}
\toprule
\textbf{Translation} & the wardrobe (accusative) \\
\textbf{Root} & \textarabic{خ-ز-ن} (x-z-n) \\
\textbf{Pattern} & \textarabic{فِعَالَة} (fiʕāla) \\
\textbf{Grammar} & Feminine noun, definite, accusative case \\
\textbf{Examples} & \textarabic{وَضَعْتُ كُلَّ مَلَابِسِي فِي الْخِزَانَةِ} (I put all my clothes in the wardrobe) [waḍaʕtu kulla malābisi fī l-xizāna] \\
\bottomrule
\end{tabular}

\subsection{\textarabic{فَفَتَحَهَا} – \textbf{fa-fataḥahā} [fa-fataħaːha]}
\begin{tabular}{p{3cm}p{10cm}}
\toprule
\textbf{Translation} & so he opened it \\
\textbf{Root} & \textarabic{ف-ت-ح} (f-t-ḥ) \\
\textbf{Grammar} & Past tense verb \textarabic{فَتَحَ} with prefix \textarabic{فَ} (so/then) + attached pronoun \textarabic{هَا} (it, feminine) \\
\textbf{Examples} & \textarabic{فَتَحَ البَابَ} (he opened the door) [fataḥa l-bāba] \\
\bottomrule
\end{tabular}

\subsection{\textarabic{وَإِذَا} – \textbf{wa-ʾiḏā} [wa ʔiðā]}
\begin{tabular}{p{3cm}p{10cm}}
\toprule
\textbf{Translation} & and behold / and suddenly \\
\textbf{Grammar} & Conjunction \textarabic{وَ} (and) + particle \textarabic{إِذَا} (if/when, often used to introduce surprise) \\
\textbf{Examples} & \textarabic{وَإِذَا الطِّفْلُ يَبْكِي} (and behold, the child is crying) [wa ʔiðā ṭ-ṭiflu yabkī] \\
\bottomrule
\end{tabular}

\subsection{\textarabic{بِجُحَا} – \textbf{bi-Juḥā} [bi-d͡ʒuħaː]}
\begin{tabular}{p{3cm}p{10cm}}
\toprule
\textbf{Translation} & with Juha / there was Juha \\
\textbf{Components} & Preposition \textarabic{بِ} (with/in) + proper noun \textarabic{جُحَا} \\
\textbf{Examples} & \textarabic{رَأَيْتُ جُحَا فِي السُّوقِ} (I saw Juha in the market) [raʔaytu juḥā fī s-sūqi] \\
\bottomrule
\end{tabular}

\subsection{\textarabic{فِيهَا} – \textbf{fī-hā} [fiːhaː]}
\begin{tabular}{p{3cm}p{10cm}}
\toprule
\textbf{Translation} & in it \\
\textbf{Components} & \textarabic{في} (in) + \textarabic{ها} (it, feminine) \\
\textbf{Grammar} & Prepositional phrase with attached pronoun \\
\textbf{Examples} & \textarabic{كُتِبَ فِيهَا} (written in it) [kutiba fiːhaː] \\
\bottomrule
\end{tabular}

% ======================== Phrase Analysis ========================
\section{Phrase Analysis}

\begin{tcolorbox}[colback=boxcolor,colframe=headercolor]
\textbf{Grammatical Structure:}\\
Conjunction + verb + subject + prepositional phrase + object + negation phrase + conjunction + verb + object + conjunction + verb + subject + prepositional phrase \\
\\
\textbf{Key Grammar Points:}
\begin{itemize}
\item The prefix \textarabic{وَ} is the conjunction “and.”
\item \textarabic{عَنْ} governs genitive case, so the noun after it is genitive (\textarabic{شَيْءٍ}).
\item Negation \textarabic{لَمْ} with past tense verb negates the action (\textarabic{يَجِدْ}).
\item The phrase \textarabic{فَرَأَى الْخِزَانَةَ} shows a past verb with a direct object.
\item Attached pronouns like in \textarabic{يَسْرِقُهُ}, \textarabic{فَفَتَحَهَا}, \textarabic{فِيهَا} refer back to masculine or feminine nouns respectively.
\item \textarabic{وَإِذَا} is a set phrase often used to introduce a sudden or surprising event.
\end{itemize}
\end{tcolorbox}

% ======================== Additional Practice Phrases ========================
\section{Similar Phrases for Practice}

\begin{enumerate}
\item \textarabic{وَجَدَ اللَّصُّ كِتَابًا فِي الْغُرْفَةِ، فَفَتَحَهُ وَقَرَأَهُ}\\
The thief found a book in the room, so he opened it and read it. [wajada al-liṣṣu kitāban fī l-ġurfati, fataḥahu waqaraʔahu]

\item \textarabic{بَحَثَ الطِّفْلُ عَنْ كُرْسِيٍّ فِي الْمَكْتَبَةِ، فَجَلَسَ عَلَيْهِ}\\
The child searched for a chair in the library, so he sat on it. [baḥatha ṭ-ṭiflu ʕan kursiyyin fī l-maktabati, fa-jalasa ʕalayh]

\item \textarabic{رَأَى رَجُلٌ كَلْبًا يَجْرِي فِي الشَّارِعِ، فَتَبِعَهُ}\\
A man saw a dog running in the street, so he followed it. [raʔā rajulun kalban yajri fī ʃ-ʃāriʕi, fata biʕahu]
\end{enumerate}

% ======================== Levantine Dialect Tip ========================
\section{Levantine Arabic Dialect Tip}

In Levantine Arabic, the verb \textarabic{بَحَثَ} (past tense) is commonly replaced by \textarabic{دَوَّرَ} \textit{(dawwar)}, which means “to search” or “to look for.”

Example: \textarabic{دَوَّر اللِّصّ عَ شِي يِسْرُقُه} \\
\textit{[dawwar al-liṣṣ ʕa ši yisruquh]} \\
Translation: The thief searched for something to steal.

\end{document}