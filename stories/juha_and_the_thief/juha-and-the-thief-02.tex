\documentclass[a4paper,12pt]{article}
\usepackage{fontspec}
\usepackage{polyglossia}
\setmainlanguage{english}
\setotherlanguage{arabic}
\newfontfamily\arabicfont[Script=Arabic,Scale=1.4]{Charis SIL}
\usepackage{geometry}
\geometry{margin=1.5cm}
\usepackage{array}
\usepackage{tcolorbox}
\usepackage{titlesec}
\usepackage{enumitem}
\usepackage{xcolor}
\usepackage{multicol}
\usepackage{setspace}

% Color scheme
\definecolor{mainbg}{HTML}{F9F9F9}
\definecolor{sectionbg}{HTML}{E3F2FD}
\definecolor{accent}{HTML}{1976D2}
\definecolor{accent2}{HTML}{C2185B}
\definecolor{tableheader}{HTML}{BBDEFB}

% Section title format
\titleformat{\section}{\large\bfseries\color{accent}}{}{0em}{}
\titleformat{\subsection}{\bfseries\color{accent2}}{}{0em}{}

% Box styles
\tcbset{
  colframe=accent,
  colback=white,
  sharp corners,
  boxrule=0.8pt,
  arc=3pt,
}

\begin{document}

% ===== TITLE =====
\begin{center}
    {\Huge \bfseries Juha and the Thief – Phrase Analysis}\\[4pt]
    {\Large \textarabic{وبحث اللص عن شيء يسرقه فلم يجد, فرأى الخزانة ففتحها وإذا بجحا فيها}}\\[6pt]
    \emph{From the Arabic short story “Juha and the Thief”}
\end{center}

% ===== PRONUNCIATION =====
\begin{tcolorbox}[colback=sectionbg, colframe=accent, title=Phonetic Transcription]
\begin{spacing}{1.2}
wa-baḥatha al-liṣṣu ʿan shayʾin yasriquhu fa-lam yajid, fa-raʾā al-khizānah fa-fataḥahā wa-idhā bi-juḥā fīhā
\end{spacing}
\end{tcolorbox}

% ===== SECTION 1: Word-by-word literal translation =====
\section{Word-by-Word Literal Translation}
\renewcommand{\arraystretch}{1.4}
\begin{tabular}{|>{\arabicfont}m{3cm}|m{4.5cm}|m{7cm}|}
\hline
\rowcolor{tableheader} Arabic & Phonetics & Literal meaning \\
\hline
وبحث & wa-baḥatha & and searched \\
اللص & al-liṣṣ & the thief \\
عن & ʿan & about / for \\
شيء & shayʾ & thing \\
يسرقه & yasriquhu & steal it \\
فلم & fa-lam & so not \\
يجد & yajid & find \\
فرأى & fa-raʾā & so he saw \\
الخزانة & al-khizānah & the cupboard / wardrobe \\
ففتحها & fa-fataḥahā & so he opened it \\
وإذا & wa-idhā & and suddenly / behold \\
بجحا & bi-juḥā & with Juha \\
فيها & fīhā & inside it \\
\hline
\end{tabular}

% ===== SECTION 2: Adapted translation =====
\section{Adapted Translation}
\begin{tcolorbox}[colback=sectionbg]
The thief searched for something to steal but found nothing. Then he saw the cupboard, opened it, and there was Juha inside!
\end{tcolorbox}

% ===== SECTION 3: Word-by-word analysis =====
\section{Word-by-Word Analysis}
\renewcommand{\arraystretch}{1.4}
\begin{tabular}{|>{\arabicfont}m{2.5cm}|m{2.5cm}|m{3cm}|m{6cm}|}
\hline
\rowcolor{tableheader} Arabic & Root & Part of Speech & Grammar / Notes \\
\hline
وبحث & ب-ح-ث & Verb (past) & 3rd person masc. sing., connected with \textarabic{و} meaning “and” \\
اللص & ل-ص-ص & Noun & Definite (with \textarabic{ال}), “thief” \\
عن & --- & Prep. & Means “about/for” \\
شيء & ش-ي-ء & Noun & “thing”; indefinite \\
يسرقه & س-ر-ق & Verb (present) + pron. suffix & “he steals it” \\
فلم & --- & Connector + negation & \textarabic{ف} = so, \textarabic{لم} = did not \\
يجد & و-ج-د & Verb (present jussive) & “he find” (negated by \textarabic{لم}) \\
فرأى & ر-أ-ي & Verb (past) & “so he saw” \\
الخزانة & خ-ز-ن & Noun & “the cupboard”; definite \\
ففتحها & ف-ت-ح & Verb (past) + pron. suffix & “so he opened it” \\
وإذا & --- & Fixed phrase & “and suddenly” / “behold” \\
بجحا & --- & Prep. + proper name & \textarabic{ب} = with, “Juha” \\
فيها & --- & Prep. + pron. suffix & “inside it” \\
\hline
\end{tabular}

% ===== SECTION 4: Full phrase analysis =====
\section{Full Phrase Analysis}
\begin{itemize}[noitemsep]
    \item Sequence uses \textarabic{ف} connectors to show quick events.
    \item \textarabic{وإذا} signals surprise.
    \item Verbs alternate between past for narration and present (jussive) for negation.
    \item The suffixes \textarabic{ـه} and \textarabic{ـها} replace separate pronouns.
\end{itemize}

% ===== SECTION 5: Similar phrases =====
\section{Similar Phrases}
\begin{itemize}
    \item \textarabic{فلم يجد شيئاً} – So he didn’t find anything.
    \item \textarabic{وإذا بالباب يفتح} – And suddenly the door opened.
    \item \textarabic{بحثت عن كتابي فلم أجده} – I looked for my book but didn’t find it.
\end{itemize}

% ===== SECTION 6: Levantine Dialect Version =====
\section{Levantine Dialect}
\begin{tcolorbox}[colback=sectionbg]
\textarabic{ودوّر الحرامي على شي يسرقو، وما لقى، فشاف الخزانة وفتحها، وإذَا بجحا جوّا}
\end{tcolorbox}
\textbf{Notes:}  
\begin{itemize}
    \item \textarabic{ودوّر} instead of \textarabic{وبحث} – more colloquial.  
    \item \textarabic{ما لقى} instead of \textarabic{فلم يجد} – typical negation pattern.  
    \item \textarabic{جوّا} instead of \textarabic{فيها}.  
\end{itemize}

% ===== SECTION 7: Extra Learning Notes =====
\section{Extra Learning Notes}
\begin{itemize}
    \item The word \textarabic{اللص} always takes shadda on \textarabic{ص}.
    \item \textarabic{وإذا} is a storytelling trigger for surprise.
    \item In spoken Arabic, many suffix pronouns are shortened.
\end{itemize}

\end{document}

