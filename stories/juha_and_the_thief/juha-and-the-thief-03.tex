\documentclass[letterpaper,12pt]{article}

\usepackage{fancyhdr}
\usepackage[most]{tcolorbox}
\usepackage{longtable}
\usepackage{makecell}
\usepackage{tikz}
\usepackage{geometry}
\usepackage{array}
\usepackage{multicol}
\usepackage[table]{xcolor}
\usepackage{polyglossia}
\setmainlanguage{english}
\setotherlanguage{arabic}
\setotherlanguage{hebrew}
\usepackage{fontspec}
\setmainfont{Charis SIL}
\newfontfamily\arabicfont[Script=Arabic,Scale=1.2]{Amiri}
\newfontfamily\hebrewfont{Ezra SIL}[Script=Hebrew]
\geometry{margin=1.5cm}
\usepackage{booktabs}
\usepackage{graphicx}
\usepackage{multirow}

\setlength{\headheight}{15pt}
\addtolength{\topmargin}{-2.5pt}
\pagestyle{fancy}
\fancyhf{}
\fancyfoot[C]{\thepage}
\fancyhead[R]{\textit{The Thief's Fear and Courage}}
\renewcommand{\headrulewidth}{0pt}

\usetikzlibrary{shapes,arrows,decorations.pathmorphing}

% Custom colors
\definecolor{headercolor}{RGB}{70,130,180}
\definecolor{boxcolor}{RGB}{240,248,255}
\definecolor{accentcolor}{RGB}{220,20,60}
\definecolor{tableheader}{RGB}{220,220,220}
\definecolor{dialectcolor}{RGB}{34,139,34}

\begin{document}

\title{\textbf{\Large Arabic Phrase Analyser}\\
\large The Thief's Fear and Courage\\
\normalsize \textarabic{فخاف اللص ولكنه تشجع وقال: ماذا تفعل هنا يا شيخ؟}}
\author{Igor Deruga}
\date{}
\maketitle

% ======================== Phrase Display ========================
\begin{tcolorbox}[colback=boxcolor,colframe=headercolor,title=\textbf{Arabic Phrase with Full Diactrics},breakable]
\centering
\textarabic{فَخَافَ اللِّصُّ وَلَكِنَّهُ تَشَجَّعَ وَقَالَ: مَاذَا تَفْعَلُ هُنَا يَا شَيْخ؟}
\end{tcolorbox}

% ======================== Translations ========================
\section{English Translation}
\begin{tcolorbox}[colback=white,colframe=accentcolor,breakable]
\textbf{Literal:} So he feared the thief but he encouraged himself and he said: what do you do here O old man? \\
\textit{[Arabic order retained for direct mapping]}\\[0.5em]
\textbf{Adapted:} So the thief became afraid, but he gathered his courage and said: "What are you doing here, old man?"
\end{tcolorbox}

% ======================== Detailed Word Analysis ========================
\section{Detailed Word Analysis}

\subsection{\textarabic{فَخَافَ} — [faxaːfa]}
\begin{tabular}{p{3cm}p{10cm}}
\toprule
\textbf{Translation} & so he feared / so he became afraid \\
\textbf{Root} & \textarabic{خ-و-ف} (x-w-f) \\
\textbf{Pattern} & \textarabic{فَعَلَ} (faʕala) with conjunction \textarabic{فَ} (fa-) prefix \\
\textbf{Grammar} & Past tense verb, 3rd person masculine singular, with conjunction \textarabic{فَ} (fa-) prefix \\
\midrule \\
\textbf{Examples} & \makecell[l]{\parbox{9.5cm}{
1. \textarabic{خَافَ الطِّفْلُ مِنَ الظَّلَامِ} - The child feared the darkness [xaːfa tˤ-tˤiflu min azˤ-zˤalaːm]\\
2. \textarabic{سَيَخَافُ مِنْ هَذَا الصَّوْتِ} - He will fear this sound [sajaxaːfu min haːða sˤ-sˤawt]\\
3. \textarabic{لَا تَخَفْ مِنْ شَيْءٍ} - Don't fear anything [laː taxaf min ʃajʔ]
}} \\
\midrule \\
\textbf{Synonyms} & \textarabic{رَهِبَ} (feared), \textarabic{فَزِعَ} (was frightened), \textarabic{وَجِلَ} (was anxious) \\
\textbf{Etymology} & From Proto-Semitic *xwp, related to Hebrew \texthebrew{חפף} (ḥafaf) "to cover, protect" \\
\bottomrule
\end{tabular}

\subsection{Conjugation}
\large \textbf{Verbal Noun}: \textarabic{خَوْف} /xawf/

\begin{longtable}{|>{\raggedright}p{3.5cm}|p{5cm}|p{5cm}|}
\hline
\textbf{Person} & \textbf{Perfect (Past)} & \textbf{Imperfect (Present)} \\
\hline
\textbf{3rd person masculine singular} & \textarabic{خَافَ} /xaːfa/ & \textarabic{يَخَافُ} /jaxaːfu/ \\
\hline
\textbf{3rd person feminine singular} & \textarabic{خَافَتْ} /xaːfat/ & \textarabic{تَخَافُ} /taxaːfu/ \\
\hline
\textbf{3rd person masculine dual} & \textarabic{خَافَا} /xaːfaː/ & \textarabic{يَخَافَانِ} /jaxaːfaːni/ \\
\hline
\textbf{3rd person feminine dual} & \textarabic{خَافَتَا} /xaːfataː/ & \textarabic{تَخَافَانِ} /taxaːfaːni/ \\
\hline
\textbf{3rd person masculine plural} & \textarabic{خَافُوا} /xaːfuː/ & \textarabic{يَخَافُونَ} /jaxaːfuːna/ \\
\hline
\textbf{3rd person feminine plural} & \textarabic{خِفْنَ} /xifna/ & \textarabic{يَخِفْنَ} /jaxifna/ \\
\hline
\textbf{2nd person masculine singular} & \textarabic{خِفْتَ} /xifta/ & \textarabic{تَخَافُ} /taxaːfu/ \\
\hline
\textbf{2nd person feminine singular} & \textarabic{خِفْتِ} /xifti/ & \textarabic{تَخَافِينَ} /taxaːfiːna/ \\
\hline
\textbf{2nd person dual (m./f.)} & \textarabic{خِفْتُمَا} /xiftumaː/ & \textarabic{تَخَافَانِ} /taxaːfaːni/ \\
\hline
\textbf{2nd person masculine plural} & \textarabic{خِفْتُمْ} /xiftum/ & \textarabic{تَخَافُونَ} /taxaːfuːna/ \\
\hline
\textbf{2nd person feminine plural} & \textarabic{خِفْتُنَّ} /xiftunna/ & \textarabic{تَخِفْنَ} /taxifna/ \\
\hline
\textbf{1st person singular} & \textarabic{خِفْتُ} /xiftu/ & \textarabic{أَخَافُ} /ʔaxaːfu/ \\
\hline
\textbf{1st person plural} & \textarabic{خِفْنَا} /xifnaː/ & \textarabic{نَخَافُ} /naxaːfu/ \\
\hline
\end{longtable}

\subsubsection*{Additional Verb Forms}
\begin{itemize}
  \item \textbf{Future}: \textarabic{سَيَخَافُ} [sajaxaːfu] — he will fear
  \item \textbf{Subjunctive}: \textarabic{أَنْ يَخَافَ} [ʔan jaxaːfa] — that he fear
  \item \textbf{Jussive}: \textarabic{لَمْ يَخَفْ} [lam jaxaf] — he did not fear
  \item \textbf{Imperative}: \textarabic{خَفْ} [xaf] — fear! (masc. sing.)
  \item \textbf{Passive Perfect}: \textarabic{خِيفَ} [xiːfa] — it was feared
  \item \textbf{Passive Imperfect}: \textarabic{يُخَافُ} [juxaːfu] — it is feared
\end{itemize}

\subsection{\textarabic{اللِّصُّ} — [al-lisˤsˤu]}

\begin{tabular}{p{3cm}p{10cm}}
\toprule
\textbf{Translation} & the thief \\
\textbf{Root} & \textarabic{ل-ص-ص} (l-sˤ-sˤ) \\
\textbf{Pattern} & \textarabic{فِعْل} (fiʕl) with definite article \textarabic{ال} \\
\textbf{Grammar} & Definite noun, masculine, nominative case (subject) \\
\textbf{Examples} & \makecell[l]{\parbox{9.5cm}{
1. \textarabic{رَأَيْتُ لِصًّا فِي الشَّارِعِ} - I saw a thief in the street [raʔajtu lisˤsˤan fiː ʃ-ʃaːriʕ]\\
2. \textarabic{اللُّصُوصُ سَرَقُوا الْبَيْتَ} - The thieves robbed the house [al-lusˤuːsˤu saraquː l-bajt]\\
3. \textarabic{حُكِمَ عَلَى اللِّصِّ بِالسِّجْنِ} - The thief was sentenced to prison [ħukima ʕalaː l-lisˤsˤi bi-s-siʤn]
}} \\
\midrule
\textbf{Synonyms} & \textarabic{سَارِق} (thief), \textarabic{لُصُوص} (thieves-pl), \textarabic{قَاطِع طَرِيق} (highway robber) \\
\textbf{Etymology} & From root meaning "to stick, adhere" - one who "sticks" to others' property \\
\bottomrule
\end{tabular}

\subsubsection*{Full Declension Matrix}

\begin{tabular}{|c|c|c|c|c|}
\hline
\textbf{Number} & \textbf{Case} & \textbf{Indefinite} & \textbf{Definite} & \textbf{With Suffix (1st sg.)} \\
\hline
\multirow{3}{*}{Singular (\textarabic{مُفْرَد})}
& Nominative & \textarabic{لِصٌّ} /lisˤsˤun/ & \textarabic{اللِّصُّ} /al-lisˤsˤu/ & \textarabic{لِصِّي} /lisˤsˤiː/ \\
& Accusative & \textarabic{لِصًّا} /lisˤsˤan/ & \textarabic{اللِّصَّ} /al-lisˤsˤa/ & \textarabic{لِصِّي} /lisˤsˤiː/ \\
& Genitive & \textarabic{لِصٍّ} /lisˤsˤin/ & \textarabic{اللِّصِّ} /al-lisˤsˤi/ & \textarabic{لِصِّي} /lisˤsˤiː/ \\
\hline
\multirow{3}{*}{Dual (\textarabic{مُثَنَّى})}
& Nominative & \textarabic{لِصَّانِ} /lisˤsˤaːni/ & \textarabic{اللِّصَّانِ} /al-lisˤsˤaːni/ & \textarabic{لِصَّايَ} /lisˤsˤaːja/ \\
& Acc/Gen & \textarabic{لِصَّيْنِ} /lisˤsˤajni/ & \textarabic{اللِّصَّيْنِ} /al-lisˤsˤajni/ & \textarabic{لِصَّيَّ} /lisˤsˤajja/ \\
\hline
\multirow{3}{*}{Plural (\textarabic{جَمْع})}
& Nominative & \textarabic{لُصُوصٌ} /lusˤuːsˤun/ & \textarabic{اللُّصُوصُ} /al-lusˤuːsˤu/ & \textarabic{لُصُوصِي} /lusˤuːsˤiː/ \\
& Accusative & \textarabic{لُصُوصًا} /lusˤuːsˤan/ & \textarabic{اللُّصُوصَ} /al-lusˤuːsˤa/ & \textarabic{لُصُوصِي} /lusˤuːsˤiː/ \\
& Genitive & \textarabic{لُصُوصٍ} /lusˤuːsˤin/ & \textarabic{اللُّصُوصِ} /al-lusˤuːsˤi/ & \textarabic{لُصُوصِي} /lusˤuːsˤiː/ \\
\hline
\end{tabular}

\subsection{\textarabic{وَلَكِنَّهُ} — [walaːkinnahuː]}

\begin{tabular}{p{3cm}p{10cm}}
\toprule
\textbf{Translation} & but he / however he \\
\textbf{Root} & \textarabic{ل-ك-ن} (l-k-n) with conjunction and pronoun \\
\textbf{Pattern} & \textarabic{وَ} + \textarabic{لَكِنَّ} + \textarabic{هُ} (conjunction + but + pronoun) \\
\textbf{Grammar} & Adversative conjunction with attached 3rd person masculine singular pronoun \\
\textbf{Examples} & \makecell[l]{\parbox{9.5cm}{%
1. \textarabic{وَقَالَ الْجُحَا لِلْأُسْتَاذِ} - And Juha said to the professor [waqaːla l-ʤuħaː li-l-ʔustaːð]\\
2. \textarabic{وَلَكِنَّهَا قَالَتْ شَيْئًا مُمْتَازًا} - But she said something excellent [walaːkinnahaː qaːlat ʃajʔan mumtaːzan]\\
3. \textarabic{وَلَكِنَّهُمْ رَفَضُوا الْاقْتِرَاحَ} - But they rejected the suggestion [walaːkinnhum rafadˤuː l-iqtiraːħ]%
}} \\
\midrule
\textbf{Synonyms} & \textarabic{غَيْر أَنَّهُ} (except that he), \textarabic{إِلَّا أَنَّهُ} (but that he) \\
\textbf{Etymology} & \textarabic{لَكِنَّ} from root meaning "to bend, turn" - a turning point in meaning \\
\bottomrule
\end{tabular}

\subsection{\textarabic{تَشَجَّعَ} — [taʃaʤʤaʕa]}

\begin{tabular}{p{3cm}p{10cm}}
\toprule
\textbf{Translation} & he encouraged himself / he gathered courage \\
\textbf{Root} & \textarabic{ش-ج-ع} (ʃ-ʤ-ʕ) \\
\textbf{Pattern} & \textarabic{تَفَعَّلَ} (tafaʕʕala) - Form V \\
\textbf{Grammar} & Past tense verb, 3rd person masculine singular, Form V (reflexive) \\
\textbf{Examples} & \makecell[l]{\parbox{9.5cm}{
1. \textarabic{تَشَجَّعَ الْجُنْدِيُّ لِلْمَعْرَكَةِ} - The soldier encouraged himself for battle [taʃaʤʤaʕa l-ʤundijju li-l-maʕraka]\\
2. \textarabic{يَتَشَجَّعُ كُلَّ يَوْمٍ} - He encourages himself every day [jataʃaʤʤaʕu kulla jawm]\\
3. \textarabic{تَشَجَّعِي وَلَا تَخَافِي} - Encourage yourself and don't fear [taʃaʤʤaʕiː walaː taxaːfiː]
}} \\
\midrule
\textbf{Synonyms} & \textarabic{تَجَلَّدَ} (showed endurance), \textarabic{تَقَوَّى} (strengthened himself), \textarabic{تَمَاسَكَ} (composed himself) \\
\textbf{Etymology} & From root meaning "boldness, courage" - related to Hebrew \texthebrew{שגע} (madness/boldness) \\
\bottomrule
\end{tabular}

\subsubsection*{Form V Conjugation}
\large \textbf{Verbal Noun}: \textarabic{تَشَجُّع} /taʃaʤʤuʕ/

\begin{longtable}{|>{\raggedright}p{3.5cm}|p{5cm}|p{5cm}|}
\hline
\textbf{Person} & \textbf{Perfect (Past)} & \textbf{Imperfect (Present)} \\
\hline
\textbf{3rd person masculine singular} & \textarabic{تَشَجَّعَ} /taʃaʤʤaʕa/ & \textarabic{يَتَشَجَّعُ} /jataʃaʤʤaʕu/ \\
\hline
\textbf{3rd person feminine singular} & \textarabic{تَشَجَّعَتْ} /taʃaʤʤaʕat/ & \textarabic{تَتَشَجَّعُ} /tataʃaʤʤaʕu/ \\
\hline
\textbf{3rd person masculine dual} & \textarabic{تَشَجَّعَا} /taʃaʤʤaʕaː/ & \textarabic{يَتَشَجَّعَانِ} /jataʃaʤʤaʕaːni/ \\
\hline
\textbf{3rd person feminine dual} & \textarabic{تَشَجَّعَتَا} /taʃaʤʤaʕataː/ & \textarabic{تَتَشَجَّعَانِ} /tataʃaʤʤaʕaːni/ \\
\hline
\textbf{3rd person masculine plural} & \textarabic{تَشَجَّعُوا} /taʃaʤʤaʕuː/ & \textarabic{يَتَشَجَّعُونَ} /jataʃaʤʤaʕuːna/ \\
\hline
\textbf{3rd person feminine plural} & \textarabic{تَشَجَّعْنَ} /taʃaʤʤaʕna/ & \textarabic{يَتَشَجَّعْنَ} /jataʃaʤʤaʕna/ \\
\hline
\textbf{2nd person masculine singular} & \textarabic{تَشَجَّعْتَ} /taʃaʤʤaʕta/ & \textarabic{تَتَشَجَّعُ} /tataʃaʤʤaʕu/ \\
\hline
\textbf{2nd person feminine singular} & \textarabic{تَشَجَّعْتِ} /taʃaʤʤaʕti/ & \textarabic{تَتَشَجَّعِينَ} /tataʃaʤʤaʕiːna/ \\
\hline
\textbf{2nd person dual (m./f.)} & \textarabic{تَشَجَّعْتُمَا} /taʃaʤʤaʕtumaː/ & \textarabic{تَتَشَجَّعَانِ} /tataʃaʤʤaʕaːni/ \\
\hline
\textbf{2nd person masculine plural} & \textarabic{تَشَجَّعْتُمْ} /taʃaʤʤaʕtum/ & \textarabic{تَتَشَجَّعُونَ} /tataʃaʤʤaʕuːna/ \\
\hline
\textbf{2nd person feminine plural} & \textarabic{تَشَجَّعْتُنَّ} /taʃaʤʤaʕtunna/ & \textarabic{تَتَشَجَّعْنَ} /tataʃaʤʤaʕna/ \\
\hline
\textbf{1st person singular} & \textarabic{تَشَجَّعْتُ} /taʃaʤʤaʕtu/ & \textarabic{أَتَشَجَّعُ} /ʔataʃaʤʤaʕu/ \\
\hline
\textbf{1st person plural} & \textarabic{تَشَجَّعْنَا} /taʃaʤʤaʕnaː/ & \textarabic{نَتَشَجَّعُ} /nataʃaʤʤaʕu/ \\
\hline
\end{longtable}

\subsubsection*{Additional Form V Verb Forms}
\begin{itemize}
  \item \textbf{Future}: \textarabic{سَيَتَشَجَّعُ} [sajataʃaʤʤaʕu] — he will encourage himself
  \item \textbf{Subjunctive}: \textarabic{أَنْ يَتَشَجَّعَ} [ʔan jataʃaʤʤaʕa] — that he encourage himself
  \item \textbf{Jussive}: \textarabic{لَمْ يَتَشَجَّعْ} [lam jataʃaʤʤaʕ] — he did not encourage himself
  \item \textbf{Imperative}: \textarabic{تَشَجَّعْ} [taʃaʤʤaʕ] — encourage yourself! (masc. sing.)
  \item \textbf{Active Participle}: \textarabic{مُتَشَجِّع} [mutaʃaʤʤiʕ] — one who encourages himself
  \item \textbf{Passive Participle}: \textarabic{مُتَشَجَّع} [mutaʃaʤʤaʕ] — encouraged
\end{itemize}

\subsection{\textarabic{وَقَالَ} — [waqaːla]}

\begin{tabular}{p{3cm}p{10cm}}
\toprule
\textbf{Translation} & and he said \\
\textbf{Root} & \textarabic{ق-و-ل} (q-w-l) \\
\textbf{Pattern} & \textarabic{فَعَلَ} (faʕala) with conjunction \textarabic{وَ} \\
\textbf{Grammar} & Past tense verb, 3rd person masculine singular, with conjunction \\
\textbf{Examples} & \makecell[l]{\parbox{9.5cm}{
1. \textarabic{وَقَالَ الْأُسْتَاذُ لِلطُّلَّابِ} - And the professor said to the students [waqaːla l-ʔustaːðu li-tˤ-tˤullaːb]\\
2. \textarabic{وَقَالَتِ الْأُمُّ لِابْنِهَا} - And the mother said to her son [waqaːlati l-ʔummu li-bnihaː]\\
3. \textarabic{وَقَالُوا جَمِيعًا} - And they all said [waqaːluː ʤamiːʕan]
}} \\
\midrule
\textbf{Synonyms} & \textarabic{وَتَكَلَّمَ} (and spoke), \textarabic{وَنَطَقَ} (and uttered), \textarabic{وَتَحَدَّثَ} (and talked) \\
\textbf{Etymology} & From Proto-Semitic *qwl, related to Hebrew \texthebrew{קול} (qol) "voice" \\
\bottomrule
\end{tabular}

\subsection{\textarabic{مَاذَا} — [maːðaː]}

\begin{tabular}{p{3cm}p{10cm}}
\toprule
\textbf{Translation} & what \\
\textbf{Root} & \textarabic{م-ا} + \textarabic{ذ-ا} (composite interrogative) \\
\textbf{Pattern} & \textarabic{مَا} + \textarabic{ذَا} (what + this/that) \\
\textbf{Grammar} & Interrogative pronoun, accusative (object of verb) \\
\textbf{Examples} & \makecell[l]{\parbox{9.5cm}{
1. \textarabic{مَاذَا تُرِيدُ مِنِّي؟} - What do you want from me? [maːðaː turiːdu minniiː]\\
2. \textarabic{مَاذَا حَدَثَ هُنَا؟} - What happened here? [maːðaː ħadaða hunaː]\\
3. \textarabic{مَاذَا قُلْتَ لَهُ؟} - What did you say to him? [maːðaː qulta lahu]
}} \\
\midrule
\textbf{Synonyms} & \textarabic{مَا} (what), \textarabic{أَيُّ شَيْءٍ} (which thing), \textarabic{أَيْشٍ} (dialectal: what) \\
\textbf{Etymology} & Compound of \textarabic{مَا} "what" + \textarabic{ذَا} "this" \\
\bottomrule
\end{tabular}

\subsection{\textarabic{تَفْعَلُ} — [tafʕalu]}

\begin{tabular}{p{3cm}p{10cm}}
\toprule
\textbf{Translation} & you do / you are doing \\
\textbf{Root} & \textarabic{ف-ع-ل} (f-ʕ-l) \\
\textbf{Pattern} & \textarabic{تَفْعَلُ} (tafʕalu) - present tense, 2nd person \\
\textbf{Grammar} & Present tense verb, 2nd person masculine singular, indicative mood \\
\textbf{Examples} & \makecell[l]{\parbox{9.5cm}{
1. \textarabic{تَفْعَلُ الْخَيْرَ دَائِمًا} - You always do good [tafʕalu l-xajra daːʔiman]\\
2. \textarabic{مَاذَا فَعَلْتَ بِالْكِتَابِ؟} - What did you do with the book? [maːðaː faʕalta bi-l-kitaːb]\\
3. \textarabic{سَتَفْعَلُ هَذَا غَدًا} - You will do this tomorrow [satafʕalu haːðaː ɣadan]
}} \\
\midrule
\textbf{Synonyms} & \textarabic{تَعْمَلُ} (you work/do), \textarabic{تَقُومُ بِ} (you perform), \textarabic{تُنْجِزُ} (you accomplish) \\
\textbf{Etymology} & From root meaning "to do, make, act" - fundamental action verb \\
\bottomrule
\end{tabular}

\subsection{\textarabic{هُنَا} — [hunaː]}

\begin{tabular}{p{3cm}p{10cm}}
\toprule
\textbf{Translation} & here \\
\textbf{Root} & \textarabic{ه-ن-ا} (demonstrative root) \\
\textbf{Pattern} & \textarabic{هُنَا} (adverb of place) \\
\textbf{Grammar} & Adverb of place, indeclinable \\
\textbf{Examples} & \makecell[l]{\parbox{9.5cm}{
1. \textarabic{تَعَالَ إِلَى هُنَا} - Come here [taʕaːla ʔilaː hunaː]\\
2. \textarabic{هُنَا بَيْتُنَا} - Here is our house [hunaː bajtuna]\\
3. \textarabic{مِنْ هُنَا إِلَى هُنَاكَ} - From here to there [min hunaː ʔilaː hunaːka]
}} \\
\midrule
\textbf{Synonyms} & \textarabic{هَهُنَا} (right here), \textarabic{فِي هَذَا الْمَكَانِ} (in this place) \\
\textbf{Etymology} & From demonstrative root, related to Hebrew \texthebrew{הנה} (hinneh) "behold, here" \\
\bottomrule
\end{tabular}

\subsubsection*{Related Demonstrative Adverbs}

\begin{tabular}{|c|c|c|c|}
\hline
\textbf{Adverb} & \textbf{Meaning} & \textbf{Distance} & \textbf{Usage} \\
\hline
\textarabic{هُنَا} /hunaː/ & here & Close to speaker & \textarabic{أَنَا هُنَا} - I am here \\
\hline
\textarabic{هُنَاكَ} /hunaːka/ & there & Away from speaker & \textarabic{هُوَ هُنَاكَ} - He is there \\
\hline
\textarabic{هُنَالِكَ} /hunaːlika/ & over there & Far from both & \textarabic{الْمَدْرَسَةُ هُنَالِكَ} - The school is over there \\
\hline
\textarabic{هَهُنَا} /hahunaː/ & right here & Emphatic close & \textarabic{هَهُنَا الْمُشْكِلَةُ} - Right here is the problem \\
\hline
\textarabic{ثَمَّ} /θamma/ & there & General location & \textarabic{ذَهَبَ إِلَى ثَمَّ} - He went there \\
\hline
\end{tabular}

\subsubsection*{Grammatical Notes}
\begin{itemize}
\item \textarabic{هُنَا} is \textbf{indeclinable} - it never changes form regardless of grammatical case
\item Functions as an \textbf{adverb of place} (ظَرْف مَكَان)
\item Can be preceded by prepositions: \textarabic{مِنْ هُنَا} (from here), \textarabic{إِلَى هُنَا} (to here)
\item Often used with demonstrative pronouns: \textarabic{هَذَا هُنَا} (this one here)
\item In classical poetry, may appear as \textarabic{هُنَا} or \textarabic{هَا هُنَا} for metrical purposes
\end{itemize}

\subsection{\textarabic{يَا شَيْخ} — [jaː ʃajx]}

\begin{tabular}{p{3cm}p{10cm}}
\toprule
\textbf{Translation} & O old man / O sheikh \\
\textbf{Root} & \textarabic{ي-ا} (vocative) + \textarabic{ش-ي-خ} (ʃ-j-x) \\
\textbf{Pattern} & \textarabic{يَا} + \textarabic{فَيْعَل} (vocative + pattern) \\
\textbf{Grammar} & Vocative particle + noun in vocative case (addressing someone) \\
\textbf{Examples} & \makecell[l]{\parbox{9.5cm}{
1. \textarabic{يَا شَيْخَ مُحَمَّدٍ} - O Sheikh Muhammad [jaː ʃajxa muħammad]\\
2. \textarabic{أَهْلًا وَسَهْلًا يَا شَيْخ} - Welcome, O sheikh [ʔahlan wasahlan jaː ʃajx]\\
3. \textarabic{يَا شَيْخُ قُلْ لَنَا} - O sheikh, tell us [jaː ʃajxu qul lanaː]
}} \\
\midrule
\textbf{Synonyms} & \textarabic{يَا عَجُوز} (O old man), \textarabic{يَا كَبِير} (O elder), \textarabic{يَا أُسْتَاذ} (O teacher) \\
\textbf{Etymology} & From root meaning "to age, become old" - related to Hebrew \texthebrew{זקן} (zaqen) "elder" \\
\bottomrule
\end{tabular}

% ======================== Phrase Analysis ========================
\section{Phrase Analysis}
\begin{tcolorbox}[colback=boxcolor,colframe=headercolor,breakable]
\textbf{Grammatical Structure:}\\
Sequential connector + past verb + definite subject + adversative conjunction with pronoun + reflexive past verb + coordinating conjunction + past verb + colon + interrogative pronoun + present verb + adverb + vocative + addressed noun \\

\textbf{Key Grammar Points:}
\begin{itemize}
\item The prefix \textarabic{فَ} creates narrative sequence from previous events
\item \textarabic{اللِّصُّ} is definite, indicating this thief was previously mentioned
\item \textarabic{وَلَكِنَّهُ} shows strong contrast with attached pronoun reference
\item \textarabic{تَشَجَّعَ} is Form V, indicating reflexive/self-directed action
\item Direct speech introduced by colon after \textarabic{قَالَ}
\item \textarabic{مَاذَا تَفْعَلُ} forms complete interrogative clause
\item \textarabic{هُنَا} provides spatial context for the question
\item \textarabic{يَا شَيْخ} is vocative address showing both respect and possibly condescension
\item The sequence shows psychological progression: fear → courage → confrontation
\end{itemize}
\end{tcolorbox}

% ======================== Similar Phrases for Practice ========================
\section{Similar Phrases for Practice}

\begin{enumerate}
\item \textarabic{فَخَافَ الطِّفْلُ وَلَكِنَّهُ تَشَجَّعَ وَقَالَ: مَنْ أَنْتَ يَا رَجُل؟}\\
So the child became afraid, but he gathered courage and said: "Who are you, man?" [faxaːfa tˤ-tˤiflu walaːkinnahuː taʃaʤʤaʕa waqaːla: man ʔanta jaː raʤul]

\item \textarabic{فَارْتَبَكَتِ الْفَتَاةُ وَلَكِنَّهَا تَمَالَكَتْ نَفْسَهَا وَسَأَلَتْ: لِمَاذَا جِئْتَ هُنَا يَا غَرِيب؟}\\
So the girl became confused, but she composed herself and asked: "Why did you come here, stranger?" [fartabakat l-fataːtu walaːkinnahaː tamaalakat nafsahaː wasaʔalat: limaːðaː ʤiʔta hunaː jaː ɣariːb]

\item \textarabic{فَفَزِعَ الْحَارِسُ وَلَكِنَّهُ اسْتَجْمَعَ شَجَاعَتَهُ وَصَرَخَ: مَاذَا تُرِيدُ مِنِّي يَا لِصّ؟}\\
So the guard was frightened, but he gathered his courage and shouted: "What do you want from me, thief?" [fafaziʕa l-ħaːrisu walaːkinnahuː staʤmaʕa ʃaʤaːʕatahu wasaraxa: maːðaː turiːdu minniiː jaː lisˤsˤ]

\item \textarabic{فَقَلِقَ الأَبُ وَلَكِنَّهُ هَدَّأَ نَفْسَهُ وَسَأَلَ: أَيْنَ كُنْتَ يَا وَلَدِي؟}\\
So the father became worried, but he calmed himself and asked: "Where were you, my son?" [faqaliqa l-ʔabu walaːkinnahuː haddaʔa nafsahu wasaʔala: ʔajna kunta jaː waladiː]
\end{enumerate}

% ======================== Levantine Dialect Version ========================
\section{Levantine (Shaami) Arabic Dialect}

\begin{tcolorbox}[colback=white,colframe=dialectcolor,title=\textbf{Levantine Version},breakable]
\textarabic{فَخَافِ الحَرَامِي بَسّ تَشَجَّعْ وْقَالّ: شُو عَمْ تِعْمَلْ هُونْ يَا عَجُوز؟}\\
\textbf{Phonetic:} [faxaːf il-ħaraːmiː bass taʃaʤʤaʕ wiqaːl: ʃuː ʕam tiʕmal hoːn jaː ʕaʤuːz]\\
\textbf{Translation:} So the thief got scared but he got brave and said: "What are you doing here, old man?"
\end{tcolorbox}

\textbf{Key Dialectal Changes:}
\begin{itemize}
\item \textarabic{خَافَ} → \textarabic{خَافِ} (xaːf) — dropped final vowel, typical in Levantine
\item \textarabic{اللِّصُّ} → \textarabic{الحَرَامِي} (il-ħaraːmiː) — dialectal word for "thief"
\item \textarabic{وَلَكِنَّهُ} → \textarabic{بَسّ} (bass) — simplified "but"
\item \textarabic{تَشَجَّعَ} → \textarabic{تَشَجَّعْ} (taʃaʤʤaʕ) — dropped final vowel
\item \textarabic{وَقَالَ} → \textarabic{وْقَالّ} (wiqaːl) — consonant cluster and gemination
\item \textarabic{مَاذَا} → \textarabic{شُو} (ʃuː) — dialectal interrogative "what"
\item \textarabic{تَفْعَلُ} → \textarabic{عَمْ تِعْمَلْ} (ʕam tiʕmal) — progressive aspect with \textarabic{عَمْ}
\item \textarabic{هُنَا} → \textarabic{هُونْ} (hoːn) — dialectal form "here"
\item \textarabic{شَيْخ} → \textarabic{عَجُوز} (ʕaʤuːz) — more colloquial "old man"
\end{itemize}

% ======================== Additional Learning Notes ========================
\section{Additional Learning Notes}

\begin{tcolorbox}[colback=boxcolor,colframe=accentcolor,title=\textbf{Cultural and Literary Context},breakable]
\textbf{Narrative Technique:} This phrase demonstrates classic Arabic storytelling progression from internal state (fear) through psychological transformation (courage) to external action (confrontation).

\textbf{Character Development:} The thief's journey from \textarabic{خَافَ} to \textarabic{تَشَجَّعَ} shows internal conflict resolution, making him more complex than a simple antagonist.

\textbf{Social Dynamics:} The use of \textarabic{يَا شَيْخ} reveals social hierarchy awareness — even in confrontation, there's acknowledgment of the elder's status.

\textbf{Psychological Realism:} The sequence captures authentic human response: initial fear, self-encouragement, then aggressive questioning as defense mechanism.
\end{tcolorbox}

\subsection{Memory Tips}
\begin{enumerate}
\item \textbf{Emotional Arc:} Remember \textit{Fear → Courage → Confrontation} (\textarabic{خَوْف} → \textarabic{شَجَاعَة} → \textarabic{مُوَاجَهَة})
\item \textbf{Conjunction Chain:} \textarabic{فَ...وَلَكِنَّ...وَ} creates logical narrative flow (so → but → and)
\item \textbf{Form V Pattern:} \textarabic{تَشَجَّعَ} follows \textarabic{تَفَعَّلَ} pattern for reflexive/intensive actions
\item \textbf{Interrogative Structure:} \textarabic{مَاذَا + تَفْعَلُ + هُنَا} is standard question formation
\item \textbf{Vocative Formula:} \textarabic{يَا + noun} is universal Arabic addressing pattern
\end{enumerate}

\subsection{Related Vocabulary Family}
\begin{multicols}{2}

\textbf{Emotion family:}\\
\textarabic{خَوْف} — fear [xawf]\\
\textarabic{شَجَاعَة} — courage [ʃaʤaːʕa]\\
\textarabic{جُبْن} — cowardice [ʤubn]\\
\textarabic{قَلَق} — anxiety [qalaq]\\
\textarabic{رُعْب} — terror [ruʕb]\\

\textbf{Criminal family:}\\
\textarabic{سَارِق} — thief [saːriq]\\
\textarabic{لُصُوص} — thieves [lusˤuːsˤ]\\
\textarabic{مُجْرِم} — criminal [muʤrim]\\
\textarabic{قَاطِع طَرِيق} — robber [qaːtiʕ tˤariːq]\\
\textarabic{نَشَّال} — pickpocket [naʃʃaːl]\\

\textbf{Age/respect family:}\\
\textarabic{شَيْخ} — elder/sheikh [ʃajx]\\
\textarabic{عَجُوز} — old person [ʕaʤuːz]\\
\textarabic{كَبِير} — elder [kabiːr]\\
\textarabic{مُسِنّ} — elderly [musinn]\\
\textarabic{كَهْل} — middle-aged [kahl]\\

\textbf{Action family:}\\
\textarabic{فِعْل} — action/deed [fiʕl]\\
\textarabic{عَمَل} — work/action [ʕamal]\\
\textarabic{نَشَاط} — activity [naʃaːtˤ]\\
\textarabic{تَصَرُّف} — behavior [tasˤarruf]\\
\textarabic{إِنْجَاز} — accomplishment [ʔinʤaːz]\\

\textbf{Question family:}\\
\textarabic{مَاذَا} — what [maːðaː]\\
\textarabic{لِمَاذَا} — why [limaːðaː]\\
\textarabic{مَتَى} — when [mataː]\\
\textarabic{أَيْنَ} — where [ʔajna]\\
\textarabic{كَيْفَ} — how [kajfa]\\
\textarabic{مَنْ} — who [man]\\
\end{multicols}

\subsection{Advanced Grammar Notes}

\begin{tcolorbox}[colback=white,colframe=headercolor,title=\textbf{Syntactic Analysis},breakable]
\textbf{Sentence Structure:}
\begin{enumerate}
\item \textbf{Main Clause 1:} \textarabic{فَخَافَ اللِّصُّ} [Subject-Verb, perfect aspect]
\item \textbf{Adversative Clause:} \textarabic{وَلَكِنَّهُ تَشَجَّعَ} [Contrast marker + pronoun + verb]
\item \textbf{Main Clause 2:} \textarabic{وَقَالَ} [Coordinated verb]
\item \textbf{Direct Speech:} \textarabic{مَاذَا تَفْعَلُ هُنَا يَا شَيْخ؟} [Interrogative clause + vocative]
\end{enumerate}

\textbf{Narrative Functions:}
\begin{itemize}
\item \textarabic{فَ} — Sequential connector (narrative progression)
\item \textarabic{وَلَكِنَّ} — Adversative (psychological turning point)
\item \textarabic{وَ} — Additive (action continuation)
\item Colon — Direct speech marker (dialogue introduction)
\end{itemize}
\end{tcolorbox}

\subsection{Morphological Analysis}

\begin{tabular}{|c|c|c|c|c|}
\hline
\textbf{Word} & \textbf{Root} & \textbf{Form/Pattern} & \textbf{Function} & \textbf{Features} \\
\hline
\textarabic{فَخَافَ} & خ-و-ف & Form I + prefix & Verb & Past, 3ms, narrative \\
\hline
\textarabic{اللِّصُّ} & ل-ص-ص & فِعْل + definite & Noun & Subject, definite \\
\hline
\textarabic{وَلَكِنَّهُ} & ل-ك-ن & Particle + pronoun & Conjunction & Adversative + 3ms \\
\hline
\textarabic{تَشَجَّعَ} & ش-ج-ع & Form V & Verb & Reflexive, past, 3ms \\
\hline
\textarabic{وَقَالَ} & ق-و-ل & Form I + prefix & Verb & Past, 3ms, coordinated \\
\hline
\textarabic{مَاذَا} & م-ا + ذ-ا & Interrogative & Pronoun & Object, accusative \\
\hline
\textarabic{تَفْعَلُ} & ف-ع-ل & Form I imperfect & Verb & Present, 2ms, indicative \\
\hline
\textarabic{هُنَا} & Demonstrative & Adverb & Location & Indeclinable \\
\hline
\textarabic{يَا شَيْخ} & Vocative + ش-ي-خ & فَيْعَل & Address & Vocative case \\
\hline
\end{tabular}

\section{Text Analysis Summary}

This phrase exemplifies sophisticated Arabic narrative technique, combining psychological realism with grammatical complexity. The progression from fear through self-encouragement to confrontation creates a complete character arc within a single sentence, demonstrating how Arabic can pack dense emotional and narrative content into relatively few words.

The grammatical structures employed—sequential connectors, adversative conjunctions, reflexive verbs, and direct speech markers—work together to create a vivid scene that advances both plot and character development simultaneously.

\end{document}