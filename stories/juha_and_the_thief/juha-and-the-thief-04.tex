\documentclass[letterpaper,12pt]{article}


\usepackage{fancyhdr}
\usepackage[most]{tcolorbox}
\usepackage{longtable}
\usepackage{makecell}
\usepackage{tikz}
\usepackage{geometry}
\usepackage{array}
\usepackage{multicol}
\usepackage[table]{xcolor}
\usepackage{polyglossia}
\setmainlanguage{english}
\setotherlanguage{arabic}
\setotherlanguage{hebrew}
\usepackage{fontspec}
\setmainfont{Charis SIL}
\newfontfamily\arabicfont[Script=Arabic,Scale=1.2]{Amiri}
\newfontfamily\hebrewfont{Ezra SIL}[Script=Hebrew]
\geometry{margin=1.5cm}
\usepackage{booktabs}
\usepackage{graphicx}
\usepackage{multirow}

\setlength{\headheight}{14.49998pt}
\addtolength{\topmargin}{-2.49998pt}
\pagestyle{fancy}
\fancyhf{}
\fancyfoot[C]{\thepage}
\fancyhead[R]{\textit{Juha's Apology to the Thief}}
\renewcommand{\headrulewidth}{0pt} % Remove header line

\usetikzlibrary{shapes,arrows,decorations.pathmorphing}


% Custom colors
\definecolor{headercolor}{RGB}{70,130,180}
\definecolor{boxcolor}{RGB}{240,248,255}
\definecolor{accentcolor}{RGB}{220,20,60}
\definecolor{tableheader}{RGB}{220,220,220}
\definecolor{dialectcolor}{RGB}{34,139,34}

\begin{document}


\title{\textbf{\Large Arabic Phrase Analyser}\\
\large Juha's Apology to the Thief\\
\normalsize \textarabic{فقال جحا: متأسف يا سيدي, فإني أعرف أنك لن تجد ما تسرقه, و لهذا استحيت خجلا منك}}
\author{Igor Deruga}
\date{}
\maketitle

% ======================== Phrase Display ========================
\begin{tcolorbox}[colback=boxcolor,colframe=headercolor,title=\textbf{Arabic Phrase with Full Diacritics},breakable]
\centering
\textarabic{فَقَالَ جُحَا: مُتَأَسِّفٌ يَا سَيِّدِي، فَإِنِّي أَعْرِفُ أَنَّكَ لَن تَجِدَ مَا تَسْرِقُهُ، وَلِهَذَا اسْتَحْيَيْتُ خَجَلاً مِنكَ}
\end{tcolorbox}

% ======================== Translations ========================
\section{English Translation}
\begin{tcolorbox}[colback=white,colframe=accentcolor,breakable]
\textbf{Literal:} So-said Juha: apologetic oh master-my, for-indeed-I know that-you never will-find what steal-it, and-for-this felt-ashamed shamefully from-you \\
\textit{[Arabic order retained for direct mapping]}\\[0.5em]
\textbf{Adapted:} So Juha said: "Sorry, sir, for I know that you won't find anything to steal, and that's why I felt ashamed for you."
\end{tcolorbox}

% ======================== Detailed Word Analysis ========================
\section{Detailed Word Analysis}

\subsection{\textarabic{فَقَالَ} — [faqaːla]}
\begin{tabular}{p{3cm}p{10cm}}
\toprule
\textbf{Translation} & so he said / then he said \\
\textbf{Root} & \textarabic{ق-و-ل} (q-w-l) \\
\textbf{Pattern} & \textarabic{فَعَلَ} (faʕala) \\
\textbf{Grammar} & Past tense verb, 3rd person masculine singular, with conjunction \textarabic{فَ} (fa-) prefix \\
\midrule \\
\textbf{Examples} & \makecell[l]{\parbox{9.5cm}{
1. \textarabic{قَالَ الرَّجُلُ الحَقَّ} - The man said the truth [qaːla r-radʒulu l-ħaqqa]\\
2. \textarabic{سَيَقُولُ لَكَ غَداً} - He will tell you tomorrow [sajaquːlu laka ɣadan]\\
3. \textarabic{قُلْ لِي مَاذَا حَدَثَ} - Tell me what happened [qul liː maːðaː ħadaθa]
} } \\
\midrule \\
\textarabic{تَكَلَّمَ} (spoke), \textarabic{تَحَدَّثَ} (talked), \textarabic{نَطَقَ} (uttered) \\
\textbf{Etymology} & From Proto-Semitic *qwl, related to Hebrew \texthebrew{קול} (qol) ``voice'' \\
\bottomrule
\end{tabular}

\subsection{Conjugation}
\par{ \large \textbf{Verbal Noun}: \textarabic{قَوْل} /qawl/}

\begin{longtable}{|>{\raggedright}p{3.5cm}|p{5cm}|p{5cm}|}
\hline
\textbf{Person} & \textbf{Perfect (Past)} & \textbf{Imperfect (Present)} \\
\hline
\textbf{3rd person masculine singular} & \textarabic{قَالَ} /qaːla/ & \textarabic{يَقُولُ} /jaquːlu/ \\
\hline
\textbf{3rd person feminine singular} & \textarabic{قَالَتْ} /qaːlat/ & \textarabic{تَقُولُ} /taquːlu/ \\
\hline
\textbf{3rd person masculine dual} & \textarabic{قَالاَ} /qaːlaː/ & \textarabic{يَقُولاَنِ} /jaquːlaːni/ \\
\hline
\textbf{3rd person feminine dual} & \textarabic{قَالَتَا} /qaːlataː/ & \textarabic{تَقُولاَنِ} /taquːlaːni/ \\
\hline
\textbf{3rd person masculine plural} & \textarabic{قَالُوا} /qaːluː/ & \textarabic{يَقُولُونَ} /jaquːluːna/ \\
\hline
\textbf{3rd person feminine plural} & \textarabic{قُلْنَ} /qulna/ & \textarabic{يَقُلْنَ} /jaqulna/ \\
\hline
\textbf{2nd person masculine singular} & \textarabic{قُلْتَ} /qulta/ & \textarabic{تَقُولُ} /taquːlu/ \\
\hline
\textbf{2nd person feminine singular} & \textarabic{قُلْتِ} /qulti/ & \textarabic{تَقُولِينَ} /taquːliːna/ \\
\hline
\textbf{2nd person dual (m./f.)} & \textarabic{قُلْتُمَا} /qultumaː/ & \textarabic{تَقُولاَنِ} /taquːlaːni/ \\
\hline
\textbf{2nd person masculine plural} & \textarabic{قُلْتُمْ} /qultum/ & \textarabic{تَقُولُونَ} /taquːluːna/ \\
\hline
\textbf{2nd person feminine plural} & \textarabic{قُلْتُنَّ} /qultunna/ & \textarabic{تَقُلْنَ} /taqulna/ \\
\hline
\textbf{1st person singular} & \textarabic{قُلْتُ} /qultu/ & \textarabic{أَقُولُ} /ʔaquːlu/ \\
\hline
\textbf{1st person plural} & \textarabic{قُلْنَا} /qulnaː/ & \textarabic{نَقُولُ} /naquːlu/ \\
\hline
\end{longtable}

\subsubsection*{Conjugation Notes}
\begin{itemize}
  \item The \textbf{future} is formed with the prefix \textarabic{سَ} [sa-] or \textarabic{سَوْفَ} [sawfa] before the imperfect (e.g., \textarabic{سَيَقُولُ} "he will say").
  \item The \textbf{moods of the imperfect}: 
    \begin{itemize}
      \item Indicative: \textarabic{يَقُولُ} [yaqūlu] 
      \item Subjunctive: \textarabic{لَنْ يَقُولَ} [lan yaqūla]
      \item Jussive: \textarabic{لَمْ يَقُلْ} [lam yaqul]
      \item Imperative: \textarabic{قُلْ} [qul!]
    \end{itemize}
  \item \textbf{Passive voice}: 
    \begin{itemize}
      \item Perfect: \textarabic{قِيلَ} [qīla] — it was said
      \item Imperfect: \textarabic{يُقَالُ} [yuqālu] — it is said
    \end{itemize}
\end{itemize}

\subsection{\textarabic{جُحَا} — [dʒuħaː]}

\begin{tabular}{p{3cm}p{10cm}}
\toprule
\textbf{Translation} & Juha (proper name of folk character) \\
\textbf{Root} & Proper noun, no trilateral root \\
\textbf{Pattern} & \textarabic{فُعَل} (fuʕal) pattern for names \\
\textbf{Grammar} & Proper noun, masculine, nominative case (subject), indeclinable \\
\midrule \\
\textbf{Examples} & \makecell[l]{\parbox{9.5cm}{
1. \textarabic{جُحَا رَجُلٌ حَكِيمٌ} - Juha is a wise man [dʒuħaː radʒulun ħakiːmun]\\
2. \textarabic{قِصَصُ جُحَا مُسَلِّيَةٌ} - Juha's stories are entertaining [qiṣaṣu dʒuħaː musallija]\\
3. \textarabic{يُحِبُّ النَّاسُ جُحَا} - People love Juha [juħibbu n-naːsu dʒuħaː]
}} \\
\midrule \\
\textbf{Synonyms} & \textarabic{نَصْر الدِّين} (Nasreddin), \textarabic{أَبُو نُوَاس} (Abu Nuwas - different character) \\
\textbf{Etymology} & Possibly from Aramaic, meaning "the one who makes people laugh" \\
\bottomrule
\end{tabular}

\subsection{\textarabic{مُتَأَسِّفٌ} — [mutaʔassifun]}

\begin{tabular}{p{3cm}p{10cm}}
\toprule
\textbf{Translation} & sorry / apologetic / regretful \\
\textbf{Root} & \textarabic{أ-س-ف} (ʔ-s-f) \\
\textbf{Pattern} & \textarabic{مُتَفَعِّل} (mutafaʕʕil) - Form V active participle \\
\textbf{Grammar} & Active participle, masculine singular, nominative, functioning as predicate \\
\midrule \\
\textbf{Examples} & \makecell[l]{\parbox{9.5cm}{
1. \textarabic{أَنَا مُتَأَسِّفٌ لِلتَّأْخِير} - I am sorry for the delay [ʔanaː mutaʔassifun lit-taʔxiːr]\\
2. \textarabic{هِيَ مُتَأَسِّفَةٌ جِدّاً} - She is very sorry [hija mutaʔassifatun dʒiddan]\\
3. \textarabic{كُنَّا مُتَأَسِّفِينَ} - We were sorry [kunnaː mutaʔassifiːn]
}} \\
\midrule \\
\textbf{Synonyms} & \textarabic{آسِف} (sorry), \textarabic{نَادِم} (regretful), \textarabic{مُعْتَذِر} (apologetic) \\
\textbf{Etymology} & From root meaning "to grieve, to feel sorrow" \\
\bottomrule
\end{tabular}

\subsection{\textarabic{يَا سَيِّدِي} — [jaː sajjidiː]}

\begin{tabular}{p{3cm}p{10cm}}
\toprule
\textbf{Translation} & oh my master / sir \\
\textbf{Root} & \textarabic{س-و-د} (s-w-d) \\
\textbf{Pattern} & \textarabic{فَعِّل} (faʕʕil) with 1st person possessive suffix \\
\textbf{Grammar} & Vocative particle + noun, masculine, genitive case, with possessive pronoun \textarabic{ـي} \\
\midrule \\
\textbf{Examples} & \makecell[l]{\parbox{9.5cm}{
1. \textarabic{هُوَ سَيِّدُ القَوْمِ} - He is the master of the people [huwa sajjidu l-qawmi]\\
2. \textarabic{أَطَعْتُ سَيِّدِي} - I obeyed my master [ʔatˤaʕtu sajjidiː]\\
3. \textarabic{السَّيِّدُ مُحَمَّد} - Mr. Muhammad [as-sajjidu muħammad]
}} \\
\midrule \\
\textbf{Synonyms} & \textarabic{مَوْلَى} (master), \textarabic{رَبّ} (lord), \textarabic{صَاحِب} (owner) \\
\textbf{Etymology} & From root meaning "to be black, to prevail, to rule" \\
\bottomrule
\end{tabular}

\subsection{\textarabic{فَإِنِّي} — [faʔinniː]}

\begin{tabular}{p{3cm}p{10cm}}
\toprule
\textbf{Translation} & for indeed I / because I \\
\textbf{Root} & Particle compound: \textarabic{فَ} + \textarabic{إِنَّ} + pronoun \\
\textbf{Pattern} & Causal conjunction + emphatic particle + 1st person pronoun \\
\textbf{Grammar} & \textarabic{فَ} (causal) + \textarabic{إِنَّ} (emphatic "indeed") + \textarabic{ـي} (1st person suffix) \\
\midrule \\
\textbf{Examples} & \makecell[l]{\parbox{9.5cm}{
1. \textarabic{فَإِنِّي أُحِبُّكَ} - For indeed I love you [faʔinniː ʔuħibbuka]\\
2. \textarabic{فَإِنَّكَ مُحِقٌّ} - For indeed you are right [faʔinnaka muħiqqun]\\
3. \textarabic{فَإِنَّهُمْ غَائِبُونَ} - For indeed they are absent [faʔinnahum ɣaːʔibuːn]
}} \\
\midrule \\
\textbf{Synonyms} & \textarabic{لِأَنِّي} (because I), \textarabic{إِذْ إِنِّي} (since I) \\
\textbf{Etymology} & \textarabic{فَ} from Proto-Semitic conjunction, \textarabic{إِنَّ} emphatic particle \\
\bottomrule
\end{tabular}

\subsection{\textarabic{أَعْرِفُ} — [ʔaʕrifu]}

\begin{tabular}{p{3cm}p{10cm}}
\toprule
\textbf{Translation} & I know \\
\textbf{Root} & \textarabic{ع-ر-ف} (ʕ-r-f) \\
\textbf{Pattern} & \textarabic{أَفْعِلُ} (ʔafʕilu) - 1st person singular imperfect \\
\textbf{Grammar} & Present tense verb, 1st person singular, active voice \\
\midrule \\
\textbf{Examples} & \makecell[l]{\parbox{9.5cm}{
1. \textarabic{عَرَفْتُ الحَقِيقَةَ} - I knew the truth [ʕaraftu l-ħaqiːqa]\\
2. \textarabic{سَأَعْرِفُ غَداً} - I will know tomorrow [saʔaʕrifu ɣadan]\\
3. \textarabic{لَا أَعْرِفُ شَيْئاً} - I don't know anything [laː ʔaʕrifu ʃajʔan]
}} \\
\midrule \\
\textbf{Synonyms} & \textarabic{أَعْلَمُ} (I know), \textarabic{أَدْرِي} (I am aware), \textarabic{أُدْرِكُ} (I realize) \\
\textbf{Etymology} & From Proto-Semitic root meaning "to know, recognize" \\
\bottomrule
\end{tabular}

\subsection{\textarabic{أَنَّكَ} — [ʔannaka]}

\begin{tabular}{p{3cm}p{10cm}}
\toprule
\textbf{Translation} & that you \\
\textbf{Root} & \textarabic{أَنَّ} + pronoun suffix \\
\textbf{Pattern} & Subordinating particle + 2nd person masculine singular pronoun \\
\textbf{Grammar} & \textarabic{أَنَّ} (that/indeed) + \textarabic{ـكَ} (you, masc. sing.) \\
\midrule \\
\textbf{Examples} & \makecell[l]{\parbox{9.5cm}{
1. \textarabic{أَعْلَمُ أَنَّكَ مُحِقٌّ} - I know that you are right [ʔaʕlamu ʔannaka muħiqqun]\\
2. \textarabic{ظَنَنْتُ أَنَّكِ هُنَا} - I thought that you (fem.) were here [ðˤanantu ʔannaki hunaː]\\
3. \textarabic{قَالَ أَنَّهُمْ سَيَأْتُونَ} - He said that they will come [qaːla ʔannahum sajaʔtuːn]
}} \\
\midrule \\
\textbf{Synonyms} & \textarabic{بِأَنَّكَ} (that you), \textarabic{كَوْنَكَ} (your being) \\
\textbf{Etymology} & \textarabic{أَنَّ} from Proto-Semitic emphatic particle \\
\bottomrule
\end{tabular}

\subsection{\textarabic{لَن تَجِدَ} — [lan tadʒida]}

\begin{tabular}{p{3cm}p{10cm}}
\toprule
\textbf{Translation} & you will not find \\
\textbf{Root} & \textarabic{و-ج-د} (w-dʒ-d) \\
\textbf{Pattern} & \textarabic{لَنْ} + \textarabic{تَفْعِلَ} (subjunctive) \\
\textbf{Grammar} & Future negation particle + 2nd person masculine singular subjunctive \\
\midrule \\
\textbf{Examples} & \makecell[l]{\parbox{9.5cm}{
1. \textarabic{وَجَدْتُ الكِتَابَ} - I found the book [wadʒadtu l-kitaːb]\\
2. \textarabic{سَأَجِدُ الحَلَّ} - I will find the solution [saʔadʒidu l-ħall]\\
3. \textarabic{لَمْ نَجِدْ شَيْئاً} - We didn't find anything [lam nadʒid ʃajʔan]
}} \\
\midrule \\
\textbf{Synonyms} & \textarabic{يُلَاقِي} (encounters), \textarabic{يُصَادِف} (comes across), \textarabic{يَحْصُلُ عَلَى} (obtains) \\
\textbf{Etymology} & From Proto-Semitic root meaning "to find, encounter" \\
\bottomrule
\end{tabular}

\subsection{\textarabic{مَا تَسْرِقُهُ} — [maː tasriquhu]}

\begin{tabular}{p{3cm}p{10cm}}
\toprule
\textbf{Translation} & what you steal (it) \\
\textbf{Root} & \textarabic{س-ر-ق} (s-r-q) for the verb \\
\textbf{Pattern} & \textarabic{مَا} (relative) + \textarabic{تَفْعِلُهُ} (present + object pronoun) \\
\textbf{Grammar} & Relative pronoun + 2nd person present verb + 3rd person object pronoun \\
\midrule \\
\textbf{Examples} & \makecell[l]{\parbox{9.5cm}{
1. \textarabic{سَرَقَ اللِّصُّ المَالَ} - The thief stole the money [saraqa l-lˤisˤsˤu l-maːl]\\
2. \textarabic{لَا تَسْرِقْ} - Don't steal [laː tasriq]\\
3. \textarabic{السَّرِقَةُ حَرَامٌ} - Theft is forbidden [as-sariqa ħaraːmun]
}} \\
\midrule \\
\textbf{Synonyms} & \textarabic{يَنْهَبُ} (plunders), \textarabic{يَخْتَلِسُ} (embezzles), \textarabic{يَسْلُبُ} (robs) \\
\textbf{Etymology} & From Proto-Semitic root meaning "to steal, take secretly" \\
\bottomrule
\end{tabular}

\subsection{\textarabic{وَلِهَذَا} — [walihaːðaː]}

\begin{tabular}{p{3cm}p{10cm}}
\toprule
\textbf{Translation} & and for this reason / therefore \\
\textbf{Root} & \textarabic{وَ} + \textarabic{لِ} + \textarabic{هَذَا} \\
\textbf{Pattern} & Conjunction + preposition + demonstrative pronoun \\
\textbf{Grammar} & \textarabic{وَ} (and) + \textarabic{لِ} (for/because of) + \textarabic{هَذَا} (this) \\
\midrule \\
\textbf{Examples} & \makecell[l]{\parbox{9.5cm}{
1. \textarabic{وَلِهَذَا نَجَحَ} - And for this reason he succeeded [walihaːðaː nadʒaħa]\\
2. \textarabic{وَلِذَلِكَ رَفَضَ} - And therefore he refused [waliðaːlika rafadˤa]\\
3. \textarabic{وَلِهَذَا السَّبَبِ} - And for this reason [walihaːðaː s-sababi]
}} \\
\midrule \\
\textbf{Synonyms} & \textarabic{وَلِذَلِكَ} (therefore), \textarabic{وَلِهَذَا السَّبَب} (for this reason), \textarabic{وَعَلَى هَذَا} (accordingly) \\
\textbf{Etymology} & Compound of basic grammatical particles \\
\bottomrule
\end{tabular}

\subsection{\textarabic{اسْتَحْيَيْتُ} — [istaħjajtu]}

\begin{tabular}{p{3cm}p{10cm}}
\toprule
\textbf{Translation} & I felt ashamed / I became shy \\
\textbf{Root} & \textarabic{ح-ي-ي} (ħ-j-j) \\
\textbf{Pattern} & \textarabic{اسْتَفْعَلْتُ} (istafʕaltu) - Form X, 1st person past \\
\textbf{Grammar} & Past tense verb, 1st person singular, Form X (reflexive/seeking state) \\
\midrule \\
\textbf{Examples} & \makecell[l]{\parbox{9.5cm}{
1. \textarabic{اسْتَحْيَتْ مِنَ النَّاسِ} - She felt ashamed before people [istaħjat min an-naːsi]\\
2. \textarabic{لَا تَسْتَحْيِي مِنَ الحَقِّ} - Don't be ashamed of the truth [laː tastaħjiː min al-ħaqqi]\\
3. \textarabic{يَسْتَحْيِي مِنْ أَهْلِهِ} - He feels shy around his family [jastaħjiː min ʔahlihi]
}} \\
\midrule \\
\textbf{Synonyms} & \textarabic{خَجِلَ} (felt shame), \textarabic{احْمَرَّ وَجْهُهُ} (his face reddened), \textarabic{تَحَرَّجَ} (felt embarrassed) \\
\textbf{Etymology} & From root related to "life, modesty, shame" \\
\bottomrule
\end{tabular}

\subsection{\textarabic{خَجَلاً} — [xadʒalan]}

\begin{tabular}{p{3cm}p{10cm}}
\toprule
\textbf{Translation} & shamefully / out of shame \\
\textbf{Root} & \textarabic{خ-ج-ل} (x-dʒ-l) \\
\textbf{Pattern} & \textarabic{فَعَلاً} (faʕalan) - adverbial accusative \\
\textbf{Grammar} & Verbal noun in accusative case, functioning as adverb of manner/cause \\
\midrule \\
\textbf{Examples} & \makecell[l]{\parbox{9.5cm}{
1. \textarabic{خَجِلَ مِنْ فِعْلَتِهِ} - He was ashamed of his deed [xadʒila min fiʕlatihi]\\
2. \textarabic{الخَجَلُ يَمْنَعُهُ} - Shame prevents him [al-xadʒalu jamnaʕuhu]\\
3. \textarabic{تَصَرَّفَ بِخَجَلٍ} - He acted shamefully [tasˤarrafa bixadʒalin]
}} \\
\midrule \\
\textbf{Synonyms} & \textarabic{حَيَاءً} (modestly), \textarabic{خِزْياً} (disgracefully), \textarabic{عَاراً} (shamefully) \\
\textbf{Etymology} & From Proto-Semitic root meaning "to be ashamed, embarrassed" \\
\bottomrule

\end{tabular}

\subsection{\textarabic{مِنكَ} — [minka]}

\begin{tabular}{p{3cm}p{10cm}}
\toprule
\textbf{Translation} & from you / for you \\
\textbf{Root} & Preposition + pronoun \\
\textbf{Pattern} & \textarabic{مِنْ} + 2nd person masculine singular pronoun \\
\textbf{Grammar} & Preposition \textarabic{مِنْ} (from/of) + attached pronoun \textarabic{ـكَ} (you, masc.) \\
\midrule \\
\textbf{Examples} & \makecell[l]{\parbox{9.5cm}{
1. \textarabic{أَخَذْتُ مِنكَ كِتَاباً} - I took a book from you [ʔaxaðtu minka kitaːban]\\
2. \textarabic{تَعَلَّمْتُ مِنكِ الكَثِيرَ} - I learned much from you (fem.) [taʕallamtu minki l-kaθiːr]\\
3. \textarabic{هَذَا مِنَّا إِلَيْكُمْ} - This is from us to you (plural) [haːðaː minnaː ʔilajkum]
}} \\
\midrule \\
\textbf{Synonyms} & \textarabic{عَنكَ} (about you), \textarabic{لَكَ} (for you), \textarabic{بِكَ} (with/by you) \\
\textbf{Etymology} & \textarabic{مِنْ} from Proto-Semitic preposition meaning "from, of" \\
\bottomrule
\end{tabular}

% ======================== Phrase Analysis ========================
\section{Phrase Analysis}
\begin{tcolorbox}[colback=boxcolor,colframe=headercolor,breakable]
\textbf{Grammatical Structure:}\\
Sequential connector + past verb + proper noun subject + colon + active participle predicate + vocative particle + possessed noun + conjunction + emphatic particle with pronoun + present verb + subordinating particle with pronoun + negation particle + present subjunctive verb + relative pronoun + present verb with object pronoun + conjunction + prepositional phrase + past verb + adverbial accusative + prepositional phrase \\[0.5em]
\textbf{Key Grammar Points:}
\begin{itemize}
\item The prefix \textarabic{فَ} connects this statement to the previous narrative sequence
\item \textarabic{مُتَأَسِّفٌ} is a Form V active participle functioning as a predicate nominative
\item \textarabic{فَإِنِّي} combines causal conjunction with emphatic particle for strong assertion
\item \textarabic{لَن تَجِدَ} uses future negation requiring subjunctive mood
\item \textarabic{مَا تَسْرِقُهُ} is a relative clause with attached object pronoun
\item \textarabic{خَجَلاً} functions as an adverbial accusative expressing reason/manner
\item The phrase shows empathetic irony - apologizing to the thief for having nothing to steal
\item Form X verb \textarabic{اسْتَحْيَيْتُ} indicates reflexive emotional state
\end{itemize}
\end{tcolorbox}

% ======================== Similar Phrases for Practice ========================
\section{Similar Phrases for Practice}

\begin{enumerate}
\item \textarabic{فَقَالَ الرَّجُلُ: مُتَأَسِّفٌ يَا صَدِيقِي، فَإِنِّي أَعْرِفُ أَنَّكَ لَن تَجِدَ مَا تَبْحَثُ عَنْهُ}\\
So the man said: ``Sorry my friend, for I know that you won't find what you are looking for'' [faqaːla r-radʒulu: mutaʔassifun jaː sˤadiːqiː, faʔinniː ʔaʕrifu ʔannaka lan tadʒida maː tabħaθu ʕanhu]

\item \textarabic{وَقَالَتِ البِنْتُ: مُتَأَسِّفَةٌ يَا أُسْتَاذِي، فَإِنِّي أَعْلَمُ أَنَّكَ لَن تَجِدَ الجَوَابَ هُنَا}\\
And the girl said: ``Sorry professor, for I know that you won't find the answer here'' [waqaːlat al-bintu: mutaʔassifatun jaː ʔustaːðiː, faʔinniː ʔaʕlamu ʔannaka lan tadʒida l-dʒawaːba hunaː]

\item \textarabic{فَأَجَابَ الطَّالِبُ: مُعْتَذِرٌ يَا دُكْتُورُ، لَكِنِّي أَظُنُّ أَنَّكَ لَن تَجِدَ الكِتَابَ الآنَ}\\
So the student answered: ``Apologetic, doctor, but I think that you won't find the book now'' [faʔadʒaːba tˤ-tˤaːlibu: muʕtaðirun jaː duktuːr, laːkinniː ʔaðˤunnu ʔannaka lan tadʒida l-kitaːba l-ʔaːn]

\item \textarabic{وَقَالَ التَّاجِرُ: آسِفٌ يَا زَبُونُ، وَلَكِنِّي أَعْرِفُ أَنَّكَ لَن تَجِدَ هَذِهِ البِضَاعَةَ عِنْدِي}\\
And the merchant said: ``Sorry customer, but I know that you won't find this merchandise with me'' [waqaːla t-taːdʒiru: ʔaːsifun jaː zabuːn, walaːkinniː ʔaʕrifu ʔannaka lan tadʒida haːðihi l-bidˤaːʕata ʕindiː]
\end{enumerate}

% ======================== Levantine Dialect Version ========================
\section{Levantine (Shaami) Arabic Dialect}

\begin{tcolorbox}[colback=white,colframe=dialectcolor,title=\textbf{Levantine Version},breakable]
\textarabic{فَقَال جُحَا: آسِف يَا سِيدِي، لَأَنِّي بَعْرِف إِنَّك مِش رَح تِلَاقِي شِي تِسْرُقُه، وَلَهِيك اسْتَحَيت مِنَّك}\\
\textbf{Phonetic:} [faqaːl dʒuħaː: ʔaːsif jaː siːdiː, laʔanniː baʕrif innak miʃ raħ tilaːʔiː ʃiː tisruːqu, walaheːk istaħeːt minnak]\\
\textbf{Translation:} So Juha said: ``Sorry sir, because I know that you won't find anything to steal, and that's why I felt ashamed for you.''
\end{tcolorbox}
\textbf{Key Dialectal Changes:}
\begin{itemize}
\item \textarabic{مُتَأَسِّف} → \textarabic{آسِف} (ʔaːsif) — simplified form of ``sorry''
\item \textarabic{فَإِنِّي} → \textarabic{لَأَنِّي} (laʔanniː) — ``because I'' instead of ``for indeed I''
\item \textarabic{أَعْرِف} → \textarabic{بَعْرِف} (baʕrif) — present continuous with \textarabic{بـ} prefix
\item \textarabic{أَنَّك} → \textarabic{إِنَّك} (innak) — simplified subordinating particle
\item \textarabic{لَن} → \textarabic{مِش رَح} (miʃ raħ) — future negation with ``will not''
\item \textarabic{تَجِد} → \textarabic{تِلَاقِي} (tilaːʔiː) — dialectal verb ``to find''
\item \textarabic{مَا} → \textarabic{شِي} (ʃiː) — ``something/anything'' instead of relative ``what''
\item \textarabic{وَلِهَذَا} → \textarabic{وَلَهِيك} (walaheːk) — ``and that's why'' in dialect
\item \textarabic{اسْتَحْيَيْتُ} → \textarabic{اسْتَحَيت} (istaħeːt) — simplified past tense form
\item \textarabic{مِنكَ} → \textarabic{مِنَّك} (minnak) — emphatic form with gemination
\end{itemize}

% ======================== Additional Learning Notes ========================
\section{Additional Learning Notes}

\begin{tcolorbox}[colback=boxcolor,colframe=accentcolor,title=\textbf{Cultural and Literary Context},breakable]
\textbf{Juha's Character:} This phrase exemplifies Juha's paradoxical wisdom - he apologizes to a thief not for being robbed, but for disappointing the thief by having nothing worth stealing. This reversal of expectations is classic Juha humor.

\textbf{Social Politeness:} Even in this absurd situation, Juha maintains formal politeness markers (\textarabic{يَا سَيِّدِي}), showing how deeply embedded courtesy is in Arabic culture.

\textbf{Empathetic Irony:} The phrase demonstrates sophisticated Arabic literary technique - expressing genuine concern for someone whose intentions are harmful, creating both humor and moral commentary.

\textbf{Narrative Function:} This statement serves as the climax of the Juha story, where his unexpected response transforms a potentially threatening situation into a moment of absurd comedy.
\end{tcolorbox}

\subsection{Memory Tips}
\begin{enumerate}
\item \textbf{Emotional Progression:} Remember \textit{Apology → Explanation → Shame} (\textarabic{اعْتِذَار} →  \textarabic{تَفْسِير} → \textarabic{خَجَل})
\item \textbf{Particle Chain:} \textarabic{فَ...فَإِنِّي...وَلِهَذَا} creates logical flow (so → for indeed → and therefore)
\item \textbf{Form X Pattern:} \textarabic{اسْتَحْيَيْتُ} follows pattern \textarabic{اسْتَفْعَل} for self-directed emotions
\item \textbf{Negation Structure:} \textarabic{لَن + subjunctive} is standard future negation pattern
\item \textbf{Ironic Politeness:} The contrast between \textarabic{يَا سَيِّدِي} (respectful address) and \textarabic{تَسْرِق} (you steal) creates the humor
\end{enumerate}

\subsection{Related Vocabulary Family}
\begin{multicols}{2}

\textbf{Emotion family:}\\
\textarabic{أَسَف} — sorrow [ʔasaf]\\
\textarabic{خَجَل} — shame [xadʒal]\\
\textarabic{حَيَاء} — modesty [ħajaːʔ]\\
\textarabic{ارْتِبَاك} — embarrassment [irtabaːk]\\
\textarabic{نَدَم} — regret [nadam]\\

\textbf{Knowledge family:}\\
\textarabic{مَعْرِفَة} — knowledge [maʕrifa]\\
\textarabic{عِلْم} — science/knowledge [ʕilm]\\
\textarabic{دِرَايَة} — awareness [diraːja]\\
\textarabic{فَهْم} — understanding [fahm]\\
\textarabic{إِدْرَاك} — perception [ʔidraːk]\\

\textbf{Theft family:}\\
\textarabic{سَرِقَة} — theft [sariqa]\\
\textarabic{سَارِق} — thief [saːriq]\\
\textarabic{مَسْرُوق} — stolen (item) [masruːq]\\
\textarabic{نَهْب} — plundering [nahb]\\
\textarabic{اخْتِلَاس} — embezzlement [ixtilaːs]\\

\textbf{Politeness family:}\\
\textarabic{سَيِّد} — master/sir [sajjid]\\
\textarabic{أُسْتَاذ} — professor/teacher [ʔustaːð]\\
\textarabic{دُكْتُور} — doctor [duktuːr]\\
\textarabic{مَوْلَى} — master/lord [mawlaː]\\
\textarabic{صَاحِب} — owner/sir [sˤaːħib]
\end{multicols}

\end{document}