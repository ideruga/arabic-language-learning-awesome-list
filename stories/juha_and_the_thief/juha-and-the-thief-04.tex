\documentclass[a4paper,12pt]{article}

\usepackage[utf8]{inputenc}
\usepackage[english,arabic]{babel}
\usepackage{tikz}
\usepackage{geometry}
\usepackage{array}
\usepackage{multicol}
\usepackage[table]{xcolor}
\usepackage{polyglossia}
\setmainlanguage{english}
\setotherlanguage{arabic}
\setotherlanguage{hebrew}
\usepackage{arabtex}
\usepackage{fontspec}
\setmainfont{Charis SIL}
\newfontfamily\arabicfont[Script=Arabic,Scale=1.2]{Amiri}
\newfontfamily\hebrewfont{Ezra SIL}[Script=Hebrew]
\geometry{margin=1.5cm}
\usepackage[most]{tcolorbox}
\usepackage{booktabs}
\usepackage{graphicx}
\usepackage{makecell}

\usetikzlibrary{shapes,arrows,decorations.pathmorphing}


% Custom colors
\definecolor{headercolor}{RGB}{70,130,180}
\definecolor{boxcolor}{RGB}{240,248,255}
\definecolor{accentcolor}{RGB}{220,20,60}
\definecolor{tableheader}{RGB}{220,220,220}
\definecolor{dialectcolor}{RGB}{34,139,34}

\begin{document}

\title{\textbf{\Large Arabic Phrase Analyser}\\
\large Juha's Apology to the Thief\\
\normalsize \textarabic{فقال جحا: متأسف يا سيدي, فإني أعرف أنك لن يجد ما تسرقه, و لهذا استحيت خجلا منك}}
\author{Igor Deruga}
\date{}
\maketitle

% ======================== Phrase Display ========================
\begin{tcolorbox}[colback=boxcolor,colframe=headercolor,title=\textbf{Arabic Phrase},breakable]
\centering
\textarabic{فقال جحا: متأسف يا سيدي, فإني أعرف أنك لن يجد ما تسرقه, و لهذا استحيت خجلا منك}
\\[0.5em]
\textbf{Without Diacritics}
\\[1em]
\textarabic{فَقَالَ جُحَا: مُتَأَسِّفٌ يَا سَيِّدِي، فَإِنِّي أَعْرِفُ أَنَّكَ لَنْ تَجِدَ مَا تَسْرِقُهُ، وَلِهَذَا اسْتَحْيَيْتُ خَجَلاً مِنْكَ}
\\[0.5em]
\textbf{With Full Diacritics}
\end{tcolorbox}

% ======================== Translations ========================
\section{English Translation}
\begin{tcolorbox}[colback=white,colframe=accentcolor,breakable]
\textbf{Literal:} So said Juha: Sorry oh my master, for indeed I know that you will not find what to steal it, and for this I felt ashamed shame from you \\
\textit{[Arabic order retained for direct mapping]}\\[0.5em]
\textbf{Adapted:} So Juha said: ``I'm sorry, sir, for I know that you won't find anything to steal, and that's why I felt ashamed for you.''
\end{tcolorbox}

% ======================== Detailed Word Analysis ========================
\section{Detailed Word Analysis}

\subsection{\textarabic{فَقَالَ} — [faqaːla]}
\begin{tabular}{p{3cm}p{10cm}}
\toprule
\textbf{Translation} & so he said / then he said \\
\textbf{Root} & \textarabic{ق-و-ل} (q-w-l) \\
\textbf{Pattern} & \textarabic{فَعَلَ} (faʕala) \\
\textbf{Grammar} & Past tense verb, 3rd person masculine singular, with conjunction \textarabic{فَ} (fa-) prefix \\
\midrule \\
\textbf{Table of Conjugations} &  \makecell[l]{
Infinitive: \textarabic{قَوْل} [qawl] \\
Present: \textarabic{يَقُولُ} [jaquːlu] \\
Past: \textarabic{قَالَ} [qaːla]
} \\
\midrule
\textbf{Examples} & \makecell[l]{\parbox{9.5cm}{
1. \textarabic{قَالَ الرَّجُلُ الحَقَّ} - The man said the truth [qaːla r-radʒulu l-ħaqqa]\\
2. \textarabic{سَيَقُولُ لَكَ غَداً} - He will tell you tomorrow [sajaquːlu laka ɣadan]\\
3. \textarabic{قُلْ لِي مَاذَا حَدَثَ} - Tell me what happened [qul liː maːðaː ħadaθa]
} } \\
\midrule \\
\textbf{Synonyms} & \textarabic{تَكَلَّمَ} (spoke), \textarabic{تَحَدَّثَ} (talked), \textarabic{نَطَقَ} (uttered) \\
\textbf{Etymology} & From Proto-Semitic *qwl, related to Hebrew \texthebrew{קול} (qol) ``voice'' \\
\bottomrule
\end{tabular}

\subsection{\textarabic{جُحَا} – [dʒuħaː]}
\begin{tabular}{p{3cm}p{10cm}}
\toprule
\textbf{Translation} & Juha (proper name) \\
\textbf{Root} & N/A (proper noun) \\
\textbf{Pattern} & N/A \\
\textbf{Grammar} & Proper noun, masculine, nominative case (subject of \textarabic{قال}) \\
\textbf{Table of Conjugations} & N/A (proper noun - no conjugation) \\
\textbf{Examples} & \makecell[l]{\parbox{9.5cm}{
1. \textarabic{جُحَا رَجُلٌ حَكِيمٌ} - Juha is a wise man [dʒuħaː radʒulun ħakiːmun]\\
2. \textarabic{قِصَصُ جُحَا مَشْهُورَةٌ} - Juha's stories are famous [qisˤasˤu dʒuħaː maʃhuːratun]\\
3. \textarabic{ضَحِكَ جُحَا كَثِيراً} - Juha laughed a lot [dˤaħika dʒuħaː kaθiːran]
}} \\
\textbf{Synonyms} & N/A (proper name) \\
\textbf{Etymology} & Famous character in Arabic folklore, possibly from Turkish Nasreddin Hoca \\
\bottomrule
\end{tabular}

\subsection{\textarabic{مُتَأَسِّفٌ} — [mutaʔassifun]}
\begin{tabular}{p{3cm}p{10cm}}
\toprule
\textbf{Translation} & sorry / regretful \\
\textbf{Root} & \textarabic{أ-س-ف} (ʔ-s-f) \\
\textbf{Pattern} & \textarabic{مُتَفَعِّل} (mutafaʕʕil) - Form V active participle\\
\textbf{Grammar} & \makecell[l]{\parbox{9.5cm}{
	Active participle, masculine singular, nominative (predicate)
}} \\

\textbf{Table of Conjugations} & \makecell[l]{
Infinitive: \textarabic{تَأَسُّف} [taʔassuf] \\
Present: \textarabic{يَتَأَسَّفُ} [jataʔassafu] \\
Past: \textarabic{تَأَسَّفَ} [taʔassafa]
} \\
\midrule
\textbf{Examples} & \makecell[l]{\parbox{9.5cm}{
1. \textarabic{أَنَا مُتَأَسِّفٌ جِدّاً} - I am very sorry [ʔanaː mutaʔassifun dʒiddan]\\
2. \textarabic{تَأَسَّفَ عَلَى خَطَئِهِ} - He regretted his mistake [taʔassafa ʕalaː xatˤaʔihi]\\
3. \textarabic{سَيَتَأَسَّفُ لاحِقاً} - He will regret later [sajataʔassafu laːħiqan]
}} \\
\midrule
\textbf{Synonyms} & \textarabic{آسِف} (sorry), \textarabic{نَادِم} (remorseful), \textarabic{حَزِين} (sad) \\
\textbf{Etymology} & From root meaning ``grief, sorrow'' \\
\bottomrule
\end{tabular}

\subsection{\textarabic{يَا} — [jaː]}
\begin{tabular}{p{3cm}p{10cm}}
\toprule
\textbf{Translation} & O / oh (vocative particle) \\
\textbf{Root} & N/A (particle) \\
\textbf{Pattern} & N/A \\
\textbf{Grammar} & Vocative particle used for calling/addressing \\
\textbf{Table of Conjugations} & N/A (invariable particle) \\
\textbf{Examples} & \makecell[l]{\parbox{9.5cm}{
1. \textarabic{يَا أَخِي} - O my brother [jaː ʔaxiː]\\
2. \textarabic{يَا اللهُ} - O God [jaː ʔallaːhu]\\
3. \textarabic{يَا لَيْتَنِي} - Oh, if only I... [jaː lajtaniː]
}} \\
\midrule
\textbf{Synonyms} & \textarabic{أَيُّهَا} (O - formal), \textarabic{هَيَا} (come on) \\
\textbf{Etymology} & Ancient Semitic vocative particle \\
\bottomrule
\end{tabular}

\subsection{\textarabic{سَيِّدِي} — [sajjidiː]}
\begin{tabular}{p{3cm}p{10cm}}
\toprule
\textbf{Translation} & my master / my sir \\
\textbf{Root} & \textarabic{س-و-د} (s-w-d) \\
\textbf{Pattern} & \textarabic{فَعِّل} (faʕʕil) with 1st person possessive suffix \\
\textbf{Grammar} & Noun, masculine, genitive case (after vocative), with possessive pronoun \textarabic{ـي} \\
\textbf{Table of Conjugations} & \makecell[l]{
Singular: \textarabic{سَيِّد} [sajjid]\\
Plural: \textarabic{سَادَة} [saːda] or \textarabic{أَسْيَاد} [ʔasjaːd]
} \\
\textbf{Examples} & \makecell[l]{\parbox{9.5cm}{
1. \textarabic{هُوَ سَيِّدُ القَوْمِ} - He is the master of the people [huwa sajjidu l-qawmi]\\
2. \textarabic{أَطَعْتُ سَيِّدِي} - I obeyed my master [ʔatˤaʕtu sajjidiː]\\
3. \textarabic{السَّيِّدُ مُحَمَّد} - Mr. Muhammad [as-sajjidu muħammad]
}} \\
\midrule
\textbf{Synonyms} & \textarabic{مَوْلَى} (master), \textarabic{رَبّ} (lord), \textarabic{صَاحِب} (owner) \\
\textbf{Etymology} & From root meaning ``to be black, to prevail, to rule'' \\
\bottomrule
\end{tabular}

\subsection{\textarabic{فَإِنِّي} — [faʔinniː]}
\begin{tabular}{p{3cm}p{10cm}}
\toprule
\textbf{Translation} & for indeed I / for I am \\
\textbf{Root} & N/A (compound particle) \\
\textbf{Pattern} & N/A \\
\textbf{Grammar} & Conjunction \textarabic{فَ} + emphatic particle 1 + \textarabic{إِنَّ} +1st person pronoun suffix \\
\textbf{Table of Conjugations} & \makecell[l]{
\textarabic{إِنَّنِي / إِنِّي} (I)\\
\textarabic{إِنَّكَ} (you m.)\\
\textarabic{إِنَّهُ} (he)\\
\textarabic{إِنَّنَا} (we)
} \\
\textbf{Examples} & \makecell[l]{\parbox{9.5cm}{
1. \textarabic{إِنِّي أُحِبُّكَ} - Indeed I love you [ʔinniː ʔuħibbuka]\\
2. \textarabic{فَإِنَّهُ صَادِقٌ} - For indeed he is truthful [faʔinnahu sˤaːdiqun]\\
3. \textarabic{إِنَّنَا مُسْلِمُونَ} - Indeed we are Muslims [ʔinnanaː muslimuːna]
}} \\
\midrule
\textbf{Synonyms} & \textarabic{لَأَنَّنِي} (because I), \textarabic{حَقّاً أَنَا} (truly I) \\
\textbf{Etymology} & Classical Arabic emphatic construction \\
\bottomrule
\end{tabular}

\subsection{\textarabic{أَعْرِفُ} — [ʔaʕrifu]}
\begin{tabular}{p{3cm}p{10cm}}
\toprule
\textbf{Translation} & I know \\
\textbf{Root} & \textarabic{ع-ر-ف} (ʕ-r-f) \\
\textbf{Pattern} & \textarabic{فَعَلَ} (faʕala) \\
\textbf{Grammar} & Present tense verb, 1st person singular, Form I \\
\textbf{Table of Conjugations} & \makecell[l]{
Infinitive: \textarabic{مَعْرِفَة} [maʕrifa]\\
Present: \textarabic{أَعْرِفُ} [ʔaʕrifu]\\
Past: \textarabic{عَرَفْتُ} [ʕaraftu]
} \\
\textbf{Examples} & \makecell[l]{\parbox{9.5cm}{
1. \textarabic{عَرَفْتُ الحَقِيقَةَ} - I knew the truth [ʕaraftu l-ħaqiːqata]\\
2. \textarabic{يَعْرِفُ كُلَّ شَيْءٍ} - He knows everything [jaʕrifu kulla ʃajʔin]\\
3. \textarabic{سَأَعْرِفُ غَداً} - I will know tomorrow [saʔaʕrifu ɣadan]
}} \\
\midrule
\textbf{Synonyms} & \textarabic{عَلِمَ} (knew), \textarabic{دَرَى} (was aware), \textarabic{فَهِمَ} (understood) \\
\textbf{Etymology} & From root meaning ``to recognize, to know'' \\
\bottomrule
\end{tabular}

\subsection{\textarabic{أَنَّكَ} — [ʔannaka]}
\begin{tabular}{p{3cm}p{10cm}}
\toprule
\textbf{Translation} & that you \\
\textbf{Root} & N/A (particle with pronoun) \\
\textbf{Pattern} & N/A \\
\textbf{Grammar} & Subordinating particle 2 + \textarabic{أَنَّ}nd person masculine pronoun \\
\textbf{Table of Conjugations} & \makecell[l]{
\textarabic{أَنَّنِي} (that I)\\
\textarabic{أَنَّكَ} (that you m.)\\
\textarabic{أَنَّهُ} (that he)\\
\textarabic{أَنَّنَا} (that we)
} \\
\textbf{Examples} & \makecell[l]{\parbox{9.5cm}{
1. \textarabic{أَظُنُّ أَنَّكَ مُحِقٌّ} - I think that you are right [ʔaðˤunnu ʔannaka muħiqqun]\\
2. \textarabic{عَلِمْتُ أَنَّهُ قَادِمٌ} - I knew that he is coming [ʕalimtu ʔannahu qaːdimun]\\
3. \textarabic{قَالَ أَنَّنَا أَصْدِقَاء} - He said that we are friends [qaːla ʔannanaː ʔasˤdiqaːʔ]
}} \\
\midrule

\textbf{Synonyms} & \textarabic{بِأَنَّكَ} (that you - with preposition) \\
\textbf{Etymology} & Classical Arabic subordinating construction \\
\bottomrule
\end{tabular}

\subsection{\textarabic{لَنْ} — [lan]}
\begin{tabular}{p{3cm}p{10cm}}
\toprule
\textbf{Translation} & will not (negation particle for future) \\
\textbf{Root} & N/A (particle) \\
\textbf{Pattern} & N/A \\
\textbf{Grammar} & Negation particle that requires subjunctive mood \\
\textbf{Table of Conjugations} & N/A (invariable particle) \\
\textbf{Examples} & \makecell[l]{\parbox{9.5cm}{
1. \textarabic{لَنْ أَنْسَى} - I will not forget [lan ʔansaː]\\
2. \textarabic{لَنْ يَحْدُثَ هَذَا} - This will not happen [lan jaħduθa haːðaː]\\
3. \textarabic{لَنْ نَتَأَخَّرَ} - We will not be late [lan nataʔaxxara]
}} \\
\midrule
\textbf{Synonyms} & \textarabic{لا} (not - general), \textarabic{مَا} (not - past) \\
\textbf{Etymology} & Classical Arabic future negation \\
\bottomrule
\end{tabular}

\subsection{\textarabic{تَجِدَ} – [tadʒida]}
\begin{tabular}{p{3cm}p{10cm}}
\toprule
\textbf{Translation} & you find / you will find \\
\textbf{Root} & \textarabic{و-ج-د} (w-dʒ-d) \\
\textbf{Pattern} & \textarabic{فَعَلَ} (faʕala) \\
\textbf{Grammar} & Present tense verb, 2nd person masculine singular, subjunctive mood (after \textarabic{لن}) \\
\textbf{Table of Conjugations} & \makecell[l]{
Infinitive: \textarabic{وُجُود} [wuʤuːd]\\
Present: \textarabic{تَجِدُ} [taʤidu]\\
Past: \textarabic{وَجَدْتَ} [waʤadta]
} \\
\textbf{Examples} & \makecell[l]{\parbox{9.5cm}{
1. \textarabic{وَجَدْتُ الكِتَابَ} - I found the book [waʤadtu l-kitaːba]\\
2. \textarabic{سَتَجِدُ الطَّرِيقَ} - You will find the way [sataʤidu tˤ-tˤariːqa]\\
3. \textarabic{يَجِدُونَ صُعُوبَةً} - They find difficulty [jaʤiduːna sˤuʕuːbatan]
}} \\
\midrule
\textbf{Synonyms} & \textarabic{لَقِيَ} (encountered), \textarabic{عَثَرَ عَلَى} (came across) \\
\textbf{Etymology} & From root meaning ``to find, to exist'' \\
\bottomrule
\end{tabular}

\subsection{\textarabic{مَا} — [maː]}
\begin{tabular}{p{3cm}p{10cm}}
\toprule
\textbf{Translation} & what / that which \\
\textbf{Root} & N/A (relative pronoun) \\
\textbf{Pattern} & N/A \\
\textbf{Grammar} & Relative pronoun, indeclinable \\
\textbf{Table of Conjugations} & N/A (invariable) \\
\textbf{Examples} & \makecell[l]{\parbox{9.5cm}{
1. \textarabic{مَا تُرِيدُ} - What you want [maː turiːdu]\\
2. \textarabic{أَعْطِنِي مَا عِنْدَكَ} - Give me what you have [ʔaʕtˤiniː maː ʕindaka]\\
3. \textarabic{فَعَلْتُ مَا طَلَبْتَ} - I did what you asked [faʕaltu maː tˤalabta]
}} \\
\midrule
\textbf{Synonyms} & \textarabic{الَّذِي} (that which - definite), \textarabic{مَاذَا} (what - interrogative) \\
\textbf{Etymology} & Ancient Semitic interrogative/relative pronoun \\
\bottomrule
\end{tabular}

\subsection{\textarabic{تَسْرِقُهُ} — [tasriquhu]}
\begin{tabular}{p{3cm}p{10cm}}
\toprule
\textbf{Translation} & you steal it \\
\textbf{Root} & \textarabic{س-ر-ق} (s-r-q) \\
\textbf{Pattern} & \textarabic{فَعَلَ} (faʕala) \\
\textbf{Grammar} & Present tense verb, 2nd person masculine singular, with 3rd person masculine object pronoun \\
\textbf{Table of Conjugations} & \makecell[l]{
Infinitive: \textarabic{سَرِقَة} [sariqa] \\
Present: \textarabic{تَسْرِقُ} [tasriqu] \\
Past: \textarabic{سَرَقْتَ} [saraqta]
} \\
\textbf{Examples} & \makecell[l]{\parbox{9.5cm}{
1. \textarabic{سَرَقَ المَالَ} - He stole the money [saraqa l-maːla]\\
2. \textarabic{لا تَسْرِقْ} - Don't steal [laː tasriq]\\
3. \textarabic{سَيَسْرِقُونَ كُلَّ شَيْءٍ} - They will steal everything [sajasriquːna kulla ʃajʔin]
}} \\
\midrule
\textbf{Synonyms} & \textarabic{نَهَبَ} (plundered), \textarabic{اخْتَلَسَ} (embezzled), \textarabic{غَصَبَ} (usurped) \\
\textbf{Etymology} & From root meaning ``to steal, to rob'' \\
\bottomrule
\end{tabular}

\subsection{\textarabic{وَلِهَذَا} — [walihaːðaː]}
\begin{tabular}{p{3cm}p{10cm}}
\toprule
\textbf{Translation} & and for this / and therefore \\
\textbf{Root} & N/A (compound) \\
\textbf{Pattern} & N/A \\
\textbf{Grammar} & Conjunction \textarabic{وَ} + preposition \textarabic{لِ} + demonstrative pronoun \textarabic{هَذا} \\
\textbf{Table of Conjugations} & N/A \\
\textbf{Examples} & \makecell[l]{\parbox{9.5cm}{
1. \textarabic{وَلِهَذَا السَّبَبِ} - And for this reason [walihaːðaː s-sababi]\\
2. \textarabic{لِهَذَا جِئْتُ} - For this I came [lihaːðaː dʒiʔtu]\\
3. \textarabic{وَلِذَلِكَ} - And for that [waliðaːlika]
}} \\
\midrule
\textbf{Synonyms} & \textarabic{لِذَلِكَ} (therefore), \textarabic{وَبِالتَّالِي} (and consequently) \\
\textbf{Etymology} & Classical Arabic causal construction \\
\bottomrule
\end{tabular}

\subsection{\textarabic{اسْتَحْيَيْتُ} — [istaħjajtu]}
\begin{tabular}{p{3cm}p{10cm}}
\toprule
\textbf{Translation} & I felt ashamed / I was embarrassed \\
\textbf{Root} & \textarabic{ح-ي-ي} (ħ-j-j) \\
\textbf{Pattern} & \textarabic{اسْتَفْعَلَ} (istafʕala) - Form X \\
\textbf{Grammar} & Past tense verb, 1st person singular, Form X (seeking/reflexive) \\
\textbf{Table of Conjugations} & \makecell[l]{
Infinitive: \textarabic{اسْتِحْيَاء} [istiħjaːʔ] \\
Present: \textarabic{أَسْتَحْيِي} [ʔastaħjiː] \\
Past: \textarabic{اسْتَحْيَيْتُ} [istaħjajtu]
} \\
\textbf{Examples} & \makecell[l]{\parbox{9.5cm}{
1. \textarabic{اسْتَحْيَيْتُ مِنْ فِعْلِي} - I was ashamed of my action [istaħjajtu min fiʕliː]\\
2. \textarabic{يَسْتَحْيِي مِنَ الكَلامِ} - He is shy to speak [jastaħjiː mina l-kalaːmi]\\
3. \textarabic{لا تَسْتَحْيِ} - Don't be shy [laː tastaħjiː]
}} \\
\midrule
\textbf{Synonyms} & \textarabic{خَجِلَ} (was embarrassed), \textarabic{اسْتَحَى} (alternative form) \\
\textbf{Etymology} & From root meaning ``life, modesty, shame'' \\
\bottomrule
\end{tabular}

\subsection{\textarabic{خَجَلاً} – [xadʒalan]}
\begin{tabular}{p{3cm}p{10cm}}
\toprule
\textbf{Translation} & (out of) shame / embarrassment \\
\textbf{Root} & \textarabic{خ-ج-ل} (x-dʒ-l) \\
\textbf{Pattern} & \textarabic{فَعَل} (faʕal) \\
\textbf{Grammar} & Verbal noun (masdar), accusative case (adverbial/reason) \\
\textbf{Table of Conjugations} & N/A (verbal noun - no conjugation) \\
\textbf{Examples} & \makecell[l]{\parbox{9.5cm}{
1. \textarabic{اِحْمَرَّ خَجَلاً} - He turned red from embarrassment [iħmarra xadʒalan]\\
2. \textarabic{مَاتَ خَجَلاً} - He died of shame [maːta xadʒalan]\\
3. \textarabic{الخَجَلُ صِفَةٌ حَمِيدَةٌ} - Modesty is a good quality [al-xaʒalu sˤifatun ħamiːdatun]
}} \\
\midrule
\textbf{Synonyms} & \textarabic{حَيَاء} (modesty), \textarabic{اِرْتِبَاك} (confusion), \textarabic{حَرَج} (embarrassment) \\
\textbf{Etymology} & From root meaning ``to be ashamed, confused'' \\
\bottomrule
\end{tabular}

\subsection{\textarabic{مِنْكَ} — [minka]}
\begin{tabular}{p{3cm}p{10cm}}
\toprule
\textbf{Translation} & from you / for you (on your behalf) \\
\textbf{Root} & N/A (preposition with pronoun) \\
\textbf{Pattern} & N/A \\
\textbf{Grammar} & Preposition \textarabic{مِنْ} + 2nd person masculine pronoun suffix \\
\textbf{Table of Conjugations} & \makecell[l]{
\textarabic{مِنّي} (from me) \\
\textarabic{مِنكَ} (from you m.) \\
\textarabic{مِنهُ} (from him) \\
\textarabic{مِنّا} (from us)
} \\
\textbf{Examples} & \makecell[l]{\parbox{9.5cm}{
1. \textarabic{جِئتُ مِنكَ} - I came from you [dʒiʔtu minka]\\
2. \textarabic{أَخَذتُ مِنهُ} - I took from him [ʔaxaðtu minhu]\\
3. \textarabic{نَسمَعُ مِنكُم} - We hear from you (pl.) [nasmaʕu minkum]
}} \\
\midrule
\textbf{Synonyms} & \textarabic{عَنكَ} (from you - different context), \textarabic{لَكَ} (for you) \\
\textbf{Etymology} & Classical Arabic preposition with pronominal suffixes \\
\bottomrule
\end{tabular}

% ======================== Phrase Analysis ========================
\section{Phrase Analysis}
\begin{tcolorbox}[colback=boxcolor,colframe=headercolor,breakable]
\textbf{Grammatical Structure:}\\
Sequential connector + past verb + proper noun subject + colon + active participle predicate + vocative particle + possessed noun + conjunction + emphatic particle with pronoun + present verb + subordinating particle with pronoun + negation particle + present subjunctive verb + relative pronoun + present verb with object pronoun + conjunction + prepositional phrase + past verb + verbal noun (reason) + preposition with pronoun \\
\textbf{Key Grammar Points:}
\begin{itemize}
\item The prefix \textarabic{فَ} connects this statement to the previous narrative sequence
\item \textarabic{مُتَأَسِّفٌ} is a Form V active participle functioning as a predicate
\item \textarabic{فَإِنّي} combines causal conjunction with emphatic particle for strong assertion
\item \textarabic{لَن تَجِدَ} uses future negation requiring subjunctive mood
\item \textarabic{ما تَسرِقُهُ} is a relative clause with attached object pronoun
\item \textarabic{خَجَلاً} functions as an adverbial accusative expressing reason/manner
\item The phrase shows empathetic irony - apologizing to the thief for having nothing to steal
\item Form X verb \textarabic{استَحيَيتُ} indicates reflexive emotional state
\end{itemize}
\end{tcolorbox}

% ======================== Similar Phrases for Practice ========================
\section{Similar Phrases for Practice}

\begin{enumerate}
\item \textarabic{فَقالَ الرَّجُلُ: مُتَأَسِّفٌ يا صَديقي، فَإِنّي أَعرِفُ أَنَّكَ لَن تَجِدَ ما تَبحَثُ عَنهُ}\\
So the man said: ``Sorry my friend, for I know that you won't find what you are looking for'' [faqaːla r-radʒulu: mutaʔassifun jaː sˤadiːqiː, faʔinniː ʔaʕrifu ʔannaka lan tadʒida maː tabħaθu ʕanhu]

\item \textarabic{وَقالَت البِنتُ: مُتَأَسِّفَةٌ يا أُستاذي، فَإِنّي أَعلَمُ أَنَّكَ لَن تَجِدَ الجَوابَ هُنا}\\
And the girl said: ``Sorry professor, for I know that you won't find the answer here'' [waqaːlat al-bintu: mutaʔassifatun jaː ʔustaːðiː, faʔinniː ʔaʕlamu ʔannaka lan tadʒida l-dʒawaːba hunaː]

\item \textarabic{فَأَجابَ الطّالِبُ: مُعتَذِرٌ يا دُكتور، لَكِنّي أَظُنُّ أَنَّكَ لَن تَجِدَ الكِتابَ الآن}\\
So the student answered: ``Apologetic, doctor, but I think that you won't find the book now'' [faʔadʒaːba tˤ-tˤaːlibu: muʕtaðirun jaː duktuːr, laːkinniː ʔaðˤunnu ʔannaka lan tadʒida l-kitaːba l-ʔaːn]

\item \textarabic{وَقالَ التّاجِرُ: آسِفٌ يا زَبون، وَلَكِنّي أَعرِفُ أَنَّكَ لَن تَجِدَ هَذا البِضاعَةَ عِندي}\\
And the merchant said: ``Sorry customer, but I know that you won't find this merchandise with me'' [waqaːla t-taːdʒiru: ʔaːsifun jaː zabuːn, walaːkinniː ʔaʕrifu ʔannaka lan tadʒida haːða l-bidˤaːʕata ʕindiː]
\end{enumerate}

% ======================== Levantine Dialect Version ========================
\section{Levantine (Shaami) Arabic Dialect}

\begin{tcolorbox}[colback=white,colframe=dialectcolor,title=\textbf{Levantine Version},breakable]
\textarabic{فَقال جُحا: آسِف يا سيّدي، لأنّي بَعرِف إنّك مش رَح تِلاقي شي تِسرُقُه، وَلَهيك استَحَيت مِنّك}\\
\textbf{Phonetic:} [faqaːl dʒuħaː: ʔaːsif jaː sajjiːdiː, laʔanniː baʕrif innak miʃ raħ tilaːʔiː ʃiː tisruːqu, walaheːk istaħeːt minnak]\\
\textbf{Translation:} So Juha said: ``Sorry sir, because I know that you won't find anything to steal, and that's why I felt ashamed for you.''
\end{tcolorbox}
\textbf{Key Dialectal Changes:}
\begin{itemize}
\item \textarabic{مُتَأَسِّف} → \textarabic{آسِف} (ʔaːsif) – simplified form of ``sorry''
\item \textarabic{فَإِنّي} → \textarabic{لأنّي} (laʔanniː) – ``because I'' instead of ``for indeed I''
\item \textarabic{أَعرِف} → \textarabic{بَعرِف} (baʕrif) – present continuous with \textarabic{بـ} prefix
\item \textarabic{أَنَّك} → \textarabic{إنّك} (innak) – simplified subordinating particle
\item \textarabic{لَن} → \textarabic{مش رَح} (miʃ raħ) – future negation with ``will not''
\item \textarabic{تَجِد} → \textarabic{تِلاقي} (tilaːʔiː) – dialectal verb ``to find''
\item \textarabic{ما} → \textarabic{شي} (ʃiː) – ``something/anything'' instead of relative ``what''
\item \textarabic{وَلِهَذا} → \textarabic{وَلَهيك} (walaheːk) – ``and that's why'' in dialect
\item \textarabic{استَحيَيت} → \textarabic{استَحَيت} (istaħeːt) – simplified past tense form
\end{itemize}

% ======================== Additional Learning Notes ========================
\section{Additional Learning Notes}

\begin{tcolorbox}[colback=boxcolor,colframe=accentcolor,title=\textbf{Cultural and Literary Context},breakable]
\textbf{Juha's Character:} This phrase exemplifies Juha's paradoxical wisdom - he apologizes to a thief not for being robbed, but for disappointing the thief by having nothing worth stealing. This reversal of expectations is classic Juha humor.

\textbf{Social Politeness:} Even in this absurd situation, Juha maintains formal politeness markers (\textarabic{يا سَيِّدي}), showing how deeply embedded courtesy is in Arabic culture.

\textbf{Empathetic Irony:} The phrase demonstrates sophisticated Arabic literary technique - expressing genuine concern for someone whose intentions are harmful, creating both humor and moral commentary.

\textbf{Narrative Function:} This statement serves as the climax of the Juha story, where his unexpected response transforms a potentially threatening situation into a moment of absurd comedy.
\end{tcolorbox}

\subsection{Memory Tips}
\begin{enumerate}
\item \textbf{Emotional Progression:} Remember \textit{Apology → Explanation → Shame} (\textarabic{اعتِذار} →  \textarabic{تَفسير} → \textarabic{خَجَل})
\item \textbf{Particle Chain:} \textarabic{فَ...فَإِنّي...وَلِهَذا} creates logical flow (so → for indeed → and therefore)
\item \textbf{Form X Pattern:} \textarabic{استَحيَيت} follows pattern \textarabic{استَفعَل} for self-directed emotions
\item \textbf{Negation Structure:} \textarabic{لَن + subjunctive} is standard future negation pattern
\item \textbf{Ironic Politeness:} The contrast between \textarabic{يا سَيِّدي} (respectful address) and \textarabic{تَسرِق} (you steal) creates the humor
\end{enumerate}

\subsection{Related Vocabulary Family}
\begin{multicols}{2}

\textbf{Emotion family:}\\
\textarabic{أَسَف} – sorrow [ʔasaf]\\
\textarabic{خَجَل} – shame [xadʒal]\\
\textarabic{حَياء} – modesty [ħajaːʔ]\\
\textarabic{ارتِباك} – embarrassment [irtabaːk]\\
\textarabic{نَدَم} – regret [nadam]\\

\textbf{Knowledge family:}\\
\textarabic{مَعرِفَة} – knowledge [maʕrifa]\\
\textarabic{عِلم} – science/knowledge [ʕilm]\\
\textarabic{دِراية} – awareness [diraːja]\\
\textarabic{فَهم} – understanding [fahm]\\
\textarabic{إدراك} – perception [ʔidraːk]\\

\textbf{Theft family:}\\
\textarabic{سَرِقَة} – theft [sariqa]\\
\textarabic{سارِق} – thief [saːriq]\\
\textarabic{مَسروق} – stolen (item) [masruːq]\\
\textarabic{نَهب} – plundering [nahb]\\
\textarabic{اختِلاس} – embezzlement [ixtilaːs]\\

\textbf{Politeness family:}\\
\textarabic{سَيِّد} – master/sir [sajjid]\\
\textarabic{أُستاذ} – professor/teacher [ʔustaːð]\\
\textarabic{دُكتور} – doctor [duktuːr]\\
\textarabic{مَولى} – master/lord [mawlaː]\\
\textarabic{صاحِب} – owner/sir [sˤaːħib]
\end{multicols}

\end{document}