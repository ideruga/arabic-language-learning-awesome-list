\documentclass[letter,12pt]{article}


\usepackage[most]{tcolorbox}
\usepackage{longtable}
\usepackage{makecell}
\usepackage[utf8]{inputenc}
\usepackage[english,arabic]{babel}
\usepackage{tikz}
\usepackage{geometry}
\usepackage{array}
\usepackage{multicol}
\usepackage[table]{xcolor}
\usepackage{polyglossia}
\setmainlanguage{english}
\setotherlanguage{arabic}
\setotherlanguage{hebrew}
\usepackage{arabtex}
\usepackage{fontspec}
\setmainfont{Charis SIL}
\newfontfamily\arabicfont[Script=Arabic,Scale=1.2]{Amiri}
\newfontfamily\hebrewfont{Ezra SIL}[Script=Hebrew]
\geometry{margin=1.5cm}
\usepackage{booktabs}
\usepackage{graphicx}
\usepackage{multirow}
\usepackage{fancyhdr}
\pagestyle{fancy}
\fancyhf{}
\fancyhead[R]{\textit{<Title - a short summary of the phrase in English>}} % Right header with document title


\usetikzlibrary{shapes,arrows,decorations.pathmorphing}


% Custom colors
\definecolor{headercolor}{RGB}{70,130,180}
\definecolor{boxcolor}{RGB}{240,248,255}
\definecolor{accentcolor}{RGB}{220,20,60}
\definecolor{tableheader}{RGB}{220,220,220}
\definecolor{dialectcolor}{RGB}{34,139,34}

\begin{document}


\title{\textbf{\Large Arabic Phrase Analyser}\\
\large <Title - a short summary of the phrase in English>\\
\normalsize \textarabic{<arabic phrase to analyse>}}
\author{Igor Deruga}
\date{}
\maketitle

% ======================== Phrase Display ========================
\begin{tcolorbox}[colback=boxcolor,colframe=headercolor,title=\textbf{Arabic Phrase},breakable]
\centering
\textarabic{<original arabic phrase>}
\\[0.5em]
\textbf{Without Diacritics}
\\[1em]
\textarabic{<arabic phrase with full diactricts>}
\\[0.5em]
\textbf{With Full Diacritics}
\end{tcolorbox}

% ======================== Translations ========================
\section{English Translation}
\begin{tcolorbox}[colback=white,colframe=accentcolor,breakable]
\textbf{Literal:} <Literal word-by-word phrase translation into english> \\
\textit{[Arabic order retained for direct mapping]}\\[0.5em]
\textbf{Adapted:} <Adapted phrase translation into english>"
\end{tcolorbox}

% ======================== Detailed Word Analysis ========================
\section{Detailed Word Analysis}

\subsection{\textarabic{فَقَالَ} — [faqaːla]}
\begin{tabular}{p{3cm}p{10cm}}
\toprule

\textbf{Translation} & so he said / then he said \\
\textbf{Root} & \textarabic{ق-و-ل} (q-w-l) \\
\textbf{Pattern} & \textarabic{فَعَلَ} (faʕala) \\
\textbf{Grammar} & Past tense verb, 3rd person masculine singular, with conjunction \textarabic{فَ} (fa-) prefix \\
\midrule \\
\textbf{Examples} & \makecell[l]{\parbox{9.5cm}{
1. \textarabic{قَالَ الرَّجُلُ الحَقَّ} - The man said the truth [qaːla r-radʒulu l-ħaqqa]\\
2. \textarabic{سَيَقُولُ لَكَ غَداً} - He will tell you tomorrow [sajaquːlu laka ɣadan]\\
3. \textarabic{قُلْ لِي مَاذَا حَدَثَ} - Tell me what happened [qul liː maːðaː ħadaθa]
} } \\
\midrule \\
\textbf{Synonyms} & \textarabic{تَكَلَّمَ} (spoke), \textarabic{تَحَدَّثَ} (talked), \textarabic{نَطَقَ} (uttered) \\
\textbf{Etymology} & From Proto-Semitic *qwl, related to Hebrew \texthebrew{קול} (qol) ``voice'' \\
\bottomrule
\end{tabular}

\subsection{Conjugation}
\large \textbf{Infinitive}: \textarabic{قَالَ}

\begin{longtable}{|>{\raggedright}p{3.5cm}|p{5cm}|p{5cm}|}
\hline
\textbf{Person} & \textbf{Perfect (Past)} & \textbf{Imperfect (Present)} \\
\hline
\textbf{3rd person masculine singular} & \textarabic{قَالَ} [qāla] & \textarabic{يَقُولُ} [yaqūlu] \\
\hline
\textbf{3rd person feminine singular} & \textarabic{قَالَتْ} [qālat] & \textarabic{تَقُولُ} [taqūlu] \\
\hline
\textbf{3rd person masculine dual} & \textarabic{قَالَا} [qālā] & \textarabic{يَقُولَانِ} [yaqūlāni] \\
\hline
\textbf{3rd person feminine dual} & \textarabic{قَالَتَا} [qālatā] & \textarabic{تَقُولَانِ} [taqūlāni] \\
\hline
\textbf{3rd person masculine plural} & \textarabic{قَالُوا} [qālū] & \textarabic{يَقُولُونَ} [yaqūlūna] \\
\hline
\textbf{3rd person feminine plural} & \textarabic{قُلْنَ} [qulna] & \textarabic{يَقُلْنَ} [yaqulna] \\
\hline
\textbf{2nd person masculine singular} & \textarabic{قُلْتَ} [qulta] & \textarabic{تَقُولُ} [taqūlu] \\
\hline
\textbf{2nd person feminine singular} & \textarabic{قُلْتِ} [qulti] & \textarabic{تَقُولِينَ} [taqūlīna] \\
\hline
\textbf{2nd person dual (m./f.)} & \textarabic{قُلْتُمَا} [qultumā] & \textarabic{تَقُولَانِ} [taqūlāni] \\
\hline
\textbf{2nd person masculine plural} & \textarabic{قُلْتُمْ} [qultum] & \textarabic{تَقُولُونَ} [taqūlūna] \\
\hline
\textbf{2nd person feminine plural} & \textarabic{قُلْتُنَّ} [qultunna] & \textarabic{تَقُلْنَ} [taqulna] \\
\hline
\textbf{1st person singular} & \textarabic{قُلْتُ} [qultu] & \textarabic{أَقُولُ} [aqūlu] \\
\hline
\textbf{1st person plural} & \textarabic{قُلْنَا} [qulnā] & \textarabic{نَقُولُ} [naqūlu] \\
\hline
\end{longtable}

\subsubsection*{Conjugation Notes}
\begin{itemize}
  \item The \textbf{future} is formed with the prefix \textarabic{سَ} [sa-] or \textarabic{سَوْفَ} [sawfa] before the imperfect (e.g., \textarabic{سَيَقُولُ} “he will say”).
  \item The \textbf{moods of the imperfect}: 
    \begin{itemize}
      \item Indicative: \textarabic{يَقُولُ} [yaqūlu] 
      \item Subjunctive: \textarabic{لَنْ يَقُولَ} [lan yaqūla]
      \item Jussive: \textarabic{لَمْ يَقُلْ} [lam yaqul]
      \item Imperative: \textarabic{قُلْ} [qul!]
    \end{itemize}
  \item \textbf{Passive voice}: 
    \begin{itemize}
      \item Perfect: \textarabic{قِيلَ} [qīla] – it was said
      \item Imperfect: \textarabic{يُقَالُ} [yuqālu] – it is said
    \end{itemize}
\end{itemize}

\subsection{\textarabic{سَيِّدِي} — [sajjidiː]}

\begin{tabular}{p{3cm}p{10cm}}
\toprule
\textbf{Translation} & my master / my sir \\
\textbf{Root} & \textarabic{س-و-د} (s-w-d) \\
\textbf{Pattern} & \textarabic{فَعِّل} (faʕʕil) with 1st person possessive suffix \\
\textbf{Grammar} & Noun, masculine, genitive case (after vocative), with possessive pronoun \textarabic{ـي} \\
\textbf{Table of Conjugations} & \makecell[l]{
Singular: \textarabic{سَيِّد} [sajjid]\\
Plural: \textarabic{سَادَة} [saːda] or \textarabic{أَسْيَاد} [ʔasjaːd]
} \\
\textbf{Examples} & \makecell[l]{\parbox{9.5cm}{
1. \textarabic{هُوَ سَيِّدُ القَوْمِ} - He is the master of the people [huwa sajjidu l-qawmi]\\
2. \textarabic{أَطَعْتُ سَيِّدِي} - I obeyed my master [ʔatˤaʕtu sajjidiː]\\
3. \textarabic{السَّيِّدُ مُحَمَّد} - Mr. Muhammad [as-sajjidu muħammad]
}} \\
\midrule
\textbf{Synonyms} & \textarabic{مَوْلَى} (master), \textarabic{رَبّ} (lord), \textarabic{صَاحِب} (owner) \\
\textbf{Etymology} & From root meaning ``to be black, to prevail, to rule'' \\
\bottomrule
\end{tabular}

\subsubsection*{Full Declension Matrix}

\begin{tabular}{|c|c|c|c|c|}
\hline
\textbf{Number} & \textbf{Case} & \textbf{Indefinite} & \textbf{Definite} & \textbf{With Suffix (1st sg.)} \\
\hline
\multirow{3}{*}{Singular (\textarabic{مفرد})} 
 & Nominative   & \textarabic{سَيِّدٌ} (sajjidun) & \textarabic{السَّيِّدُ} (as-sajjidu) & \textarabic{سَيِّدِي} (sajjidī) \\
 & Accusative   & \textarabic{سَيِّدًا} (sajjidan) & \textarabic{السَّيِّدَ} (as-sajjida) & \textarabic{سَيِّدِيَ} (sajjidiyya) \\
 & Genitive     & \textarabic{سَيِّدٍ} (sajjidin) & \textarabic{السَّيِّدِ} (as-sajjidi) & \textarabic{سَيِّدِي} (sajjidī) \\
\hline
\multirow{3}{*}{Dual (\textarabic{مثنّى})} 
 & Nominative   & \textarabic{سَيِّدَانِ} (sajjidāni) & \textarabic{السَّيِّدَانِ} (as-sajjidāni) & \textarabic{سَيِّدَايَ} (sajjidāya) \\
 & Acc/Gen      & \textarabic{سَيِّدَيْنِ} (sajjidaini) & \textarabic{السَّيِّدَيْنِ} (as-sajjidaini) & \textarabic{سَيِّدَيَّ} (sajjidayya) \\
\hline
\multirow{6}{*}{Plural (\textarabic{جمع})} 
  & \multirow{2}{*}{Nominative}    & \textarabic{سَادَةٌ} (sādah) & \textarabic{السَّادَةُ} (as-sādatu) & \textarabic{سَادَتِي} (sādatī) \\
  &                                                   & \textarabic{أَسْيَادٌ} (ʔasyādun) & \textarabic{الأَسْيَادُ} (al-ʔasyādu) & \textarabic{أَسْيَادِي} (ʔasyādī) \\
  & \multirow{2}{*}{Accusative}      & \textarabic{سَادَةً} (sādatan) & \textarabic{السَّادَةَ} (as-sādata) & \textarabic{سَادَتِيَ} (sādatīya) \\
  &                                                    & \textarabic{أَسْيَادًا} (ʔasyādan) & \textarabic{الأَسْيَادَ} (al-ʔasyāda) & \textarabic{أَسْيَادِيَ} (ʔasyādiyya) \\
  & \multirow{2}{*}{Genitive}          & \textarabic{سَادَةٍ} (sādatin) & \textarabic{السَّادَةِ} (as-sādati) & \textarabic{سَادَتِي} (sādatī) \\
  &                                                    & \textarabic{أَسْيَادٍ} (ʔasyādin) & \textarabic{الأَسْيَادِ} (al-ʔasyādi) & \textarabic{أَسْيَادِي} (ʔasyādī) \\
\hline
\end{tabular}

% ======================== Phrase Analysis ========================
\section{Phrase Analysis}
\begin{tcolorbox}[colback=boxcolor,colframe=headercolor,breakable]
\textbf{Grammatical Structure:}\\
Sequential connector + past verb + proper noun subject + colon + active participle predicate + vocative particle + possessed noun + conjunction + emphatic particle with pronoun + present verb + subordinating particle with pronoun + negation particle + present subjunctive verb + relative pronoun + present verb with object pronoun + conjunction + prepositional phrase + past verb + verbal noun (reason) + preposition with pronoun \\
\textbf{Key Grammar Points:}
\begin{itemize}
\item The prefix \textarabic{فَ} connects this statement to the previous narrative sequence
\item \textarabic{مُتَأَسِّفٌ} is a Form V active participle functioning as a predicate
\item \textarabic{فَإِنّي} combines causal conjunction with emphatic particle for strong assertion
\item \textarabic{لَن تَجِدَ} uses future negation requiring subjunctive mood
\item \textarabic{ما تَسرِقُهُ} is a relative clause with attached object pronoun
\item \textarabic{خَجَلاً} functions as an adverbial accusative expressing reason/manner
\item The phrase shows empathetic irony - apologizing to the thief for having nothing to steal
\item Form X verb \textarabic{استَحيَيتُ} indicates reflexive emotional state
\end{itemize}
\end{tcolorbox}

% ======================== Similar Phrases for Practice ========================
\section{Similar Phrases for Practice}

\begin{enumerate}
\item \textarabic{فَقالَ الرَّجُلُ: مُتَأَسِّفٌ يا صَديقي، فَإِنّي أَعرِفُ أَنَّكَ لَن تَجِدَ ما تَبحَثُ عَنهُ}\\
So the man said: ``Sorry my friend, for I know that you won't find what you are looking for'' [faqaːla r-radʒulu: mutaʔassifun jaː sˤadiːqiː, faʔinniː ʔaʕrifu ʔannaka lan tadʒida maː tabħaθu ʕanhu]

\item \textarabic{وَقالَت البِنتُ: مُتَأَسِّفَةٌ يا أُستاذي، فَإِنّي أَعلَمُ أَنَّكَ لَن تَجِدَ الجَوابَ هُنا}\\
And the girl said: ``Sorry professor, for I know that you won't find the answer here'' [waqaːlat al-bintu: mutaʔassifatun jaː ʔustaːðiː, faʔinniː ʔaʕlamu ʔannaka lan tadʒida l-dʒawaːba hunaː]

\item \textarabic{فَأَجابَ الطّالِبُ: مُعتَذِرٌ يا دُكتور، لَكِنّي أَظُنُّ أَنَّكَ لَن تَجِدَ الكِتابَ الآن}\\
So the student answered: ``Apologetic, doctor, but I think that you won't find the book now'' [faʔadʒaːba tˤ-tˤaːlibu: muʕtaðirun jaː duktuːr, laːkinniː ʔaðˤunnu ʔannaka lan tadʒida l-kitaːba l-ʔaːn]

\item \textarabic{وَقالَ التّاجِرُ: آسِفٌ يا زَبون، وَلَكِنّي أَعرِفُ أَنَّكَ لَن تَجِدَ هَذا البِضاعَةَ عِندي}\\
And the merchant said: ``Sorry customer, but I know that you won't find this merchandise with me'' [waqaːla t-taːdʒiru: ʔaːsifun jaː zabuːn, walaːkinniː ʔaʕrifu ʔannaka lan tadʒida haːða l-bidˤaːʕata ʕindiː]
\end{enumerate}

% ======================== Levantine Dialect Version ========================
\section{Levantine (Shaami) Arabic Dialect}

\begin{tcolorbox}[colback=white,colframe=dialectcolor,title=\textbf{Levantine Version},breakable]
\textarabic{فَقال جُحا: آسِف يا سيّدي، لأنّي بَعرِف إنّك مش رَح تِلاقي شي تِسرُقُه، وَلَهيك استَحَيت مِنّك}\\
\textbf{Phonetic:} [faqaːl dʒuħaː: ʔaːsif jaː sajjiːdiː, laʔanniː baʕrif innak miʃ raħ tilaːʔiː ʃiː tisruːqu, walaheːk istaħeːt minnak]\\
\textbf{Translation:} So Juha said: ``Sorry sir, because I know that you won't find anything to steal, and that's why I felt ashamed for you.''
\end{tcolorbox}
\textbf{Key Dialectal Changes:}
\begin{itemize}
\item \textarabic{مُتَأَسِّف} → \textarabic{آسِف} (ʔaːsif) – simplified form of ``sorry''
\item \textarabic{فَإِنّي} → \textarabic{لأنّي} (laʔanniː) – ``because I'' instead of ``for indeed I''
\item \textarabic{أَعرِف} → \textarabic{بَعرِف} (baʕrif) – present continuous with \textarabic{بـ} prefix
\item \textarabic{أَنَّك} → \textarabic{إنّك} (innak) – simplified subordinating particle
\item \textarabic{لَن} → \textarabic{مش رَح} (miʃ raħ) – future negation with ``will not''
\item \textarabic{تَجِد} → \textarabic{تِلاقي} (tilaːʔiː) – dialectal verb ``to find''
\item \textarabic{ما} → \textarabic{شي} (ʃiː) – ``something/anything'' instead of relative ``what''
\item \textarabic{وَلِهَذا} → \textarabic{وَلَهيك} (walaheːk) – ``and that's why'' in dialect
\item \textarabic{استَحيَيت} → \textarabic{استَحَيت} (istaħeːt) – simplified past tense form
\end{itemize}

% ======================== Additional Learning Notes ========================
\section{Additional Learning Notes}

\begin{tcolorbox}[colback=boxcolor,colframe=accentcolor,title=\textbf{Cultural and Literary Context},breakable]
\textbf{Juha's Character:} This phrase exemplifies Juha's paradoxical wisdom - he apologizes to a thief not for being robbed, but for disappointing the thief by having nothing worth stealing. This reversal of expectations is classic Juha humor.

\textbf{Social Politeness:} Even in this absurd situation, Juha maintains formal politeness markers (\textarabic{يا سَيِّدي}), showing how deeply embedded courtesy is in Arabic culture.

\textbf{Empathetic Irony:} The phrase demonstrates sophisticated Arabic literary technique - expressing genuine concern for someone whose intentions are harmful, creating both humor and moral commentary.

\textbf{Narrative Function:} This statement serves as the climax of the Juha story, where his unexpected response transforms a potentially threatening situation into a moment of absurd comedy.
\end{tcolorbox}

\subsection{Memory Tips}
\begin{enumerate}
\item \textbf{Emotional Progression:} Remember \textit{Apology → Explanation → Shame} (\textarabic{اعتِذار} →  \textarabic{تَفسير} → \textarabic{خَجَل})
\item \textbf{Particle Chain:} \textarabic{فَ...فَإِنّي...وَلِهَذا} creates logical flow (so → for indeed → and therefore)
\item \textbf{Form X Pattern:} \textarabic{استَحيَيت} follows pattern \textarabic{استَفعَل} for self-directed emotions
\item \textbf{Negation Structure:} \textarabic{لَن + subjunctive} is standard future negation pattern
\item \textbf{Ironic Politeness:} The contrast between \textarabic{يا سَيِّدي} (respectful address) and \textarabic{تَسرِق} (you steal) creates the humor
\end{enumerate}

\subsection{Related Vocabulary Family}
\begin{multicols}{2}

\textbf{Emotion family:}\\
\textarabic{أَسَف} – sorrow [ʔasaf]\\
\textarabic{خَجَل} – shame [xadʒal]\\
\textarabic{حَياء} – modesty [ħajaːʔ]\\
\textarabic{ارتِباك} – embarrassment [irtabaːk]\\
\textarabic{نَدَم} – regret [nadam]\\

\textbf{Knowledge family:}\\
\textarabic{مَعرِفَة} – knowledge [maʕrifa]\\
\textarabic{عِلم} – science/knowledge [ʕilm]\\
\textarabic{دِراية} – awareness [diraːja]\\
\textarabic{فَهم} – understanding [fahm]\\
\textarabic{إدراك} – perception [ʔidraːk]\\

\textbf{Theft family:}\\
\textarabic{سَرِقَة} – theft [sariqa]\\
\textarabic{سارِق} – thief [saːriq]\\
\textarabic{مَسروق} – stolen (item) [masruːq]\\
\textarabic{نَهب} – plundering [nahb]\\
\textarabic{اختِلاس} – embezzlement [ixtilaːs]\\

\textbf{Politeness family:}\\
\textarabic{سَيِّد} – master/sir [sajjid]\\
\textarabic{أُستاذ} – professor/teacher [ʔustaːð]\\
\textarabic{دُكتور} – doctor [duktuːr]\\
\textarabic{مَولى} – master/lord [mawlaː]\\
\textarabic{صاحِب} – owner/sir [sˤaːħib]
\end{multicols}

\end{document}